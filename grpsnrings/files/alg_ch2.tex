\documentclass[../Main.tex]{subfiles}

\begin{document}

\chapter{Groups}
\section{Basix}
\defn{A group $(G,\cdot)$}{A group consists of a set and a binary relation $\cdot:G \times G \to G$ (which makes it closed by definition) such that:
\begin{enumerate}
    \item $\forall$ $a,b,c \in G$, $(a\cdot b)\cdot c=a \cdot (b\cdot c)$ (Associative)
    \item There exists an element $e\in G$ called identity so that for every $a \in G$ we have $a\cdot e=e\cdot a=a$
    \item For every element $a$ in $G$ we have another element $a^{-1}$ so that $aa^{-1}=a^{-1}a=e$ 
\end{enumerate}
A way to remember group axioms is to remember ASCII: \textbf{AS}sociative, \textbf{C}losed, \textbf{I}dentity, and \textbf{I}nverse
}


\exm{Some group examples:}{ $\ZZ $ with the usual addition, with $0$ as identity. Inverse being $-a$.
\\\\ $\ZZ / n\ZZ$ with the modular addition, with identity being $\overline{0}$ and inverse being $\overline{-a}$.
\\\\ In fact $ \ZZ, \QQ, \RR, \CC $ are groups with respective addition, identity being $0$ and inverse being $-a$.
\\\\ $\RR^+, \CC-\{0\}, \RR-\{0\}$, etc. are groups with multiplication as the operation. Here identity is $1$, and inverse is $\frac{1}{a} $.
\\\\ $\ZZ / n\ZZ*$, the set of all congruence classes in $\ZZ/ n\ZZ$ which have a multiplicative inverse (or equivalently, those that have gcd with $n$ as $1$) forms a group under multiplication. The identity is $\overline{1}$ and the inverse is that $\overline{c}$, which was shown to exist, such that $\overline{a} \cdot \overline{c}=\overline{1}$. 

}

\defn{Direct Product}{If $(A, !) \text{ and } (B,*)$ are each groups, then we define the \textbf{Direct Product} as the group formed by $A \times B:= \{(a,b): a\in A, b \in B \}$ with the operation $\&:(A \times B)\times(A \times B) \to A \times B$ defined by $(a_1,b_1)\&(a_2,b_2)=(a_1 ! a_2, b_1 * b_2)$}

\prop{If $G, \cdot$ is a group, then the following hold:
\begin{enumerate}
    \item The identity element $e$ is unique.
    \item for every $a \in G$, the inverse element $a^{-1}$ is unique
    \item $(a \cdot b)^{-1}=b^{-1} \cdot a^{-1}$
    \item For any $a_1,a_2,\dots a_n \in G$, the expression $a_1 \cdot a_2 \cdot \dots \cdot a_n$ is independent of how it is bracketed.

\end{enumerate}}
\pf{(1) Suppose the identity is not unique, i.e, there exists $e_1$ and $e_2$ so that it obeys identity axioms. We have $a \cdot e=e\cdot a=a$, which means $(e_1)e_2=e_2(e_1)=e_2$, treating $e_2$ as true identity. But also, $(e_2)e_1=e_1(e_2)=e_1, =e_2$. Hence we see easily that $e_1=e_2$.
\\\\ (2) Suppose two inverses $x$ and $y$ exist. $ax=e$, which means $yax=ye=y$, but from associativity, $(ya)x=x=y$. Hence, $x=y:=a^{-1}$
\\\\ (3) $a \cdot b (a \cdot b)^{-1}=e$ which implies
$a^{-1}a \cdot b(a \cdot b)^{-1}=a^{-1} \implies $
$b^{-1}(a^{-1}a) \cdot b(a \cdot b)^{-1}=b^{-1}a^{-1} $ which directly gives $(a\cdot b)^{-1}=b^{-1}a^{-1}$ 
\\\\ (4) (\textbf{PEDANTIC PROOF AHEAD, SKIP IF NOT A PEDANT}) For just one element $a_1$, there is no need to even check. Assume that the bracketing does not change the meaning for any consequetive $n$ operations. Consider 
$$a_1 \cdot a_2 \cdot a_3 \cdots a_n \cdot a_{n+1}$$
First look at the bracketing 
$$\{(a_1 \cdot a_2 \cdot a_3 \cdots a_n )\}\cdot (a_{n+1})$$
From induction hypothesis, no bracketing inside the $\{ \}$ affects the operations. Next, consider the kind
$$\{(a_1 \cdot a_2 \cdot a_3 \cdots)\} (a_n \cdot a_{n+1})$$
Again, from induction, no bracketing affects the operations. By means of reverse induction, we show that no bracketing affects the end result of these operations.
}

\prop{Let $G$ be a group and let $a,b$ be elements in the group. Then the equations $ax=b$ and $ya=b$ have unique solutions. Explicitly, we have the left and right cancellation laws:
\\\\ If $au=av, $ then $u=v$
\\\\ If $ub=vb, $ then $u=v$

}
\pf{If $au=av$, we multiply both sides by $a^{-1}$ to preserve equality $u=v$. Similarly, we multiply $b^{-1}$ to either side of the equation $ub=vb$ which gives $u=b$}

\defn{Order of an element $g$ in a group $G$}{We say an element $g$ in $G$ is of \emph{order} $n \in \NN$ if $n$ is the smallest natural number so that $g^n=g \cdot g \cdots g=e$, the identity. We denote this as $O(g)$.}
\defn{Order of a Group $G$, denoted by $|G|$.}{The cardinality of the group.}

\thmp{}{If $G$ is a group and $a$ an element in $G$ with $O(a)=n$, then $a^m=1$ if and only if $n|m$}{
$\implies$) Given $O(a)=n$ we have $n$ to be the smallest natural number so that $a^n=1$. If we have that $a^m=1$, and $n \not{|} m$, then $m=qn+r$ where $0<r<n$. Therefore, $a^r \neq 1$. We have that $a^{qn+r}=a^{qn} \cdot a^r=a^r \neq 0$ which is absurd.
\\\\ $\impliedby$) Given $n | m$, obviously then $a^m=1$. 
}

\thmp{}{If $O(a)=n$, then $O(a^m)=\frac{n}{gcd(m,n)}$.}{We understand that $\frac{n}{gcd(m,n)}$ is atleast a candidate, since we can see clearly that $(a^m)^{\frac{n}{gcd(m,n)}}=(a^n)^{\frac{m}{gcd(m,n)}}=1$.
Suppose $k$ is the order, with $k < \frac{n}{gcd(m,n)}$ so that $a^{mk}=1$. From the previous theorem, we see that $n | mk$. i.e, $n\delta=mk \implies 
\frac{n}{gcd(m,n)}\delta=\frac{m}{gcd(m,n)}k$. Note that $\frac{n}{(m.n)}$ and $\frac{m}{(m,n)}$ share no common divisors, for if they did, then that, multiplied with the actual gcd would yield a divisor larger than the gcd. Hence, $gcd(\frac{n}{(m,n)} ,\frac{m}{(m,n)})=1$. This means, from previous lemmas, that $\frac{n}{(m,n)}$ divides $k$. This is, ofcourse, absurd.

}

\thmp{Real Numbers $mod(1)$}{Let $G:=\{x \in \RR: 0 \leq x<1\}$. Define $x \circ y= \{x+y\}$ where $\{\cdot \}$ denotes the fractional part (and $[\cdot ]$ denotes the integral part, or the GIF). Then, $G$ is an abelian group under $\{\circ \}$}{Closure of $x \circ y$ is pretty obvious. We freely use $\{\cdot \}$, $frac\{\cdot\}$ and $\underline{\cdot}$ interchangibly. We consider
$x \circ (y\circ z)=frac(\underline{x}+[x] +frac(y+z) )$=$frac(\underline{x}+[x] +frac(\underline{y}+[y]+\underline{z}+[z]))=frac(\underline{x}+frac(\underline{y}+\underline{z}))$=
$frac(\underline{x}+(\underline{y}+\underline{z})-[\underline{y}+\underline{z}])=frac(\underline{x}+\underline{y}+\underline{z})$
\\\\ Now consider $(x \circ y)\circ z=frac(frac(\underline{x}+\underline{y})+\underline{z}+[z])$
$=frac(frac(\underline{x}+\underline{y})+\underline{z})$ $= frac((\underline{x}+\underline{y})-[\underline{x}+\underline{y}]+\underline{z}+[z])=frac(\underline{x}+\underline{y}+\underline{z})$. Hence we see $\circ$ is associative. Trivial to note that the idenity element is $\underline{0}$ and the inverse for every $\underline{x}$ is $\underline{-x}$.

}

\thmp{Group of the $n$-th roots of unity}{Suppose $G:=\{z \in \CC: z^n=1: \text{ for some } n\}$}{We want to solve $z^n=1$. Applying polar coordinates we have $|z|^n(cis(\theta))^n=1$. Taking mod gives us $|z|=1$. We have to solve for, then, $cis(theta)^n=1$. It is simple computation to see that $cis(\theta)^n=cis(n\theta)$ which gives us $cis(n\theta)=1$. The solutions to this are $\theta=\frac{2\pi k}{n}$ for any integer $k$. Therefore, the solutions to $z^n=1$ are of the form $z=cis(\frac{2k \pi}{n})$. We assume a modulo $2\pi$ structure, i.e, we classify solutions of the kind $\theta+2k\pi$ in the class of $\theta$. We see then, that for $k \leq n-1$, each solution is unique. If we let $\omega=cis(\frac{2\pi}{n})$. We see that all the other elements are generated by $\omega$ since for $k=2$, we just have $\omega^2$ (from the way cis powers work). Till $k=n-1$, we have unique solutions generated by $\omega$ given by $1, \omega, \omega^2 \cdots \omega^{n-1}$. We see that when $k=n$ we get $\theta=\frac{2\pi n}{n}=2\pi \equiv 0 mod(2\pi)$. For $n+j$ where $j<n$, we see that $\theta=\frac{2\pi(n+j)}{n} =2\pi+ \frac{2\pi j}{n} \equiv \frac{2\pi j}{n} mod(2\pi)$. Hence, all the unique solutions are $1, \omega, \omega^2 \cdots \omega^{n-1}$.
\\\\ To see that this is a group under multiplication, we note that $\omega^x (\omega^y \omega^z)=(\omega^x \omega^y) \omega^z =\omega^{(x+y+z) mod(n)}$. Every element has an inverse since $\omega ^j \cdot \omega^{n-j}=1$ ($1$ is the identity here since $1 \omega^j=\omega^j \cdot 1=\omega^j$)
\\\\ $G$, though a group under multiplication, is not one under addition. For example, consider $1+0i \in G$. $1+1=2+0i$ which is not in $G$.}

\fact{If $a, b \in G$, then $|ab|=|ba|$}
\pf{We have $(ab)(ab) \cdots (ab)=(ab)^n=e$. Rearranging the brackets we get $a(ba)(ba)\cdots (b)=a(ba)^{n-1}(b)=e$ which gives $(ba)^{n-1}=a^{-1}b^{-1}=(ba)^{-1}$ which eventually gives $(ba)^{n}=e$. Therefore, if $m$ was the order of $ba$, then $m|n$. Similarly we can re-run the argument in the other direction starting with $(ba)^m=e$ to get $n|m$. This gives $n=m$.

}
\fact{If $x^2=1$ for every $x \in G$, then $G$ is abelian}
\pf{Let $ab \neq ba \implies a^2b=b \neq a(ba)$. This implies $b^2=e\neq (ba)^2 \implies 1 \neq 1$. Absurd.}

\fact{Any finite group of even order contains an element $a$ with order $2$.}
\pf{Suppose that for every non-identity element $x$ we have $o(x)=p \neq 2$ with $p\geq 3$. We can then notice that for every element, $x \neq x^{-1}$. Hence, every element along with its inverses would form an even sized set (due to uniqueness of inverses, none overlap). Hence, adding identity to this would make the group odd.}

\exm{$G=\{1,a,b,c\}$ is $|G|=4$ with $1$ identity. Say no element has order $4$. Then this group has a unique multiplication table}{We can immediately fill up the initial parts:
\begin{center}

\begin{tabular}{ c c c c c }
x & 1 & a & b & c \\
1 & 1 & a & b & c \\
a & a & x & x & x \\
b & b & x & x & x \\
c & c & x & x & x \\
\end{tabular}
\end{center}
Since this is a finite group of order $4$, there should be atleast one element with order $2$. We WLOG select that element to be $b$ so that $b^2=1$. Is $ab=a$ or $b$? Nope, since that would make either one identity. So $ab=c$. Is $ba=a$ or $b$? In much the same way, we conclude $ba=ab=c$.
$b(ba)=bc=a$ and $(ab)b=ab^2=cb$. Hence $bc=cb=a$. So far we got: (This is applicable for any group of size $4$, since we did not use the property that this group has no element with order $4$.)

\begin{center}

    \begin{tabular}{ c c c c c }
    x & 1 & a & b & c \\
    1 & 1 & a & b & c \\
    a & a & x & c & x \\
    b & b & c & 1 & a \\
    c & c & x & a & x \\
    \end{tabular}
    \end{center}
(\textbf{The Klein Route}) Is $a^2=b$? Can't be, because then, since $b^2=1$, we'd have $a^4=1$ which is against hypothesis. Hence $a^2=1, $ or $a^2=c$. Likewise, we can conclude that $c^2=1$ or $c^2=a$ (Ask the same questions, is $c^2=b$? No). Suppose $a^2=1$ and $c^2=a$. That would make $c^4=1$, which is against hypothesis. Hence, if $a^2=1$ then $c^2=1$ as well. Likewise, if $c^2=1$, then $a^2=1$ as well. Suppose neither, i.e, $c^2=a$ and $a^2=c$. Then $c^4=a^2=c$ and $a^4=c^2=a$. We have $a^3=1$ and $c^3=1$. $(ba)a^2=b$ which means $ca^2=b \implies c^2=b$. But $c^2=a$. Absurd. Hence, this scenario is impossible. Hence, for the Klein route, $a^2=c^2=1$. 
\\\\ Question for $ac$ and $ca$, then arises. Is $ac=1$? That would mean $a^2c=1c=a$, absurd. Hence, $ac=b$. Similarly, is $ca=1$? we would then have $c=a$ again. Therefore, $ac=ca=b$. This completes the Klein Route: 
\begin{center}

    \begin{tabular}{ c c c c c }
    \textbf{x} & \textbf{1} & \textbf{a} & \textbf{b} & \textbf{c} \\
    \textbf{1} & 1 & a & b & c \\
    \textbf{a} & a & 1 & c & b \\
    \textbf{b} & b & c & 1 & a \\
    \textbf{c} & c & b & a & 1 \\
    \end{tabular}
\end{center}
(\textbf{The $\mathbf{\ZZ/4\ZZ}$ Route}) Suppose that $G$ has an element of order $4$. Since the size of the cyclic subgroup of this element is $4$ as well, this group is cyclic. WLOG, assume that $G=\langle a \rangle$. Then every element is $1$, $a=a$, $a^2=b$, $a^3=c$. We have (for a general 4 membered group)

\begin{center}

    \begin{tabular}{ c c c c c }
    x & 1 & a & b & c \\
    1 & 1 & a & b & c \\
    a & a & x & c & x \\
    b & b & c & 1 & a \\
    c & c & x & a & x \\
    \end{tabular}
\end{center}

Since the group is cyclic, we can immediately write $a^2=b$. Since $a^3=c$, $a^6=a^2=c^2=b$. We can write that in as well. All that is left is $ac$ and $ca$. Let us rule out the obvious: $ac \neq a$, $ca \neq a$, $ac \neq c$, $ca \neq q$. Is $ac=b$? That would mean $a^4=b$, which makes $b=1$. Same way, $ca \neq b$. Hence, $ac$ and $ca$ have only one option left, $1$. We can fill that in to get the $\ZZ/ 4\ZZ$ isomorph:


\begin{center}

    \begin{tabular}{ c c c c c }
    x & 1 & a & b & c \\
    1 & 1 & a & b & c \\
    a & a & b & c & 1 \\
    b & b & c & 1 & a \\
    c & c & 1 & a & b \\
    \end{tabular}
\end{center} 
Note that Klein is the unique 4 membered group with no element of order $4$. $\ZZ/4\ZZ$ isomorph is the unique group with one element with order $4$. 


}

\defn{Subgroup}{A set $H \subseteq G$ of group $G$ is said to be a subgroup if $H$ is itself a group, i.e, follows ASCII axioms under the operation inherited from $G$. If $H$ is a proper subgroup of $G$, then we denote it by $H<G$. Else, $H \leq G$}

\defn{Cyclic Subgroup}{Suppose $G,\cdot $ is a group, with an element $a$. Suppose $<a>$ is a subgroup of $G$ that contains $a$. Must definitely have $e$ which is notated to be $a^0$. It must then definitely have $a \cdot a$, $a\cdot a \cdot a$ and so on till $a^n$ where $o(a)=n$. If no order exists, we take it to be $\forall n \in \ZZ$. $<a>:=\{a^n: n \in \ZZ \}$ This is enough for it to be a group:
\\\\ $e=a^0$ is in the group. For every $b$, i.e, $a^k$ in the group, $a^{-k}$ is also in the group by definition. It obeys ASCII.
\\\\ \textbf{Fact:} $<a>$ is the smallest subgroup of $G$ containing $a$. Analogous to \textit{span}.}
\exm{Some groups cyclically generated}{$\ZZ/ n\ZZ$ as an additive group is generated by $1$. That is, $<1>$ is precisely $\ZZ /n\ZZ$.
\\\\n-th roots of unity: $1, \omega, \omega^2 \cdots \omega^{n-1}$, is generated by $<\omega>$. }
\fact{If $O(a)=n <\infty$ for $a \in G$ and $G=\langle a \rangle$, then $|G|=n$}
\thmp{}{Suppose $G=\langle a \rangle$ with $O(a)=n<\infty$, then $\langle a^j \rangle=G$ if and only if $gcd(j,n)=1$}{$\implies$) Since $O(a)=n$, the order of $a^j$ is given by $n/gcd(j,n)$. If $gcd(j,n)\neq 1$, then clearly the orders are different, implying the groups they generate will be of different cardinality. 
\\\\ $\impliedby$) Suppose $gcd(j,n)=1$ with $O(a)=n$ and $G=\langle a \rangle$. Then $O(a^j)=n$. Note that $\langle a^j \rangle \leq \langle a \rangle$ since every element of the former is in the latter. But the order of each is the same, whilst being finite. Therefore, $\langle a^j \rangle =\langle a \rangle$}

\exm{An application of the previous theorem to $\ZZ/ n\ZZ$}{We know that $\langle 1\rangle =\ZZ/n \ZZ$ under addition. Order of $1$ is $n$ here. Consider another element $j \in \ZZ/ n\ZZ$ so that $gcd(j,n)=1$. Then order of $j$ is $n$. As such, $\langle j \rangle = \ZZ/ n\ZZ$. All the elements of $\ZZ /n\ZZ$ that generate $\ZZ/ n\ZZ$ belong to the multiplicative $\ZZ/ n\ZZ^*$ group. }
\cor{The number of generators for a cyclic group of order $n$ is $\phi(n)$.}

\thmp{}{Subgroup of a cyclic group is cyclic.}{
Let $G=\langle a \rangle$. Suppose $H \leq G=\langle a \rangle $ is the subgroup of $G$. 
\\\\ Say $e, a^{j_1}, a^{j_2} \cdots a^{j_n} \cdots $ are in $H$. Case (1), if there exists a finite subcollection of these indices so that their $gcd$ is $1$. Let them be $j_1,j_2 \cdots j_n$. This means $gcd(j_1,j_2 \cdots j_n)=1$ and from generalised bezout, we have $x_1j_1+x_2j_2 \cdots x_nj_n=1$ whence we see that $H$ has to be $G$ necessarily.
\\\\ The other case, case (2) is that for every finite subcollection of $\{j_1,j_2 \cdots \}$, their $gcd$ is not $1$. Does this mean that $gcd(j_1,j_2 \cdots (\text{till } \infty))$ is not $1$? i.e, do they all share one common factor? Suppose there exists $j_1'$ and $j_2'$ so that they do not share a common factor. This would mean that $gcd(j_1',j_2')=1$, which contradicts the hypothesis of case (2). Hence, in this case, every $j$ is a multiple of some number $\gamma$ which makes $H= \langle a^{\gamma} \rangle$.
\\\\ \textbf{Alt Proof:(Similar)} Let $m$ be the smallest index so that $a^m \in H$. We claim $a^m$ is the cyclic generator of $H$. Suppose $a^n$ where $n>m$ is in the group $H$. Then $a^n=a^{mq+r}$ where $0 \leq r<m$. This means $a^n \cdot (a^m)^{-q}=a^r$. By virtue of being a group which is closed, we see that $a^r \in H$. If $r \neq 0$, we get a contradiction. Hence, $r=0$. Therefore, every element is $(a^m)^{\text{something}}$.}
\corp{If $G$ is a cyclic group generated by $a$ and a subgroup has two elements $a^j$ and $a^k$, then this subgroup would necessarily have to be the bigger group $G$ if $(j,k)=1$. }{Let $G=\langle a \rangle:=\{a^n: n \in \ZZ\}$ where $a \in G$ (the generator of $G$). Cosider a $H$ subgroup of $G$, given by elements $a^j: j \in \{n_1,n_2, \cdots\}$ where $n_1, n_2, \cdots$ is a sequence of integers. Note that, since $a^{n_1}$ is in $H$, $(a^{q(n_1)})$ for $q \in \ZZ$ is also in $H$. Suppose that there exists $n_{j} $ and $n_k$ indices so that $gcd(n_j,n_k)=1$. This means that $xn_j+yn_k=1$. Hence, $(a^{n_j})^{x} (a^{n_k})^y=a$ Which would make $a^{xn_j+yn_k}$ the cyclic generator of $G$ itself, which would force $H$ to become $G$.}{}
\exm{}{Consider $G=\ZZ/ n \ZZ=\langle 1 \rangle$. Consider a subgroup that is known to contain $2$ and $3$. In notation, $3=1^3$ and $2=1^2$, and $gcd(3,2)=1$. This means that This subgroup must be $\ZZ/ n \ZZ$ itself.
 }
\exm{}{Consider $\ZZ /n\ZZ =\langle 1 \rangle$. Let a subgroup be such that it contains $2,4$ and $3,6$. The gcd(2,4)=2$\neq 1$ and the gcd(3,6)=3 $\neq 1$, but gcd(2,1)=1. This means that this subgroup must necessarily be the main group.}

\lemp{}{If a cyclic group is infinite, then every subgroup is infinite (except the trivial subgroup)}{Suppose $G=\langle a \rangle$ that is infinite. i.e, $a$ has no order. Consider a subgroup that is non trivial, i.e, has an element $a^j, j \neq 0$. If this group is of finite order, then $a^j$ must be of finite order, obviously. $O(a^j)=q$ which means $a^{qj}=1$ which is absurd. }
\lemp{}{A cyclic group is of prime order if and only if it has no non trivial proper subgroups.}{$\implies$) if $|G|=p$ where $p$ is a prime, then the only subgroups (which are cyclic from the previous theorems) for this group are itself and identity, for the order of any subgroup divides the order of the group (because it is cyclic).
\\\\ $\impliedby$) Suppose the only subgroups of $G$ are the trivial one and itself. Suppose $G$ is of a non prime order $q$. Then, if $m | q $, there is an element $a^{q/m}$, and its corresponding generated set, that is a proper subgroup of $G$. Absurd.nan }
\newpage
\subsection{The Dihedral Group $D_{2n}$}
Given an $n-gon$ that is regular, we define the symmetries on it by permutation maps or bijective maps from $\{1,2,3\cdots,n\}$ into itself. 

\defn{Rotation $r$}{$r:\{1,2,3 \cdots n\} \to \{1,2,\cdots,n \}$ is defined as $$1 \overset{\mbox{r}}{\longrightarrow} 2$$
$$2 \overset{\mbox{r}}{\longrightarrow} 3$$
$$ \vdots$$

$$n-1 \overset{\mbox{r}}{\longrightarrow} n$$

$$n \overset{\mbox{r}}{\longrightarrow} 1$$

Whose inverse is, as one can guess:
$$2 \overset{\mbox{inverse(r)}}{\longrightarrow} 1$$
$$3 \overset{\mbox{inverse(r)}}{\longrightarrow} 2$$
$$ \vdots$$

$$n \overset{\mbox{inverse(r)}}{\longrightarrow} n-1$$

$$1 \overset{\mbox{inverse(r)}}{\longrightarrow} n$$
}

\defn{Symmetry, or flipping, or mirror whatever}{$s$ is defined as $s:\{1 \cdots n\} \to \{1 \cdots n\}$ as follows:
$$1 \overset{s}\mapsto 1 $$
$$2 \overset{s}\mapsto n $$
$$3 \overset{s}\mapsto n-1 $$
$$\vdots$$
$$n \overset{s}\mapsto 2 $$
Note that, $s^2=1$}

\textbf{Some Properties of $D_{2n}$}
The symmetries of $D_{2n}$ are the functions listed above. Note the following:
\begin{enumerate}
    \item $1$, $r$, $\cdots r^{n-1}$ form distinct elements. $|r|=n$ since $r^n=1$
    \item $r$ follows $\ZZ /n \ZZ$ structure in that, $r^{j}$ has, as its inverse, $r^{n-j}$. It obeys similar modular structure.
    \item $s^2=1$
    \item $rs=sr^{-1}$. Note that $rs$ amounts to "Pivoting" about $2$ and flipping the dihedron, which can be achieved by reverse rotating, i.e, $r^-1$ first, and then flipping, i.e $sr^{-1}$. Hence, $rs=sr^{-1}$.
    \item Since the inverse elements of $r^i$ are $r^{-1}$, the previous result can be more generally written as $(r^i)s=sr^{-i}$. In a spoon feedy way we see that $rs=sr^{-1}$ $\implies$ $r(rs)=r^2s=r(sr^{-1})=(rs)(r^{-1})=(sr^{-1}r^{-1})=sr^{-2}$. Keep going as such.
    \item The elements $1,r,r^2, \cdots r^{n-1}$ constitute the subgroup of rotations, each one corresponding to a rotation of $\frac{2j\pi}{n}$.
    \item The elements $s, rs, r^2s \cdots r^{n-1}s$ correspond to "pivoting" the $j-th$ number and flipping about that. These on their own dont constitute a group for, $(r^ns)(r^ms)=r^n(sr^m)s=r^n(r^{-m})$ which falls into the rotation group. 
    \item Note that $s \neq r^{i}$ for any $i$. This ought to be intuitively clear.
    \item $sr^{i}\neq sr^{j}$ since flipping about different pivots achieves a different structure, one that is different by rotations alone (obviously).
    \item The set $\{1,r,r^2 \cdots r^{n-1}; s; rs,r^2s, \cdots r^{n-1}s\}$ Constitutes a group, of order $2n$. This is stated formally in the next theorem, with proof.  
\end{enumerate}

\thmp{}{The set $\{1,r,r^2 \cdots r^{n-1}; s; rs,r^2s, \cdots r^{n-1}s\}$ Constitutes a group, of order $2n$. }{We note that $1,r,r^2, \cdots r^{n-1}$ all obey ASCII. So does $s$, since it is self inverse(The identity here is the identity function). Consider the permutations of the kind $r^js$. These have inverses as well, for if we compose this with $r^{n-j}$, we would have $r^{n-j} \circ (r^j s)=s$. If we compose this still, with $s$, we get $1$. The total composition on $r^j s$ would have been $sr^{n-j}$. Infact, these elements too are self inverses. Easier way to see this is $(r^i s)(r^i s)=r^i(sr^i)s=r^i(r^{-i}s)s=1$. These also, then follow ASCII. }
\subsection{More basix, Homomorphisms, isomorphisms, centers}
\defn{Homomorphism}{Let $\langle G, \cdot \rangle$ and $\langle H, *\rangle$ be two groups. We say a function $\phi: G \to H$ is a \textbf{homomorphism} if $\forall x,y \in G$, $\phi(x \cdot y)=\phi(x)*\phi(y)$. 
\\\\ Some notable features of a homomorphism are:
\begin{enumerate}
    \item $\phi(e_G)=e_H$
    \item $\phi(a^{-1})=\phi(a)^{-1}$
\end{enumerate}

}
\defn{Group Isomorphism}{A homomorphism from $\langle G, \cdot \rangle$ to $\langle H,* \rangle$ is a group isomorphism if it is bijective.}
\thmp{}{Let $\langle G, \cdot \rangle$ be a group. Consider $*:G \times G \to G$ a binary operation defined as $$a*b=b \cdot a$$. Then this is a group isomorphism from $G \cdot$ to $G, *$.}{Consider $\phi:G \to G$ given by $\phi(a)=a^{-1}$. This is a bijection since every $a$ maps to a unique $a^{-1}$, and vice versa. Consider $\phi(a\cdot b)=b^{-1} \cdot a^{-1}=\phi(b) \cdot \phi(a) = \phi(a)*\phi(b)$, which makes $\phi$ a homomorphism, hence, an isomorphism.}

\defn{Centralizer of $a \in G$}{\textbf{Centralizer} of an element $a$ in group $G$ is defined as $$H_a:=\{x \in G: xa=ax \}$$ or, the set of all elements in $G$ that commute with $a$. }
\lemp{}{Centralizer of $a \in G$ is a subgroup of $G$}{$e$ is obviously in $H_a$. Suppose some $b \in H_a$, i.e, $ab=ba$. Consider $abb^{-1}=a=bab^{-1} \implies b^{-1}a=ab^{-1}$ which means that if $b \in H_a$, $b^{-1}$ is also in $H_a$. That it is closed and associative is also obvious (since $ba=ab$ and $ca=cb$ would mean $bca=abc$).}
\defn{Centralizer of a subset $S \subset G$}{\textbf{Centralizer of a set} $S$ in $G$ is defined as $$H_S:=\{x \in G: xz=zx \forall z \in S \}$$}
\lemp{}{Centralizer of a set $S \subset G$ is a subgroup of $G$}{Again, obviously $e$ is in $H_S$. Let $b \in H_S$, i.e, $bx=xb, \forall x \in S$. $b^{-1}bx=x=b^{-1}xb \implies xb^{-1}=b^{-1}xbb^{-1}$ which gives $xb^{-1}=b^{-1}x, \forall x \in S$. Hence, $b^{-1} \in H_S$. Suppose $a,b \in H_S$, i.e, $ax=xa, \forall x \in S$ and $bx=xb, \forall x \in S$. $a(bx)=a(xb)=(ax)b=x(ab)$ which makes $ab \in H_S$.}
\defn{Center of a group $G$}{\textbf{The center of a group} $G$ is defined as the centralizer of $G$, i.e $$H_G:=\{x\in G: xz=zx, \forall z \in G\} $$
by the previous lemma, this is also a group. }
\lemp{}{Center of a group $G$ is an abelian subgroup.}{$H_G:=\{x \in G: xz=zx, \forall z \in G\}$ is easily seen to be a group. Consider $k_1, l_2 \in H_G$, and consider $k_1 \cdot k_2 \in H_G$ (which exists in $H_G$ due to closure). Treating $k_1$ as an element in $H_G$ and $k_2$ as an element in $G$, we note that by definition, $k_1(k_2)=k_2k_1$. Hence, the group is abelian.}
\defn{Normal Subgroup}{A subgroup $H \leq G$ is said to be \emph{normal} if for every element $h \in H$ and for every element $g \in G$, $ghg^{-1} \in H$. This essentially says (see next notation) that $\forall g \in G$, $gHg^{-1} \subseteq H$. But we can also say equivalently that $\forall g \in G$, $gHg^{-1}=H$ by the following argument:
\\\\ Let $g$ be arbitrary. We ask the question, given any $h \in H$, can it be written in the form $gh'g^{-1}$ for some $h' \in H$? For that would guarentee the back inclusion. $h \in H$ implies $h^{-1} \in H$ which implies $gh^{-1}g^{-1} \in H$ which implies $g^{-1}hg \in H$ for any $h \in H$. This means $g^{-1}hg=h'$ for some $h'$ which gives $h=gh'g^{-1}$ which gives us the back inclusion.}
\defn{Notation: $gH$, $Hg$ and $gHg^{-1}$ for a set $H$}{\begin{enumerate}
\item $gH:=\{gh: h \in H \}$
\item $Hg:=\{hg: h \in H\}$
\item $gHg^{-1}:=\{ghg^{-1}: h \in H\}$
\end{enumerate} }
\defn{Normalizer}{The normalizer $N_G(S):=\{z \in G: zSz^{-1}=S\}$ which is subtle, as it means not only that for every $z \in N_G(S)$ it is that for every $s \in S$, $zsz^{-1} \in S$, but also another condition that for every $s \in S$, there exists $s' \in S$ so that $s=gs' g^{-1}$. 
\\\\ We denote the normalizer of a set $S$ as $N_G(S)$ which tells us with respect to what group we are normalizing $S$. }
\rmkb{Note that instead of stating the back inclusion as "$\forall s \in S, \exists s'$ so that $s=xs'x^{-1}$" we can also equivalently say $\forall s \in S$, $x^{-1}sx\in S$. }
\thmp{}{Normalizer $N_G(S)$ of a set $S$ is a group}{Consider $N_G(S):=\{g \in G: gSg^{-1}=G\}$. $e$ clearly belongs in $N_G$ since $ese^{-1}=s$ and for every $s \in S, \exists s$ such that $s=ese^{-1}$. Let $x$ and $y$ be in $N_G(S)$. This means that for any $s \in S$, $xsx^{-1} \in S$ and $ysy^{-1} \in S$. Does $x(ysy^{-1})x^{-1} \in S$? Yes, obvious from the bracketing. We know that (since $x,y \in N_G(S)$) if $s \in S$, $\exists s',s'' \in S$ so that $s=xs'x^{-1}$ and $s=ys''y^{-1}$. Now we ask, for any given $s \in S$, does there exist a $t \in S$ so that $s=x(yty^{-1})x^{-1}$? For the given arbitrary $s$, there exists $s'$ so that $s=xs'x^{-1}$, and for $s',$ there exists $s''$ so that $s'=ys''y^{-1}$ which means $s=xys''y^{-1}x^{-1}$ which means $xy$ is also in $N_G(S)$. All that is left is the inverse.
Let $x \in N_G(S)$ which means that $\forall s,$ $xsx^{-1} \in S$ and for all $s,$ $\exists s'$ so that $s=xs'x^{-1}$. For every $s$, there exists $s' \in S$ so that $s=xs'x^{-1}$ which means $x^{-1}sx=s' \in S$ which means for all $s$, $x^{-1}sx \in S$. If $s \in S$,  does there exists $s' \in S$ so that $s=x^{-1}s'x$? Or rephrased, does there exist $s'$ so that $s'=xsx^{-1}$? Obviously, if $s$ is in $S$, there exists $xsx^{-1}$ so that $xsx^{-1}=s' \in S$ which means $s=x^{-1}s'x$ which completes the proof. }

\thmp{}{$\langle S \rangle \triangleleft N_G(S)$
\\\\ And $N_G(H)$ is the largest subgroup of $G$ which $H$ is \emph{normal} to.}{First we tackle whether $\langle S \rangle \triangleleft N_G(S)$. If $S$ is itself a group, then our job is easier. $\langle S\rangle=S$ in that case. $N_G(S):=\{g \in G: gSg^{-1}\subseteq S \iff gSg^{-1}=S\}$ which means that for every element in $N_G(S)$, and for every element $h$ in group $S$, we have $ghg^{-1} \in S$ which makes $S$ a normal subgroup of $N_G(S)$. Say $S \triangleleft K$ for some subgroup $K$, which means for every element $k \in K$, $kSk^{-1} \subseteq S \iff kSk^{-1}=S$(since $S$ is a group). This means that every point $k$ of $K$ is actually a point of $N_G(S)$ which makes $N_G(S)$ the largest subgroup $S$ is normal to.
\\\\If $S$ is just a set, $\langle S \rangle :=\{a_1^{e_1}a_2^{e_2}\cdots a_k^{e_k}: \{a_1,a_2 \cdots a_k\}\in S, e_1,e_2 \cdots e_k \in \{-1,1\} \}$. Let $s$ be an arbitrary element in $\langle S \rangle $ of the form $a_1^{e_1}a_2^{e_2} \cdots a_k^{e_k}$. Let $z$ be an arbitrary element in $N_G(S)$. This means that for every element $s \in S$, we have $zsz^{-1} \in S$ and for every $s \in S$, there exists $s' \in S$ so that $s=zs'z^{-1}$( this needn't be said since, if $zsz^{-1} \in S$, $z^{-1}sz \in S$ too, this comes straight from the remark). We want to show that $\langle S \rangle \triangleleft N_G(S)$ which means that for every element $z$ of $N_G(S)$, and for every element $x$ of $\langle S \rangle$, $zxz^{-1} \in \langle S \rangle$ (or $x\langle S \rangle x^{-1}=\langle S \rangle$). If $a_j$ (with positive exponent) is in $S$, then $za_jz^{-1}$ is also in $S$. If $a_j$  is in $S$ but has a negative exponent, then $za_jz^{-1} \in S$ (from normalizer condition, it also means $z^{-1}a_jz \in S$) which means $z(a_j)^{-1}z^{-1} \in \langle S \rangle $. So, if given arbitrary $s=a_1^{e_1}a_2^{e_2} \cdots a_k^{e_k} \in \langle S \rangle$, and an element $z \in N_G(S)$, We can decompose the product as $a_1^{e_1}z^{-1}za_2^{e_2}z^{-2} \cdots a_{k-1}^{e_{k-1}}z^{-1}za_k^{e_k} \in \langle S \rangle$. If we premultiply and post multiply the previous expression with $z$ and its inverse, we get $z(a_1^{e_1}z^{-1}za_2^{e_2}z^{-2} \cdots a_{k-1}^{e_{k-1}}z^{-1}za_k^{e_k})z^{-1}$ which will then be a product of terms of the kind $za_jz^{-1}$ which are in $\langle S \rangle$, which means finally that $zsz^{-1} \in \langle S \rangle$. Hence, we are done with showing that $\langle S\rangle  \triangleleft N_G(S)$. Suppose $\langle S \rangle \triangleleft K$ for some subgroup $K$. This means that for every element $k \in K$ and every $s \in \langle S \rangle$, $ksk^{-1} \in \langle S \rangle$.   
\\\\ Also note that $N_G(S) \leq N_G(\langle S \rangle)$ since if $g$ is such that $\forall s \in S$ $gsg^{-1} \in S$ and $g^{-1}sg \in S$, then $g^{-1}s^{-1}g \in \langle S \rangle$ so $g$ would be an element in $N_G(\langle S \rangle)$.}

\thmp{}{A subgroup $H$ is normal to $G$ (denoted $H \triangleleft G$)if and only if $N_G(H)=G$.}{$\implies)$ Suppose $H \triangleleft G$ which means that for every $q \in G$, $\forall h \in H$, $qhq^{-1} \in H$. The normalizer of $H$, $N_G(H):=\{x \in G: \forall h \in H, xhx^{-1} \in H\}$, which is easily seen to be the whole set.
\\\\ $\impliedby)$ Suppose $N_G(H)=G$ which means for every element $g$ of $G$, every element $h$ is so that $ghg^{-1} \in H$ which makes $H$ normal in $G$.}
\defn{$G//H$}{This is called as the "quotient" of $G$ over $H$. We define the equivalce relation $\equiv mod(H)$ as follows:
\\\\$a \equiv b mod(H)$ if and only if $a^{-1}b \in H$ or $b \in aH$. We denote the set of all equivalence classes of this relation as $G//H$. The notation for a particular equivalence class of, say, $a \in G$, looks like $aH$, the left coset of $H$ wrt $a$. }
\subsubsection{Something on Cosets and Lagrange's Theorem}
Let $G$ be a group and $H$ be a subgroup of $G$. Define the equivalence relation $\equiv$ on $G$ as follows: $a \equiv b$ if and only if $a^{-1}b \in H$ (which is equivalent to saying $\exists h \in H$ so that $b=ah$). We digress to define the following:
\\\\ \textbf{Definition:(Left Coset)} We digress to define what is called a \textbf{left coset} of $H\leq G$ given an element $b \in G$ (denoted $bH$). $$bH:=\{z\in G: \exists h \in H: z=bh\}$$ or equivalently $$bH:=\{bh:h\in H\}$$
\\\\ \textbf{Definition:(Right Coset)} Similarly, a \textbf{Right Coset} of $H \leq G$ given an element $a \in G$ (denoted $Ha$) is defined as $$Ha:=\{z \in G: \exists h \in H: z=ha\} $$ or equivalently $$Ha:=\{ha: h \in H\} $$
Note that, the relation "$\equiv$" is an equivalence one, and that the equivalence classes formed are precisely the left cosets of $H$.
\\\\ Note the following: every left coset of $H$ is bijective to every other. This is easily seen by the canonical bijection $ah \mapsto bh$. Also note that the equivalence class of $[e]$ is precisely $H$ itself. therefore, the cardinality of every coset is the same, and is equal to the cardinality of $H$. Owing to the fact that equivalce classes split or partition the set (the set on which the relation is defined) into disjoint subsets whose union gives us back the whole set. Therefore, $\cup_{a \in G} aH=G$. If $G$ is a finite group, then (Let $X:=\{[x]: x \in G\}$), $$\sum_{[a]\in X} [a]=\sum_{[a]\in X} |aH|=k_0|H|=|G|$$ where $k_0$ counts the number of distinct cosets of $H$.  
\\\\ This is Lagrange's theorem:
\thmp{Lagrange's Theorem}{Suppose $G$ is a given finite group, and $H$ a subgroup of $G$. Then, $|H| \ | \ |G|$}{We saw from the previous analysis that for some integer $k$, $k|H|=|G|$. This means that for any subgroup $H$ of finite group $G$, $|H|$ divides $|G|$.}
\corp{(\textbf{Euler's Theorem}) Let $gcd(a,n)=1$, then $a^{\phi(n)}\equiv 1 mod(n)$ where $\phi(n)$ is the totient function. }{Consider $(\ZZ / n \ZZ)^{*}$, the multiplicative group. This has, as we know, $\phi(n)$ elements (i.e, all numbers $q$ smaller than $n$ such that $gcd(q,n)=1$). Cosnider a given $a$ so that $(a,n)=1$, which means $a \in \ZZ/ n\ZZ^{*}$. Cosnider $\langle a \rangle \leq (\ZZ/ n\ZZ)^*$. Since the main group is of finite order, this cyclic subgroup also has to have finite order $k$. i.e, $a^k=1$. Moreover, from Lagrange's theorem, we note that any subgroup's cardinality divides the main group's cardinality, which means $k | \phi(n)$. This means $k\gamma=\phi(n)$ which gives us $(a^k)^{\gamma}=a^{\phi(n)}=1$, which proves the reuslt.
\\\\ \textbf{Alt proof:}\emph{work in progress....} }{}
\corp{\textbf{Fermat's Little Theorem} Let $a,p \in \ZZ$, $p$ being a prime, with $gcd(a,p)=1$, then $a^{p-1}\equiv 1 mod(p)$}{Simply plug $n=p$ and $\phi(n)=\phi(p)=p-1$ into Euler's Theorem. We are done.
\\\\ \textbf{Alt proof:}\emph{work in progress....}}{}

\begin{center}
\textbf{On Quotients, and its maps}
\end{center}
We define operations on $G//H$ as follows: $[a][b]:=[ab]$. Is this well defined? Say $a\equiv a' mod(H)$ and $b \equiv b' mod(H)$, then $a' \in aH$ and $b' \in bH$. $a'b'=ahbh'$. $b^{-1}a^{-1}a'b'=(b^{-1}a^{-1}ahbh')=b^{-1}hbh' \in H $. Hence, $ab \equiv a'b' mod(H)$ and the operation is well defined. 
\\\\ Under this operation, when does $G//H$ become a group? We require: associativity, closure, identity and inverse. $[a] \cdot [e]=[e][a]=[a]$ where $[e]$ is just $H$. This shall be our idenity. Consider $[a]$ and $[b]$. By definition, $[a]\cdot[b]=[ab]$, which gives closure.  
\newpage
\thmp{}{A subgroup $H$ is normal to $G$ if and only if it is the kernel of some Homomorphism $f$ from $G$ to some other group $K$}{$\implies)$ Consider }
\subsection{Even more basix, Gnerators etc.}

\defn{Generator}{Let $G$ be a group and $S$ a subset of $G$. We say $G$ is \emph{\textbf{generated by}} $S$, denoted by $G=\langle S \rangle$ if every element of $G$ can be written as a finite sequence of products of elements in $S$. More specifically, for every $x \in G$, there exists $q_1, q_2, \cdots q_{n_x}$ (needn't all be distinct)and indices $p_1,p_2 \cdots p_{n_x}$ so that $x=q_1^{p_1} q_2^{p_2} \cdots q_{n_x}^{p_{n_x}}$. $$\langle S \rangle:=\{ a_1^{e_1}a_2^{e_2} \cdots a_n^{e_n}: \text{for any } a_1,a_2 \cdots a_n\text{ in }S, \text{ and any } e_1,e_2 \cdots e_n \in \ZZ \} $$ }

\thmp{}{Let $G$ be a cyclic group $\langle a \rangle$ of order $n$. Suppose $m|n$, then there exists a cyclic subgroup of order $m$ in $G$. Moreover, this group is the unique subgroup of order $m$. }{Consider $\langle a^{n/m} \rangle$. $O(a^{m/n})=n/(gcd(n/m, n))=n/(n/m)=m$. So existence is clear. Now onto uniqueness:
\\\\ We found $\langle a^{n/m} \rangle$ to be one such group. Suppose another subgroup $\langle a^j \rangle$ also is $m$ order. $O(a^j)=n/gcd(j,n)$ which is the order of the group. Hence $n/(j,n)=m \implies n/m=gcd(j,n)$ which means $n/m | j$ or $\delta(n/m)=j$ which puts $a^j$ inside $\langle a^{n/m} \rangle$ which makes $\langle a^j \rangle $ a subgroup of $\langle a^{n/m} \rangle $. But since order is the same, the two groups must be same.}
\end{document}