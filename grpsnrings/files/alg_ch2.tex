\documentclass[../Main.tex]{subfiles}

\begin{document}

\chapter{Groups}
\section{Basix}
\defn{A group $(G,\cdot)$}{A group consists of a set and a binary relation $\cdot:G \times G \to G$ (which makes it closed by definition) such that:
\begin{enumerate}
    \item $\forall$ $a,b,c \in G$, $(a\cdot b)\cdot c=a \cdot (b\cdot c)$ (Associative)
    \item There exists an element $e\in G$ called identity so that for every $a \in G$ we have $a\cdot e=e\cdot a=a$
    \item For every element $a$ in $G$ we have another element $a^{-1}$ so that $aa^{-1}=a^{-1}a=e$ 
\end{enumerate}
A way to remember group axioms is to remember ASCII: \textbf{AS}sociative, \textbf{C}losed, \textbf{I}dentity, and \textbf{I}nverse
}


\exm{Some group examples:}{ $\ZZ $ with the usual addition, with $0$ as identity. Inverse being $-a$.
\\\\ $\ZZ / n\ZZ$ with the modular addition, with identity being $\overline{0}$ and inverse being $\overline{-a}$.
\\\\ In fact $ \ZZ, \QQ, \RR, \CC $ are groups with respective addition, identity being $0$ and inverse being $-a$.
\\\\ $\RR^+, \CC-\{0\}, \RR-\{0\}$, etc. are groups with multiplication as the operation. Here identity is $1$, and inverse is $\frac{1}{a} $.
\\\\ $\ZZ / n\ZZ*$, the set of all congruence classes in $\ZZ/ n\ZZ$ which have a multiplicative inverse (or equivalently, those that have gcd with $n$ as $1$) forms a group under multiplication. The identity is $\overline{1}$ and the inverse is that $\overline{c}$, which was shown to exist, such that $\overline{a} \cdot \overline{c}=\overline{1}$. 

}

\defn{Direct Product}{If $(A, !) \text{ and } (B,*)$ are each groups, then we define the \textbf{Direct Product} as the group formed by $A \times B:= \{(a,b): a\in A, b \in B \}$ with the operation $\&:(A \times B)\times(A \times B) \to A \times B$ defined by $(a_1,b_1)\&(a_2,b_2)=(a_1 ! a_2, b_1 * b_2)$}

\prop{If $G, \cdot$ is a group, then the following hold:
\begin{enumerate}
    \item The identity element $e$ is unique.
    \item for every $a \in G$, the inverse element $a^{-1}$ is unique
    \item $(a \cdot b)^{-1}=b^{-1} \cdot a^{-1}$
    \item For any $a_1,a_2,\dots a_n \in G$, the expression $a_1 \cdot a_2 \cdot \dots \cdot a_n$ is independent of how it is bracketed.

\end{enumerate}}
\pf{(1) Suppose the identity is not unique, i.e, there exists $e_1$ and $e_2$ so that it obeys identity axioms. We have $a \cdot e=e\cdot a=a$, which means $(e_1)e_2=e_2(e_1)=e_2$, treating $e_2$ as true identity. But also, $(e_2)e_1=e_1(e_2)=e_1, =e_2$. Hence we see easily that $e_1=e_2$.
\\\\ (2) Suppose two inverses $x$ and $y$ exist. $ax=e$, which means $yax=ye=y$, but from associativity, $(ya)x=x=y$. Hence, $x=y:=a^{-1}$
\\\\ (3) $a \cdot b (a \cdot b)^{-1}=e$ which implies
$a^{-1}a \cdot b(a \cdot b)^{-1}=a^{-1} \implies $
$b^{-1}(a^{-1}a) \cdot b(a \cdot b)^{-1}=b^{-1}a^{-1} $ which directly gives $(a\cdot b)^{-1}=b^{-1}a^{-1}$ 
\\\\ (4) (\textbf{PEDANTIC PROOF AHEAD, SKIP IF NOT A PEDANT}) For just one element $a_1$, there is no need to even check. Assume that the bracketing does not change the meaning for any consequetive $n$ operations. Consider 
$$a_1 \cdot a_2 \cdot a_3 \cdots a_n \cdot a_{n+1}$$
First look at the bracketing 
$$\{(a_1 \cdot a_2 \cdot a_3 \cdots a_n )\}\cdot (a_{n+1})$$
From induction hypothesis, no bracketing inside the $\{ \}$ affects the operations. Next, consider the kind
$$\{(a_1 \cdot a_2 \cdot a_3 \cdots)\} (a_n \cdot a_{n+1})$$
Again, from induction, no bracketing affects the operations. By means of reverse induction, we show that no bracketing affects the end result of these operations.
}

\prop{Let $G$ be a group and let $a,b$ be elements in the group. Then the equations $ax=b$ and $ya=b$ have unique solutions. Explicitly, we have the left and right cancellation laws:
\\\\ If $au=av, $ then $u=v$
\\\\ If $ub=vb, $ then $u=v$

}
\pf{If $au=av$, we multiply both sides by $a^{-1}$ to preserve equality $u=v$. Similarly, we multiply $b^{-1}$ to either side of the equation $ub=vb$ which gives $u=b$}

\defn{Order of an element $g$ in a group $G$}{We say an element $g$ in $G$ is of \emph{order} $n \in \NN$ if $n$ is the smallest natural number so that $g^n=g \cdot g \cdots g=e$, the identity. We denote this as $O(g)$.}
\defn{Order of a Group $G$, denoted by $|G|$.}{The cardinality of the group.}

\thmp{}{If $G$ is a group and $a$ an element in $G$ with $O(a)=n$, then $a^m=1$ if and only if $n|m$}{
$\implies$) Given $O(a)=n$ we have $n$ to be the smallest natural number so that $a^n=1$. If we have that $a^m=1$, and $n \not{|} m$, then $m=qn+r$ where $0<r<n$. Therefore, $a^r \neq 1$. We have that $a^{qn+r}=a^{qn} \cdot a^r=a^r \neq 0$ which is absurd.
\\\\ $\impliedby$) Given $n | m$, obviously then $a^m=1$. 
}

\thmp{}{If $O(a)=n$, then $O(a^m)=\frac{n}{gcd(m,n)}$.}{We understand that $\frac{n}{gcd(m,n)}$ is atleast a candidate, since we can see clearly that $(a^m)^{\frac{n}{gcd(m,n)}}=(a^n)^{\frac{m}{gcd(m,n)}}=1$.
Suppose $k$ is the order, with $k < \frac{n}{gcd(m,n)}$ so that $a^{mk}=1$. From the previous theorem, we see that $n | mk$. i.e, $n\delta=mk \implies 
\frac{n}{gcd(m,n)}\delta=\frac{m}{gcd(m,n)}k$. Note that $\frac{n}{(m.n)}$ and $\frac{m}{(m,n)}$ share no common divisors, for if they did, then that, multiplied with the actual gcd would yield a divisor larger than the gcd. Hence, $gcd(\frac{n}{(m,n)} ,\frac{m}{(m,n)})=1$. This means, from previous lemmas, that $\frac{n}{(m,n)}$ divides $k$. This is, ofcourse, absurd.

}

\thmp{Real Numbers $mod(1)$}{Let $G:=\{x \in \RR: 0 \leq x<1\}$. Define $x \circ y= \{x+y\}$ where $\{\cdot \}$ denotes the fractional part (and $[\cdot ]$ denotes the integral part, or the GIF). Then, $G$ is an abelian group under $\{\circ \}$}{Closure of $x \circ y$ is pretty obvious. We freely use $\{\cdot \}$, $frac\{\cdot\}$ and $\underline{\cdot}$ interchangibly. We consider
$x \circ (y\circ z)=frac(\underline{x}+[x] +frac(y+z) )$=$frac(\underline{x}+[x] +frac(\underline{y}+[y]+\underline{z}+[z]))=frac(\underline{x}+frac(\underline{y}+\underline{z}))$=
$frac(\underline{x}+(\underline{y}+\underline{z})-[\underline{y}+\underline{z}])=frac(\underline{x}+\underline{y}+\underline{z})$
\\\\ Now consider $(x \circ y)\circ z=frac(frac(\underline{x}+\underline{y})+\underline{z}+[z])$
$=frac(frac(\underline{x}+\underline{y})+\underline{z})$ $= frac((\underline{x}+\underline{y})-[\underline{x}+\underline{y}]+\underline{z}+[z])=frac(\underline{x}+\underline{y}+\underline{z})$. Hence we see $\circ$ is associative. Trivial to note that the idenity element is $\underline{0}$ and the inverse for every $\underline{x}$ is $\underline{-x}$.

}

\thmp{Group of the $n$-th roots of unity}{Suppose $G:=\{z \in \CC: z^n=1: \text{ for some } n\}$}{We want to solve $z^n=1$. Applying polar coordinates we have $|z|^n(cis(\theta))^n=1$. Taking mod gives us $|z|=1$. We have to solve for, then, $cis(theta)^n=1$. It is simple computation to see that $cis(\theta)^n=cis(n\theta)$ which gives us $cis(n\theta)=1$. The solutions to this are $\theta=\frac{2\pi k}{n}$ for any integer $k$. Therefore, the solutions to $z^n=1$ are of the form $z=cis(\frac{2k \pi}{n})$. We assume a modulo $2\pi$ structure, i.e, we classify solutions of the kind $\theta+2k\pi$ in the class of $\theta$. We see then, that for $k \leq n-1$, each solution is unique. If we let $\omega=cis(\frac{2\pi}{n})$. We see that all the other elements are generated by $\omega$ since for $k=2$, we just have $\omega^2$ (from the way cis powers work). Till $k=n-1$, we have unique solutions generated by $\omega$ given by $1, \omega, \omega^2 \cdots \omega^{n-1}$. We see that when $k=n$ we get $\theta=\frac{2\pi n}{n}=2\pi \equiv 0 mod(2\pi)$. For $n+j$ where $j<n$, we see that $\theta=\frac{2\pi(n+j)}{n} =2\pi+ \frac{2\pi j}{n} \equiv \frac{2\pi j}{n} mod(2\pi)$. Hence, all the unique solutions are $1, \omega, \omega^2 \cdots \omega^{n-1}$.
\\\\ To see that this is a group under multiplication, we note that $\omega^x (\omega^y \omega^z)=(\omega^x \omega^y) \omega^z =\omega^{(x+y+z) mod(n)}$. Every element has an inverse since $\omega ^j \cdot \omega^{n-j}=1$ ($1$ is the identity here since $1 \omega^j=\omega^j \cdot 1=\omega^j$)
\\\\ $G$, though a group under multiplication, is not one under addition. For example, consider $\omega$ and $1$. $(1+\omega)^n=1+ {n \choose 1} \omega + {n \choose 2} \omega^2  \cdots + 1 $
(\textbf{TO BE FILLED IN LATER})}

\fact{If $a, b \in G$, then $|ab|=|ba|$}
\pf{We have $(ab)(ab) \cdots (ab)=(ab)^n=e$. Rearranging the brackets we get $a(ba)(ba)\cdots (b)=a(ba)^{n-1}(b)=e$ which gives $(ba)^{n-1}=a^{-1}b^{-1}=(ba)^{-1}$ which eventually gives $(ba)^{n}=e$. Therefore, if $m$ was the order of $ba$, then $m|n$. Similarly we can re-run the argument in the other direction starting with $(ba)^m=e$ to get $n|m$. This gives $n=m$.

}
\fact{If $x^2=1$ for every $x \in G$, then $G$ is abelian}
\pf{Let $ab \neq ba \implies a^2b=b \neq a(ba)$. This implies $b^2=e\neq (ba)^2 \implies 1 \neq 1$. Absurd.}

\fact{Any finite group of even order contains an element $a$ with order $2$.}
\pf{Suppose that for every non-identity element $x$ we have $o(x)=p \neq 2$ with $p\geq 3$. We can then notice that for every element, $x \neq x^{-1}$. Hence, every element along with its inverses would form an even sized set (due to uniqueness of inverses, none overlap). Hence, adding identity to this would make the group odd.}

\exm{$G=\{1,a,b,c\}$ is $|G|=4$ with $1$ identity. This group has a unique multiplication table}{We can immediately fill up the initial parts:
\begin{center}

\begin{tabular}{ c c c c c }
x & 1 & a & b & c \\
1 & 1 & a & b & c \\
a & a & x & x & x \\
b & b & x & x & x \\
c & c & x & x & x \\
\end{tabular}
\end{center}
Since this is a finite group of order $4$, there should be atleast one element with order $2$. We WLOG select that element to be $a$ so that $a^2=q$. Question: Is $ab=a, $ or $b$? Neither, because that would imply $a$ or $b$ is identity. So $ab=c$. We then have that $a^2b=b=ac$. Is $ba=c$? It can't be identity obviously, so yes. Same way, $ac=b$ and likewise $ca=b$ (cuz what else is there?). Same way, $bc=cb=a$. So far we got:


\begin{center}

    \begin{tabular}{ c c c c c }
    x & 1 & a & b & c \\
    1 & 1 & a & b & c \\
    a & a & 1 & c & b \\
    b & b & c & x & a \\
    c & c & b & a & x \\
    \end{tabular}
    \end{center}

Suppose $c^2=a$. Then $c(ca)=a^2$ which would mean $cb=1$. Is it then that $c^2=b$? $(ac)(c)=ab=c$ but $ac=b$ which means $bc=c$. Again, absurd. So $c^2=1$. In a similar vein, $b^2=1$.
Finally we got 


\begin{center}

    \begin{tabular}{ c c c c c }
    \textbf{x} & \textbf{1} & \textbf{a} & \textbf{b} & \textbf{c} \\
    \textbf{1} & 1 & a & b & c \\
    \textbf{a} & a & 1 & c & b \\
    \textbf{b} & b & c & 1 & a \\
    \textbf{c} & c & b & a & 1 \\
    \end{tabular}
    \end{center}
}
\defn{Subgroup}{A set $H \subseteq G$ of group $G$ is said to be a subgroup if $H$ is itself a group, i.e, follows ASCII axioms under the operation inherited from $G$. If $H$ is a proper subgroup of $G$, then we denote it by $H<G$. Else, $H \leq G$}

\defn{Cyclic Subgroup}{Suppose $G,\cdot $ is a group, with an element $a$. Suppose $<a>$ is a subgroup of $G$ that contains $a$. Must definitely have $e$ which is notated to be $a^0$. It must then definitely have $a \cdot a$, $a\cdot a \cdot a$ and so on till $a^n$ where $o(a)=n$. If no order exists, we take it to be $\forall n \in \ZZ$. $<a>:=\{a^n: n \in \ZZ \}$ This is enough for it to be a group:
\\\\ $e=a^0$ is in the group. For every $b$, i.e, $a^k$ in the group, $a^{-k}$ is also in the group by definition. It obeys ASCII.
\\\\ \textbf{Fact:} $<a>$ is the smallest subgroup of $G$ containing $a$. Analogous to \textit{span}.}
\exm{Some groups cyclically generated}{$\ZZ/ n\ZZ$ as an additive group is generated by $1$. That is, $<1>$ is precisely $\ZZ /n\ZZ$.
\\\\n-th roots of unity: $1, \omega, \omega^2 \cdots \omega^{n-1}$, is generated by $<\omega>$. }

\subsection{The Dihedral Group $D_{2n}$}
Given an $n-gon$ that is regular, we define the symmetries on it by permutation maps or bijective maps from $\{1,2,3\cdots,n\}$ into itself. 

\defn{Rotation $r$}{$r:\{1,2,3 \cdots n\} \to \{1,2,\cdots,n \}$ is defined as $$1 \overset{\mbox{r}}{\longrightarrow} 2$$
$$2 \overset{\mbox{r}}{\longrightarrow} 3$$
$$ \vdots$$

$$n-1 \overset{\mbox{r}}{\longrightarrow} n$$

$$n \overset{\mbox{r}}{\longrightarrow} 1$$

Whose inverse is, as one can guess:
$$2 \overset{\mbox{inverse(r)}}{\longrightarrow} 1$$
$$3 \overset{\mbox{inverse(r)}}{\longrightarrow} 2$$
$$ \vdots$$

$$n \overset{\mbox{inverse(r)}}{\longrightarrow} n-1$$

$$1 \overset{\mbox{inverse(r)}}{\longrightarrow} n$$
}

\defn{Symmetry, or flipping, or mirror whatever}{$s$ is defined as $s:\{1 \cdots n\} \to \{1 \cdots n\}$ as follows:
$$1 \overset{s}\mapsto 1 $$
$$2 \overset{s}\mapsto n $$
$$3 \overset{s}\mapsto n-1 $$
$$\vdots$$
$$n \overset{s}\mapsto 2 $$
Note that, $s^2=1$}

\textbf{Some Properties of $D_{2n}$}
The symmetries of $D_{2n}$ are the functions listed above. Note the following:
\begin{enumerate}
    \item $1$, $r$, $\cdots r^{n-1}$ form distinct elements. $|r|=n$ since $r^n=1$
    \item $r$ follows $\ZZ /n \ZZ$ structure in that, $r^{j}$ has, as its inverse, $r^{n-j}$. It obeys similar modular structure.
    \item $s^2=1$
    \item $rs=sr^{-1}$. Note that $rs$ amounts to "Pivoting" about $2$ and flipping the dihedron, which can be achieved by reverse rotating, i.e, $r^-1$ first, and then flipping, i.e $sr^{-1}$. Hence, $rs=sr^{-1}$.
    \item Since the inverse elements of $r^i$ are $r^{-1}$, the previous result can be more generally written as $(r^i)s=sr^{-i}$. In a spoon feedy way we see that $rs=sr^{-1}$ $\implies$ $r(rs)=r^2s=r(sr^{-1})=(rs)(r^{-1})=(sr^{-1}r^{-1})=sr^{-2}$. Keep going as such.
    \item The elements $1,r,r^2, \cdots r^{n-1}$ constitute the subgroup of rotations, each one corresponding to a rotation of $\frac{2j\pi}{n}$.
    \item The elements $s, rs, r^2s \cdots r^{n-1}s$ correspond to "pivoting" the $j-th$ number and flipping about that. These on their own dont constitute a group for, $(r^ns)(r^ms)=r^n(sr^m)s=r^n(r^{-m})$ which falls into the rotation group. 
    \item Note that $s \neq r^{i}$ for any $i$. This ought to be intuitively clear.
    \item $sr^{i}\neq sr^{j}$ since flipping about different pivots achieves a different structure, one that is different by rotations alone (obviously).
    \item The set $\{1,r,r^2 \cdots r^{n-1}; s; rs,r^2s, \cdots r^{n-1}s\}$ Constitutes a group, of order $2n$. This is stated formally in the next theorem, with proof.  
\end{enumerate}

\thmp{}{The set $\{1,r,r^2 \cdots r^{n-1}; s; rs,r^2s, \cdots r^{n-1}s\}$ Constitutes a group, of order $2n$. }{We note that $1,r,r^2, \cdots r^{n-1}$ all obey ASCII. So does $s$, since it is self inverse(The identity here is the identity function). Consider the permutations of the kind $r^js$. These have inverses as well, for if we compose this with $r^{n-j}$, we would have $r^{n-j} \circ (r^j s)=s$. If we compose this still, with $s$, we get $1$. The total composition on $r^j s$ would have been $sr^{n-j}$. Infact, these elements too are self inverses. Easier way to see this is $(r^i s)(r^i s)=r^i(sr^i)s=r^i(r^{-i}s)s=1$. These also, then follow ASCII. }

\end{document}