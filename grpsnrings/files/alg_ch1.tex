\documentclass[../Main.tex]{subfiles}

\begin{document}
\chapter{Prelimnaries}

\section{Sets, functions and all that}

\defn{Prelimnary definitions}{
\begin{enumerate}
    \item (Cartesian Product): if $A$ and $B$ are non empty sets, the \emph{Cartesian Product} $A \times B$ is defined as the set of ordered pairs $a,b$ wherein $a \in A, b \in B$. i.e, $ A \times B := \{ (a,b): a \in A, b \in B  \}$ 
    \item (Function): A function from $A$ to $B$ is a set $f \subseteq A \times B$ such that, $a,b \in f \text{ and } a,c \in f \implies b=c$. A is called the \textbf{Domain of $f$}. $Range(f):=f(A)$ (see next definition)
    \item (Direct Image): Direct image $f(A):=\{y \in B: \exists x\in A \text{ such that } f(x)=y \}$
    \item (Inverse Image):$f^{-1}(S\subseteq B):=\{x \in A: f(x) \in S\}$
    \item (Relation): Any subset $R\subseteq A \times B$ is a relation from $A$ to $B$.\\We say $x \in X$ is "related to" $y \in Y$ under the relation $R$, or simply $xRy$ or $R(x)=y$ if $(x,y) \in R \subseteq X \times Y$. 
    \item (Injection): $f: A \to B$ is injective if $\forall x_1,x_2 \in A, (x_1,b) \in f \text{ and } (x_2,b) \in f \iff x_1=x_2$
    \item (Surjection): $f:A \to B$ is surjective if $\forall b \in B, \exists a \in A \text{ such that } f(a)=b$
    \item (Bijection): $f:A\to B$ is bijective if its both surjective and injective.
    \item {(Identity function on a set)}: $I_A: A \to A$ defined by $\forall x \in A, I_A(x)=x$
    \item  (Permutation): Simply a bijection from $A$ to itself is called a permutation.
\end{enumerate}
}

\defn{(Left Inverse)}{ We say $f:A \to B$ has a left inverse if there is a function $g: B \to A$ such that $g \circ f=I_A$}
\thmp{}{$f:A \to B$ has a left inverse if and only if it is injective.}{
$\implies$) If $f$ has a left inverse $g$, Consider $x,y \in A$ such that $f(x)=f(y)=p$. 

We have $g \circ f(x)=g(p)=x=g \circ f(y)=y$. Hence, $x=y$, Injective.\\
$\impliedby$) Given that $f:A \to B$ is injective, define $g:B \to A$ as: 
$$
g(z \in B)=
\begin{cases}
a, \text{ where } f(a)=z, \ \text{if } z\in f(A) \\
\text{whatever}, \ \text{if } z \not\in f(A)\end{cases}
$$
consider $g \circ f (x \in A) = g \circ (f(x))$.

Obviously, $f(x) \in f(A)$, therefore, $g(f(x))=$ that $a$ such that $f(a)=f(x)$. 

That $a$ is $x$. 
Hence, $g(f(x))=x$
}
\defn{(Right Inverse)}{$f:A \to B$ is said to have a right inverse if there is a function $g:B \to A$ such that $f \circ g=I_B$}

\thmp{}{$f:A \to B$ has a right inverse if and only if $f$ is Surjective.}
{$\implies$) If $f$ has a right inverse $g$, such that $f \circ g=I_B:B \to B$, then it is evident that the range of $f$ is B, for if not, range of $f \circ g$ wouldn't be $B$ either.\\
$\impliedby$) If $f$ is surjective, then for all $b \in B$, there exists atleast one $a \in A$ such that $f(a)=b$ define $g$ as:
$$g(x \in B)= \text{one of those a} \in A \text{ such that} f(a)=b$$
Consider $f \circ g (x \in B) = f ( \text{ one of the } a \text{ such that } f(a)=b)=b, \forall b \in B$

Hence, $f \circ g=I_B$
}

\thmp{}{If $f$ has left inverse $g_1$ and right inverse $g_2$, then $g_1=g_2$.
\rmk{(True for anything that is Associative, and function composition is associative.)}}{$g_1 \circ f=I_A$ and $f \circ g_2 =I_B$

$g_1 \circ (f \circ g_2)=g_1 \circ I_B= g$

$= (g_1\circ f) \circ g_2=I_A \circ g_2=g_2$

Hence $g_1=g_2$}
\corp{ $f$ is invertible (i.e, both left and right inverse exist) if and only if it is bijective.}{Obvious}


\subsection{Operations on Relations}
If $R$ and $S$ are binary relations over $X \times Y$:
\begin{enumerate}
  \item $R \bigcup U := \{(x,y) | xRy \text{ or } xSy \}$
  \item $R \bigcap S := \{(x,y) | xRy \text{ and } xSy \}$
  \item Given $S:Y \to Z$ and $R:X \to Y$, $S \circ R:=\{(x,z) | \exists y \text{ such that } ySz \ \& \ xRy \}$
  \item If $R$ is binary over $X \times Y$, $\bar{R} := \{(x,y) | \neg (xRy)\}$
\end{enumerate}

\subsection{Homogeneous Relations}
If $R$ is a binary relation over $X \times X$, it is Homogeneous.
\defn{Definitions Regarding Relations}{
\begin{enumerate}
  \item (Reflexive): $\forall x \in X, xRx$
  \item (Symmetric): $\forall x,y \in X, xRy \implies yRx$
  \item (Transitive): $\forall x,y,z \in X, \text{ if } xRy \ \& \ yRz \implies xRz$
  \item (Dense): $\forall x,y \in X, \ \text{if} \ xRy, \text{ then there is some } z \in X \text{ such that } xRz \ \& \ zRy$
  \item (\textbf{Equivalence Relation}): $R$ is an equivalence relation if it is Reflexive, Symmetric and Transitive.
  \item (Equivalence class of $a\in A$(where there is an equivalence relation defined)): Set of all $b \in A$ such that $bRa$. 
  \item (Partition of $A$): Any collection of sets $\{A_i :i\in I \}$ (where $I$ is some indexing set) such that: $$A= \bigcup_{i\in I} A_i$$ $$A_i \cap A_j = \phi \text{ if } \forall i,j \in I, i \neq j$$
\end{enumerate}
  }

\thmp{}{Let $A$ be a non-empty set. If $R$ defines an equivalence Relation on $A$, then the set of all equivalence classes of $R$ form a partition of A}{Define our collection $\{ A_{\alpha}\}$ as the set of all equivalence classes of $A$. Clearly, $\bigcup_{\alpha \in I} A_{\alpha}=A$. If $A$ only has one element, obviously, that singleton set makes up the partition. Let $A_{\alpha}$ and $A_{\alpha'}$ be equivalence classes of two elements $a$ and $a'$ in $A$. If $a R a'$, then $A_{\alpha}=A_{\alpha'}$ since every element in the equivalence class of $a$ will, from the transitive property, be in the equivalence class of $a'$ . Suppose $\neg (a R a')$. If, then, $\exists x \in A_{\alpha}$ such that $x\in A_{\alpha'}$, this means that $xR\alpha$ and $xR\alpha'$, but from transitive property, this means $\alpha R\alpha'$, which is a contradiction. Therefore, the pairwise intersection is disjoint.}

\thmp{}{If $\{A_i: i \in I \}$ is a partition of A, then there exists an equivalence relation $R$ on $A$ whose equivalence classes are $\{A_i:i \in I\}$.}{ Define $R(x,y)$ if and only if $\exists$ unique $m \in I$ such that $x \in A_m$ and $y \in A_m$.
    

$R(x,x)$ is obvious if non empty, hence $R$ is reflexive.\\\\   
Suppose $R(x,y)$ and $R(y,z)$. Then, there exists a unique $m \in I$ such that $x,y$ are in $A_m$. Similarly, there exists a unique $n \in I$ such that $y,z$ are in $A_n$ . Obviously, if $n \neq m$, intersection of $A_n$ and $A_m$ would be non empty, hence, $n=m$. Hence, $R$ is transitive.\\\\
Consider $R(x,y)$, which means $\exists$ unique $n\in I$ such that $x,y \in A_n \implies R(y,x)$. 
Hence, $R$ is an equivalence relation.}

\section{Induction, Naturals, Rationals and the Axiom of Choice }
\asum{Peano Axioms, characterisation of $\mathbb{N}$}{\begin{enumerate}
  \item $1\in \mathbb{N}$
  \item every $n \in \mathbb{N}$ has a predecessor $n-1 \in \mathbb{N}$ except $1$
  \item $\text{if } n \in \mathbb{N} \implies n+1 \in \mathbb{N}$
\end{enumerate} }
\defn{(Sequence of something)}{A sequence of some object is simply a collection of objects $\{O_l: l\in \mathbb{N}\}$ which can be counted.}

\asum{Well Ordering Property of $\mathbb{N}$}{Every non empty subset of $\mathbb{N}$ has a least element.}

\asum{Weak Induction}{For all subsets $S \subseteq \mathbb{N}$, (($1\in S$)\&(($\forall k \in \mathbb{N}$)($k\in S \implies k+1 \in S$))
$\iff$ $S=\mathbb{N}$)\\\\\textbf{Weak Induction's Negation}:(One direction)

There exists subset $S_0 \subseteq \mathbb{N}$, (($1\in S_0$)\&(($\forall k \in \mathbb{N}$)($k\in S_0 \implies k+1 \in S_0$))
but $S_0 \neq \mathbb{N}$)}

\asum{Strong Induction}{For all subsets $S \subseteq \mathbb{N}$, (($1\in S$)\&(($\forall k \in \mathbb{N}$)(${1,2,...k}\in S' \implies k+1 \in S'$))
$\iff$ $S=\mathbb{N}$)\\\\ \textbf{Strong Induction's Negation}:(One direction)

There exists subset $S' \subseteq \mathbb{N}$, (($1\in S'$)\&(($\forall k \in \mathbb{N}$)(${1,2,...k}\in S' \implies k+1 \in S'$))
but $S' \neq \mathbb{N}$)}

\thmp{}{Weak Induction $\iff$ Strong Induction.}{$\implies$) Suppose Weak induction is true, but not strong induction. Take our set to be that $S'$ in the negation of the Strong Induction Statement. $S' \neq \mathbb{N}$ implies that, either $1 \not\in S'$ or $\exists k \in \mathbb{N}$ such that $k\in S'$ but $k+1 \not\in S'$. We know that $1 \in S'$, so it must be that $\exists k \in \mathbb{N}$ such that $k \in S'$ but $k+1 \not\in S'$. $\{1\} \in S' \implies \{1,2\} \in S'$. Assume that for $n$, $\{1,2,..n\} \in S'$. This means that $\{1,2,...n+1\} \in S'$. This means that for every $n \in \mathbb{N}$, $\{1,2,...n \}\in S' \implies n \in S'$. Contradiction.\\
$\impliedby$)Suppose Strong Induction is true, but not weak induction. Take the set $S_0$ from the negation of Weak Induction. $S_0 \neq \mathbb{N}$. This means, from strong induction, either $1\not\in S_0$ or $\exists k \in \mathbb{N}$ such that $1,2, \dots , k \in S_0$ but $k+1 \not\in S_0$. $1 \in S_0$, hence, $2 \in S_0$ and $\{1,2\}\in S_0$. assume that $\{1,2,....n\} \in S_0$. This means, $n\in S_0 \implies n+1 \in S_0$, which means that $\forall k \in \mathbb{N}, \{1,2,..k\} \in S_0 \implies k+1 \in S_0$. Therefore, $S_0$ is $\mathbb{N}$.}

\thmp{}{Weak Induction $\iff$ Strong Induction $\iff$ Well ordering.}{$\implies$) Suppose that, on the contrary, $S_0$ is a non empty subset of $\mathbb{N}$ , with no least element. Does $1$ exist in $S_0$? No, for that will be the least element. Likewise, then, $2$ does not belong in $S_0$. Assume that $\{1,2,...n\} \not\in S_0$. Does $n+1$ exist in $S_0$? No, for that will become the least element then. From Strong Induction, $\mathbb{N}-S_0=\mathbb{N}\implies S_0=\phi$. Contradiction.\\
$\impliedby$)Suppose $\exists S_0 \subseteq \mathbb{N}$ such that $1 \in S_0$ and $\forall k \in \mathbb{N}$, $k\in S_0 \implies k+1 \in S_0$. Suppose on the contrary, $S_0$ is not $\mathbb{N}$. $\mathbb{N}-S_0$ is then, non-empty. From Well Ordering, there is a least element $q \in \mathbb{N}-S_0$. $\implies, q-1 \in S_0$. But this would imply $q-1+1 \in S_0$. Contradiction. $\mathbb{N}-S_0$ is empty.}

\defn{(Finite Sets)}{A set $X$ is said to be finite, with $n$ elements in it, if $\exists n \in \mathbb{N}$ such that there exists a bijection $f: \{1,2..,n\} \to X$. Set $X$ is \emph{infinite} if it is non-finite.}


\thmp{}{If $A$ and $B$ are finite sets with $m$ and $n$ elements respectively, and $A \cap B=\phi$, then $A \cup B$ is finite, with $m+n$ elements.}{$f:\mathbb{N}_m \to A$ and $g:\mathbb{N}_n \to B$.

Define $h:\mathbb{N}_{m+n} \to A \cup B$ given by:
$$h(i)=\begin{cases}f(i) \text{ if } i=1,2...m\\
g(i-1) \text{ if } i=m+1,m+2,...m+n

\end{cases}$$

If $i=1,2,...m$,$h(i)$ covers all the elements in $A$ through $f$. If $i=m+1,...m+n$, $h(i)$ covers all the elements in $B$ through $g$.

Moreover, $h(i) \neq h(j); i \in [1,m], j\in[m+1,m+n] $ since $A\cap B= \phi$}

\thmp{}{If $C$ is infinite, and $B$ is finite, then $C-B$ is infinite.}{Suppose $C-B$ is finite. We have $B \cap (C-B)=\phi$ and $B \cup (C-B)=C \cup B$

$n(C \cup B)=n(B \cup (C-B))=n(B)+n(C-B)$
This implies $C \cup B$ is finite. Contradiction.
}

\thmp{}{\textbf{Theorem}: Suppose $T$ and $S$ are sets such that $T \subseteq S$. Then:\\

a) If $S$ is finite, $T$ is finite.\\

b)If $T$ is infinite, $S$ is infinite.}{Given that $S$ is finite, there is a function $f:\mathbb{N}_{m} \to S$. Suppose that $S$ has $1$ element. Then either $T$ is empty, or $S$ itself, which means $T$ is finite. Suppose that, upto $n$, it is true that, if $S$ is finite with $n$ elements, all its subsets are finite. Consider $S$ with $n+1$ elements. 

$f:\mathbb{N}_{n+1} \to S$.\\
If $f(n+1) \in T$, consider $T_1:=T-\{f(n+1)\}$
We have $T_1 \subseteq S-\{f(n+1)\}$, and since $S-\{f(n+1)\}$ is a finite set with $n-1$ elements, from induction hypothesis, $T_1$ is finite. Moreover, since $T=T \cup \{f(n+1) \}$, $T$ is also finite with one more element than $T_1$.\\
If $f(n+1) \not\in T$, then $T \subseteq S-\{f(n+1)\}$, we are done.\\\\
(b) is simply the contrapositive of (a).}

\defn{(Countable Sets)}{A set $S$ is said to be \emph{countable},or \emph{denumerable} if, either $S$ is finite, or $\exists f:\mathbb{N} \to S$ which is a bijection. If $S$ is \emph{not countable}, $S$ is said to be \emph{uncountable}}

\thmp{}{The set $\mathbb{N} \times \mathbb{N}$ is countable.}{The number of points on diagonals $1,2,...l$ are: $\psi(k)=1+2+...k=\frac{k(k+1)}{2}$

The point $(m,n)$ occurs on the $(m+n-1$ th diagonal, on which the number $m+n$ is an invariant. The $(m,n)$ point occurs $m$ points down the diagonal. So, to characterise a point, it is enough to specify the diagonal it falls in, and its ordinate (the "rank" of that point on that diagonal). Count the elements till the $m+n-2$nd diagonal, then add $m$, and this would be the position of the point $(m,n)$.\\
Define $r:\mathbb{N}\times \mathbb{N} \to  \mathbb{N} $ given by $r(m,n)= \psi(m+n-2)+m$
That this is a bijection is pretty clear because we are counting the position of the point $(m,n)$. For a given point $(m,m)$, there can only be one unique diagnonal on which it exists, and on the diagonal, its rank is unique. Moreover, for every $q \in N$, there corresponds an $(m,n)$ such that $r(m,n)=q$, for, we simply count along each diagonal in the "zig-zag" manner until we reach that $(m,n)$ for which the position is given by $q$. Therefore, $r$ is a bijection. (There are other explicit bijections too)}

\thmp{}{The following are equivalent:
\begin{enumerate}
  \item $S$ is countable
  \item $\exists$ a surjective function from $\mathbb{N} \to S$
  \item $\exists$ an injective function from $S \to \mathbb{N}$
\end{enumerate}}{($1\implies2)$ is obvious\\
($2\implies3)$ $f:\mathbb{N}\to S$, every element of of $S$ has atleast one preimage in $\mathbb{N}$. Define a function from $S \to \mathbb{N}$ by taking for each $s \in S$ the least such $n \in \mathbb{N}$ such that $f(n)=s$. This defines an injection.\\
($3\implies1$) If there is an injection from $S \to \mathbb{N}$, then there is a bijection from $S \to $ a subset of $\mathbb{N}$, which implies $S$ is countable.}

\corp{The set of Rational Numbers $\mathbb{Q}$ is countable.}{We know that a surjection from $\mathbb{N} \times \mathbb{N}$ to $\mathbb{Q}$ exists (where $f(0,0)=0$, and $f(m,n)=\frac{m}{n}$).
We know that $\mathbb{N}\times \mathbb{N}$ is bijective to $\mathbb{N}$. This means $\mathbb{N}$ is surjective to $\mathbb{Q}$. We are Done.}

\thmp{}{Every infinite subset of a countable set is countable.}{Consider $N_s \subseteq \mathbb{N}$ which is infinite. 

Define $g(1)=$ least number in $N_s$\\

Having defined $g(n)$, define $g(n+1)$= least number in $N_s$ which is larger than $g(n)$.\\

That it is an injection is obvious, for $g(m)>g(n)$ if $m>n$.\\

Suppose it is not a surjection, i.e, $g(\mathbb{N}) \neq N_s \implies g(\mathbb{N})\subset N_s\implies N_s-g(\mathbb{N}) \neq \phi$\\
Therefore, $N_s-g(\mathbb{N})$ has a least element, $k$. This means that $k-1$ is in $g(\mathbb{N})$. Therefore, there exists $q$ in $\mathbb{N}$ such that $g(q)=k-1$. But then, $g(q+1)$= least number in $N_s$ such that it is bigger than $g(q)$. This would, ofcourse be, $k$, which means $k=g(q+1)$, which puts $k$ in $g(\mathbb{N})$. Contradicton. Hence, $g(\mathbb{N})=N_s$, therefore, $g$ is a bijection from $\mathbb{N} \to N_s$. Since every countable set is bijective to $\mathbb{N}$, and every infinite subset of a countable set is bijective to an infinite subset of $\mathbb{N}$, the theorem holds generally for countable sets.}

\thmp{}{$\mathbb{N} \times \mathbb{N} \cdots \mathbb{N}$ is bijective to $\mathbb{N}$}{$\mathbb{N} \times \mathbb{N}$ is bijective to $\mathbb{N}$ obviously. Assuume that $f:\mathbb{N} \to \mathbb{N} \cdots \mathbb{N} (\text{ n times })$ is bijective.\\

Consider $g:\mathbb{N}\times \mathbb{N}\to \mathbb{N} \cdots \mathbb{N} (n+1\text{ times })$
given by $g(m,n)=(f(m),n)$. Clearly, this is bijective. }



\subsection{Axiom of Choice}
\asum{Axiom of Choice (AC)}{For any collection of non empty sets $C=\{ A_l: l \in L\}$, there exists a function $f$ called the "counting function" which maps each set $A_l$ to an element in $A_l$.\\
Formally: $f:C \to \bigcup_l A_l$ such that $\forall l\in L, f(A_l)\in A_l$}
\thmp{}{Countable union of Countable sets is countable
\rmk{(This theorem is an example of a theorem that requires \emph{Axiom of Choice})}}{Suppose we are given a sequence of countable sets $\{S_n : n\in \mathbb{N}\}$. Since each $S_j$ is countable, we have for each $j$, at least one bijective map $f_j:\mathbb{N} \to S_j$. Define $k:\mathbb{N} \times \mathbb{N} \to \bigcup_j S_j$ given by: $k(m,n)=f_m(n)$. Suppose $x \in \bigcup_j S_j$, i.e, $x \in S_j$ for some $j$. This means that, $f(n)=x$ for some $n$. Therefore, $k(j,n)=x$. Hence, $k$ is surjective. From theorem 2.14, we are done.\\\\
(\textbf{Remark:}{ Keep in mind, for each $S_j$, there are a myriad of functions $(f_j)_k:\mathbb{N}\to S_j$. For each $S_j$, which is countably infinite, we have to choose one of the many functions that biject $\mathbb{N}$ to $S_j$. So we have a countable collection of sets $C=\{E_j: j\in \mathbb{N}\}$, where $E_j$ denotes the set of all functions that biject $\mathbb{N}$ into $S_j$. So for every element in $C$, we need to choose one element in each element of $C$. This is where the Axiom of Choice comes into play.)}}

\thmp{}{If $f:A \to B$ is a surjection, then $B$ is bijective to a subset of $A$}{We are told that $f(A)=B$, i.e, for every $b\in B$, $\exists x_b$(many such $x_b$-s are possible) such that $f(x_b)=b$. Define a functino $g:B \to A$ as: $g(b)=\text{one of those } x_b \text{ such that } f(x_b)=b$. $g$ is bijective to the set of all the chosen $x_b$ for every $b$ }
\rmkb{We make use of the Axiom of Choice in the previous theorem when we choose an $x_b$ from a set of all possible $x_b$-s for $b$. Let $A_b$ be the set of all possible $x_b$-s. Then the collection $\{A_b:b\in B\}$ is a collection of non-empty sets. And we are to select "one" element from each $A_b$. This requires AC.}
\defn{(Power Set of a set)}{Power set of $A$, denoted by $P(A)$ is the set of all subsets of $A$.}

\thmp{Cantor's Theorem}{For any set $A$, there \emph{does not exist} any surjection from $A$ onto $P(A)$}{Suppose, on the contrary, a surjection $\psi:A \to P(A)$ exists. For every subset $A_s$ of $A$, there exists an element $x$ of $A$ such that $\psi(x)=A_s$. Either this $x$ exists in $A_s$, or it doesnt. Conider $D:=\{x \in A: x \not\in \psi(x) \}$. $D$ is a subset of $A$, so there must be some element $y \in A$ such that $\psi(y)=D$. does $y$ belong in $D$? If so, $y \not\in \psi(y)=D$. Which means $y \not\in D$. If, though, $y \not\in D$, that implies $y\not\in \psi(y) \implies y\in D$. Contradictions left and right.}
\newpage

\section{Elementary Results regarding Integers}
\defn{}{\begin{enumerate}
\item (Divides) We say $a \in \ZZ \setminus \{0\}$ divides $b \in \ZZ$ if there exists an integer $\delta$ such that $a\delta=b$. We denote it by $a|b$.
\item (GCD) We call a number $d$ the "Greatest Common Divisor" of two integers $a$ and $b$ if $d|a$ and $d|b$, and $d$ is the largest such number that divides both $a$ and $b$ (that the largest such number exists is clear, since divisors are finite).
\item (LCM) We call a number $l$ the "Least Common Multiple" of two integers $a$ and $b$ if $a|l$ and $b|l$ and $l$ is the smallest such integer. 
\end{enumerate}}

\defn{Prime Number}{A number $p$ in $\NN$ is \emph{prime} if it has only itself and $1$ as divisors. Non-primes are called composite. }

\thmp{}{\begin{enumerate}
\item The GCD $d$ of $a,b \in \ZZ$ is unique, and has the property that, if any other integer $q$ is a divisor of $a$ and $b$, then $q$ divides $d$.
\item The LCM $l$ of $a,b \in \ZZ$ is unique, and has the property that, if another integer $p$ is a multiple of $a$ and $b$, then $l$ divides $p$.
\item If LCM of $a,b$ and GCD of $a,b$ are $l$ and $d$ respectively, then $dl=ab$.
\end{enumerate}}{(1) This will be proved below with the division algorithm. For now note that, if every divisor divides $d$, then $d$ is the GCD.
\\\\(2) This is also proved using the divison algorithm. For now note that if every multiple of $a,b$ is divisible by a multiple $l$, then it is the least common multiple.
\\\\(3) Suppose $d$ is the unique GCD of $a,b$ and $l$ is the unique LCM of $a,b$.\\\\
Note that $d|ab$ which means $dc_0=ab$ for some $c_0$.
$c_0=a(\frac{b}{d})$ and $c_0=b(\frac{a}{d})$. This means $c_0$
is a multiple of $a$ and $b$ which means $l|c_0$. We have
$lq_0=c_0$ for some $q_0$. This means $dlq_0=dc_0=ab$.
$(dq_0)(\frac{l}{a})=b$ and $(dq_0)(\frac{l}{b})=a$ which 
makes $dq_0$ a divisor. This would necessarily mean $q_0=1$, whence, we are done.

}
\subsection{Euclid's Divison Algorithm}
\lemp{The Lemma}{Given integers $a,b \in \ZZ$ with $b \neq 0$, we get a unique $q \in \ZZ$ and $r\in \ZZ$ such that:
$$a=bq+r$$ with $0 \leq r<|b|$.
}{We Prove for the case that $a,b \in \NN$. Assume that for $a_0 \in \NN$, the divison lemma works, 
i.e, $\exists q_0$ and $r_0$ so that $$a_0=bq_0+r_0$$ with $0\leq r<|b|$.
Look at $a_0+1=bq_0+r_0+1$. We have that, either $r_0+1=b$, or $r_0+1<b$. If its the former,
then we see that $a_0+1=bq_0+b=b(q_0+1)+0$ whence we see that the new quotient is $q_0+1$ and the new remainder is $0$.
Hence, by induction, the lemma is proved.\\\\
For the cases where $a<0$ or $b<0$, we can simply multiply by $-1$ to get the result.
}
\textbf{The Algorithm:}
\\\\ We start with $a,b \in \ZZ \setminus \{ 0\}$, and without loss of 
generality, we assume that $a\geq b$. We then have:

$$\centering{a=bq_0+r_0} \text{ with } 0 \leq r_0<|b|$$
$$\centering{b=r_0q_1+r_1} \text{ with } 0 \leq r_1<r_0<|b|$$
$$\centering{r_0=r_1q_2+r_2} \text{ with } 0 \leq r_2<r_1<r_0<|b|$$
$$\centering{r_1=r_2q_3+r_3} \text{ with } 0 \leq r_3<r_2<r_1<r_0<|b|$$
$$\vdots$$
$$r_{n_0-1}=r_{n_0}q_{n_0+1}+r_{n_0+1} \text{ with } 0 \leq r_{n_0+1}<r_{n_0} \cdots b $$
$$r_{n_0}=r_{n_0+1}q_{n_0+2}+r_{n_0+2} \text{ with } 0 \leq r_{n_0+2}<r_{n_0+1} \cdots <b$$
Since we cannot have a sequence of strictly decreasing positive integers, we note that at some point, $r_{n}=0(=r_{n_0+2} \text{ for our sake})$. 

We would then have (in the last step), $$r_{n_0}=r_{n_0+1}q_{n_0+2}+0$$ and back substituting,
$$r_{n_0-1}=r_{n_0+1}(q_{n_0+2}q_{n_0+1}+1)$$
$$r_{n_0-2}=(r_{n_0+1}(q_{n_0+2}q_{n_0+1}+1))q_{n_0}+r_{n_0+1}q_{n_0+2} $$
And finally we would end up with $$a=r_{n_0+1}(\text{something}_1) $$
and $$b=r_{n_0+1}(\text{something}_2) $$ with the added fact that $r_{n_0+1}$ divides every remainder $r_n$ in the division algorithm performed with $a$ and $b$. 
\\\\ \textbf{Proof that any divisor divides the GCD:} Suppose that $z$ is a divisor of $a$ and $b$. 
From $a=bq_0+r_0$, and $$\frac{a}{z}=\frac{b}{z}q_0+\frac{r_0}{z} $$
we see that $z$ divides $r_0$. From $b=r_0q_1+r_1$ and $$\frac{b}{z}=\frac{r_0}{z}q_1+\frac{r_1}{z} $$
we see that $z$ divides $r_1$ too. Suppose $z$ divides all remainders till $r_{n_0}$. $r_{n_0-1}=r_{n_0}q_{n_0+1}+r_{n_0+1}$
gives us $\frac{r_{n_0-1}}{z}=\frac{r_{n_0}}{z}q_{n_0+1} +\frac{r_{n_0+1}}{z}$ whence we see that $r_{n_0+1}$ is 
divisible by $z$. Therefore, $r_{n_0+1}$ is the GCD.
\\\\ \textbf{Proof that any multiple is divisible by LCM:} Suppose that $l$ and $m$ are multiples of $a,b$ and $l$ is the least such multiple.
Then $l \leq m$ with equality case being trivial. Suppose $l<m$. From Euclid's division lemma, we have $m=lq_0+r_0$ with $0<r_0<l<m$. Both $a$ and $b$ divide $m$ and $l$, which means
$$\frac{m}{a}=\frac{l}{a}q_0+\frac{r_0}{a}$$ and $$\frac{m}{a}=\frac{l}{a}q_0+\frac{r_0}{a}$$ Which makes $r_0$ a multiple of $a$ and $b$, which is absurd.
\thmp{Bezout's Theorem}{If $a,b \in \ZZ\setminus \{0\}$, then there exists $x,y \in \ZZ$ so that gcd($a,b$)
=$xa+yb$}{From the first equation we see $$ a=bq_0+r_0 \implies r_0=a-bq_0$$
putting $r_0$ in the 2nd equation we see: $$b=(a-bq_0)q_1+r_1 \implies r_1= b-(a-bq_0)q_1$$
Putting $r_1$ and $r_0$ into the 3rd equation, we get, likewise, $r_2$ in terms of $a$ and $b$ (a linear combination of $a$ and $b$)
As such, we can keep doing this to express $r_{n_0+1}$ as a linear combination of $a$ and $b$, like, $xa+yb$.}

\thmp{Fundamental Theorem of Arithmetic}{Given an integer $a>1$, we can decompose $a$ as a product of primes uniquely (upto ordering)}{If $a=2$, obviously we can. Suppose we can decompose every positive integer $q$ $1<q<n_0$ as a product of primes. Consider $n_0+1=p_1^{\alpha_1}p_2^{\alpha_2}\cdots p_n^{\alpha_n}+1$. Either $p_1^{\alpha_1}p_2^{\alpha_2}\cdots p_n^{\alpha_n}+1$
is a prime or is composite. If it is prime, we are done. If it is a composite number, then $p_1^{\alpha_1}p_2^{\alpha_2}\cdots p_n^{\alpha_n}+1=xy$ where $x$ and $y$ are numbers smaller than $n_0+1$.
But since $x$ and $y$ can be expressed as a product of primes, we see that $n_0+1$ can also be represented as a product of primes.
\\\\ Suppose $a=p_1^{\alpha_1}p_2^{\alpha_2}\cdots p_n^{\alpha_n}$, but also $a=p_1^{\alpha'_1}p_2^{\alpha'_2}\cdots p_m^{\alpha'_m}$.
\\ We have ($m\geq n$): $$1=p_1^{\alpha_1-\alpha'_1}p_2^{\alpha_2-\alpha'_2} \cdots p_n^{\alpha_n-\alpha'_n} p_{n+1}^{-\alpha'_{n+1}} \cdots p_m^{-\alpha'_m}$$
$$1=p_1^{\alpha'_1-\alpha_1}p_2^{\alpha'_2-\alpha_2} \cdots p_n^{\alpha'_n-\alpha_n} p_{n+1}^{\alpha'_{n+1}} \cdots p_m^{\alpha'_m}$$
Obviously, not all $\alpha_j-\alpha'_j$ are positive, in the same way they all aren't negative. Suppose some of these
powers are positive, while some negative. Consider some $p_j^{\delta_j}$ for which the power is negative, which moves it to the denominator.
We have: $$1=\frac{\text{some primes raised to negative}\cdot {\text{ some primes raised to positive} }}{p_j^{\delta_j}}$$
If we multiply the negative powers out to both sides, we get to a stage where we see that $p_j$ divides a product of primes (raised to positive powers).
But, $p_j$ is different from all the primes, and hence does not divide any of them individually, which means,
from the lemma(s) below, that $p_j$ actually does not divide the whole product. A contradiction. Hence, the only case remaining is that of
product of primes raised to $0$ powers, which makes it unique.

$$\textbf{Requisite Lemmas:}$$ 
\\\\Lemma: Suppose $m$ is such that gcd($a,m$)$=1$, and $|ab$. Then, we have that $m|b$.
\\ To see this, we use Bezout's Theorem: there exists $x$ and $y$ integers such that $ax+my=1$.
This means $abx+mby=b \implies \frac{ab}{m}x+by=\frac{b}{m}$ whence it becomes clear.
\\\\Lemma: If a prime $p$ divides $z^n$ for some integer $z$ and some natural $n$, then $p$ divides $z$.
\\ To see this, Suppose that $p$ does not divide $z$. Which means, from the lemma above, that 
$p$ divides $z^{n-1}$, which likewise means it divides $z^{n-2}$ and finally reaching to a contradiction that
it finally divides $z$.
}
If we arrange the primes $p_1,p_2,\cdots p_n$ in ascending order, we find that this representation becomes the only one (no order permutations).
\\\\\textbf{AN ALTERNATE LOOK AT GCD AND LCM:}
\\\\ Suppose that $a=p_1^{q_1} p_2^{q_2}\cdots p_n^{q_n}$ and $b=p_1^{r_1} p_2^{r_2} \cdots p_n^{r_n}\cdots p_m^{r_m}$. Since all numbers whatsoever are
product of primes, we have that every divisor is of the form (for a number $z=p_1^{l_1}p_2^{l_2}\cdots p_k^{l_k}$) $p_1^{j_1}p_2^{j_2}\cdots p_k^{j_k}$
with $j_1\leq l_1, j_2\leq l_2 \cdots j_k \leq l_k$. The number of divisors for a given $z=p_1^{l_1}p_2^{l_2}\cdots p_k^{l_k}$ would therefore be:
${(l_1+1)\choose{1}} {(l_2+1)\choose{1}} \cdots {(l_k+1)\choose{1}}$.
\\\\ We note that, given $a=p_1^{x_1}p_2^{x_2}\cdots p_m^{x_m}$ and $b=p_1^{y_1}p_2^{y_2}\cdots p_m^{y_m}$ with $x_j, y_j \geq 0$,
$$gcd(a,b)=p_1^{min(x_1,y_1)}p_2^{min(x_2,y_2)} \cdots p_m^{min(x_m,y_m)} $$
and 
$$lcm(a,b)=p_1^{max(x_1,y_1)}p_2^{max(x_2,y_2)} \cdots p_m^{max(x_m,y_m)} $$
\\\\ \textbf{Euler Totient Function ($\phi$):}
\\\\ The Euler Totient function $\phi$ is defined as follows:
$$\phi(n)=\text{ number of positive integers } z \text{ less than or equal to } n \text{ such that } gcd(n,z)=1 $$
For primes $p$, every number smaller than $p$ is coprime to $p$, hence $\phi(p)=p-1$
More generally we have for primes $p$ and natural $q$, 
$\phi(p^q)=p^q-p^{q-1}$. 
\\\\ \pf{Consider $\phi(p^k)=||\text{all such numbers }x \text{ such that } x\leq p^k \text{ and }gcd(x,p^k)=1||$. Note that apart from the multiples of $p,p^2,p^3 \cdots$, all else are coprimes with $p^k$. The numbers that are not coprimes with $p^k$ can be counted as: 
$1p,2p,\cdots p(p), \cdots (p^2)p \cdots p^{k-1}p$. Hence, the number of multiples of $p$ in this list are $p^{k-1}$. Total number of "numbers" smaller or equal to $p^k$ are obviously, $p^k$. From here, the answer is obvious. 
}
 \emph{\textbf{Just accept the following fact}}
\fact{If $(m,n)=1$, then $\phi(mn)=\phi(m)\phi(n)$}
\subsection{Modular Arithmetic:}
Given $n \in \NN$ and the set of all integers $\ZZ$, we define the following equivalence
class:
$$aRb \iff n|(b-a) $$
We have $aR$ obviously. If $aRb$, obviously
$bRa$. If $aRb$ and $bRc$, then $n|(b-a)$
and $n|(c-b)$ which means $n|(b-a)+(c-b) \implies n|c-a$.
Hence, $R$ is an equivalce relation. Consider
arbitrary $a \in \ZZ$. A number $x$ is in the 
equivalence class of $a$ under the relation 
defined above if and only if $n|(x-a)$
which means $\exists z\in \ZZ$ so that
$nz=x-a \implies x=a+nz$. A number $x$ is in  
the equivalence class of $a$ under $R$ (also
called as \emph{congruence class of $a\text{mod}n$})
if and only if $x$ is $a$ plus an integral
multiple of $n$.
\\\\ It is also fruitful to view this from the following
perspective: 
\\\\ \textbf{Conjecture:} \emph{$zRa$ (i.e, $n|z-a$) if and only if $z$, when divided by $n$, gives the same remainder as $a$ when divided by $n$.}
\pf{($\implies$) Consider $n|z-a$ which means 
$\exists p$ so that $z=a+pn$. If $a<n$, we have
the remainder when $a$ is divided by $n$, as $a$ itself,
whence we see that the remainder when $z$ is 
divided by $x$ is $a$ as well.
If $a>n$, then $a=qn+r_0$ with $0 \leq r_0<n$.
We then have $z=pn+a=pn+qn+r_0=z_0n+r_0$ where $r_0<n$.
Therefore, same remainder.
\\\\($\impliedby$) Suppose $z$ and $a$
give the same remainder when divided by $n$. 
This means $z=q_z n+ r$ and $a=q_an+r$
which means $z-a=(q_z-q_a)n$ which means 
$n|z-a$.
}
\textbf{Notation:} We say $zRb$ in the above context if 
either $n|z-b$ or $z$ and $b$ share the same remainders
when divided by $n$. The equivalence class of $a$ in this context
is sometimes also called as \emph{congruence class of $a$mod($n$)}.
A symbolic way to say $zRa$ in this context is 
$$z\equiv a \text{mod}(n)$$
We also denote the congruence class (or residue class) of an integer $a$ by $\overline{a}$.
\thmp{}{If $z$ is a given integer, and $n$ is a given
natural number, then the congruence class that
$z$ falls in would be one of the congruence classes 
formed by the $n-1$ numbers before $n$. Therefore, the 
entire partition created by $R$ can be listed out as the 
congruence classes of the $n-1$ numbers before $n$}{Suppose $z<n$. then 
obviously there is an integer $q$ less than $n$ (which is $z$ itself) so 
that $q\equiv z \text{mod}(n)$. If $z>n$, then $z=qn+r$ 
where $0\leq r<n$, which also means that, when $r$ is divided by $n$, the remainder is $r$, just like $z$.
This means $z \equiv r \text{mod}(n)$. From here it is clear that every number in $Z$ would be 
in the equivalence class of some integer less than $n$. It is also obvious that for any two
integers less than $n$, they each form unique residue classes. }

From the previous lemma, it is clear to see that all the residue classes,
or "partitions" of $\ZZ$ under the congruence mod (n) relation, can be succinctly listed out as
$$\overline{0},\overline{1},\overline{2}\cdots \overline{n-1}$$. 
\defn{$\ZZ/{n\ZZ}$}{The set of all equivalence classes (or residue classes) under the relation
defined by $n$ on $\ZZ$ is denoted by $$\ZZ/{n\ZZ} $$ which is basically $\overline{0},\overline{1},\overline{2}\cdots \overline{n-1}$}
\thmp{}{If $a_1\equiv b_1 \text{mod}(n)$ and $a_2\equiv b_2 \text{mod}(n)$, then
$a_1+a_2 \in \overline{b_1+b_2}$ and $a_1 a_2 \in \overline{b_1b_2}$}{(1) We have $a_1=b_1+kn$ and $a_2=b_2+ln$
which gives $a_1+a_2=(k_l)n+b_1+b_2$ where we see that $a_1+a_2$ is integer multiple of $n$ $+$ $b_1+b_2$. Therefore,
$a_1+a_2 \in \overline{b_1+b_2}$.
\\\\ Consider $a_1 a_2=(kn+b_1)(ln+b_2)=(kln^2+b_2kn+b_1ln)+b_1b_2$. Obvious from here.}

\defn{Modular Arithmetic}{Treating the residue classes that form $\ZZ/n\ZZ$ as elements with which arithmetic can be done, we define addition and multiplication as follows:
$$ \overline{a}+\overline{b}=\overline{a+b}$$ i.e, the sum of two residue classes is the residue class of the sum of an element from the class of $a$ and the class of $b$. (This sum is well defined from the previous theorem.)
$$\overline{a}\overline{b}=\overline{ab}$$ i.e, the product of residue class of $a$ and $b$ is the residue class of the product of an element in the class of $a$ and an element from the class of $b$. (Yet again, well defined) }

\defn{$(\ZZ / n\ZZ)^{*}$}{$$(\ZZ / n \ZZ)^{*}:=\{\overline{a}\in \ZZ /n\ZZ: \exists \overline{c}\in \ZZ / n\ZZ \text{ such that } \overline{c} \cdot \overline{a}=\overline{1}\} $$}

\lemp{}{If $gcd(a,n)=1$, i.e, $a$ and $n$ are coprimes, then every element in the residue class of $a \text{mod}(n)$ is coprimes with $n$}{Consider $z=a+kn$ for some $k$ to be "non coprimes with $n$". i.e, $gcd(z,n)=j \neq 1$. We have $j|a+kn$ and $j|n$. But this means directly that $j|a$ whence we see that $j \neq 1$ is a divisor of $a$ and $n$. Absurd.}
\lemp{}{If $a \in \overline{a} \in \ZZ /n \ZZ$, $a\leq n$ and if $gcd(a,n)=j$, then for any $z\in \overline{a}$, $gcd(z,n)=j$.}{
$gcd(a,n)=k$. Consider $gcd(a+gn,n)$. We apply the division 
algorithm:
$$(a+gn)=g_0(n)+a$$
$$ n=g_1a+r_1$$
$$ \vdots$$
Notice that after step (1), the procedure is exactly the same
as the procedure to find $gcd(a,n)$. The last living
remainder is the GCD, and from here we can clearly see 
that the gcd are the same.

}
\lemp{}{If $1\leq a \leq n$, with $n \geq 2$, and $(a,n)=j\neq 1 (\geq 2)$, then there exists a number $1 \leq b<n$ so that $\overline{a} \cdot \overline{b}=\overline{0}$. A corollary of this is that there exists \emph{no} $c \in \ZZ$ so that $\overline{a}\cdot \overline{c}=\overline{1}$}{We know $gcd(a,n)=j \neq 1(\geq 2)$. This means $j\alpha=a$, $j\mu=n$ where $\alpha<a$ and $\mu<n$. We can see that $a\mu=j\mu \alpha=n\alpha$. So the $b$ in the lemma is the $\mu$ here. Therefore, $\overline{a} \cdot \overline{b}=\overline{0}$. Suppose $ac\equiv 1 mod(n)$. This means from the multiplicative property of 
mod(n), we have $ac(b) \equiv b mod(n)$, but if you commute the multiplication, we see $$c(ab)=\overline{c} \cdot \overline{ab}=\overline{c}\cdot \overline{0}\equiv 0 mod(n) $$ But this would imply $b\equiv 0 mod(n)$ which is not true. Hence, there can be no $c$ so that $ac \equiv 1 mod(n)$.}
\prop{$\ZZ /n\ZZ^*$ is the same as $\{\overline{a} \in \ZZ/n\ZZ:
(a,n)=1 \}$}
\pf{Let $$A=\{\overline{a} \in \ZZ/ n\ZZ:
\exists \overline{z} \in \ZZ/ n\ZZ \text{ such that }  
\overline{z} \cdot \overline{a}=\overline{1}\}$$ and 
$$B=\{\overline{b} \in \ZZ/ n\ZZ:(b,n)=1\}$$
Consider an arbitrary $\overline{a}$ in $B$ such that $\forall z \in \overline{a}$ we have
$gcd(z,n)=1$. We know that $z=a+k_z n$. From Bezout Identity,
we have $x_z$ and $y_z$ so that $x_z(a+k_zn)+y_z(n)=1$
which gives us $x_za=(-y_z-k_zx_z)(n)+1$. We see That
$x_za \in \overline{1}$. Consider arbitrary $t \in \overline{x_z}$.
We have $t=x_z+jn$ or $x_z=t-jn$. Plugging this back we have
$x_za=(t-jn)a=(-y_z-k_z(t-jn))n+1$ whence we can easily see that $ta \in \overline{1}$.
Therefore, $\overline{a} \cdot \overline{x}=\overline{1}$. Therefore, if 
$\overline{b}$ is such that $(b,n)=1$, then an inverse exists for it. Hence, $B \subseteq A$.
\\\\ We showed that if $x$ is in $B$, it must be in $A$. Consider $x$ not in $B$. i.e, it is not coprimes with $n$. From the previous proposition, we see that there would exist no $c$ so that $\overline{x}\cdot \overline{c}=\overline{1}$, i.e, no inverse element for any $x$ not in $B$. Inexistence in $B$ therefore implies inexistsnce in $A$, which gives us $A \subseteq B$. We can conclude that $A=B$ or in pithy words: 
\\ \begin{center}"The set of all congruence classes $b$ so that $gcd(b,n)=1$ is the same as the set of all congruence classes $b$ so that there exists another congruence clas $c$ so that $\overline{b} \cdot \overline{c}=\overline{1}$ "
\end{center}

Symbolically: 
$$\{\overline{b} \in \ZZ/ n\ZZ: gcd(b,n)=1 \}=\{\overline{b} \in \ZZ/n\ZZ: \exists \overline{c} \in \ZZ/n\ZZ: \overline{b}\cdot \overline{c}=\overline{1} \}=(\ZZ/n\ZZ)* $$
}

\thmp{Generalised Bezout's Identity}{Suppose $j_1, j_2 \cdots j_n$ are given integers such that $gcd(j_1,j_2 \cdots j_n)=d$, then there exists integers $x_1,x_2 \cdots x_n$ so that $x_1j_1+x_2j_2 \cdots x_nj_n=d$}{We prime decompose each of $j_1,j_2 \cdots j_n$. For $j_k$, we have $$j_k=p_1^{q^k_1}p_2^{q^k_2}\cdots p_{n_k}^{q^k_{n_k}} $$
We note that the gcd of all these numbers are the minimum of each index for each prime multiplied throughout. An important bit of information will be cricual here: $$ min\{w_1,w_2 \cdots w_k,w_{k+1} \cdots w_n \}=min\{ min\{w_1,w_2 \cdots w_k \}, min\{w_{k+1},w_{k+2} \cdots w_n \} \}$$
So we split $gcd(j_1,j_2, \cdots j_n)$ into $gcd(gcd(j_1,j_2),gcd(j_3 \cdots ,j_n))=$ some linear combination of $j_1,j_2$ and $gcd(j_3, \cdots j_n)$. Do the same split to $gcc(j_3, \cdots j_n)$, to keep getting linear combinations, until finally you find $gcd(j_1,j_2 \cdots j_n)=x_1j_1+x_2j_2 \cdots x_nj_n$
}


\end{document}
