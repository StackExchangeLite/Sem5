\documentclass[../Main.tex]{subfiles}

\begin{document}
\begin{center}\textbf{ASSIGNMENT-1}\end{center}

1) Consider $H=\{1,a,b,c\}$. Consider an arbitrary $x,y \in H$. Is $xy=1?$ If so, it would be commutative. If suppose $xy \neq 1$. Is $xy=x$ or $y$? Neither, since that would result in either $x$ or $y$ being $1$. Therefore, $xy=z$ where $z$ is the element different from $x,y$. Same way, the argument can be extended to $yx=z$. Hence, either $xy=1$, or $xy=yx=z$, which makes $\{1,a,b,c \}$ an abelian group.
\\\\ Let us now classify the groups of order $4$. We can immediately fill up the initial parts:
\begin{center}

\begin{tabular}{ c c c c c }
x & 1 & a & b & c \\
1 & 1 & a & b & c \\
a & a & x & x & x \\
b & b & x & x & x \\
c & c & x & x & x \\
\end{tabular}
\end{center}
Since this is a finite group of order $4$, there should be atleast one element with order $2$. We WLOG select that element to be $b$ so that $b^2=1$. Is $ab=a$ or $b$? Nope, since that would make either one identity. So $ab=c$. Is $ba=a$ or $b$? In much the same way, we conclude $ba=ab=c$.
$b(ba)=bc=a$ and $(ab)b=ab^2=cb$. Hence $bc=cb=a$. So far we got: (This is applicable for any group of size $4$, since we did not use the property that this group has no element with order $4$.)

\begin{center}

    \begin{tabular}{ c c c c c }
    x & 1 & a & b & c \\
    1 & 1 & a & b & c \\
    a & a & x & c & x \\
    b & b & c & 1 & a \\
    c & c & x & a & x \\
    \end{tabular}
    \end{center}
(\textbf{The Klein Route}) Suppose our group has no element with order $4$. Is $a^2=b$? Can't be, because then, since $b^2=1$, we'd have $a^4=1$ which is against hypothesis. Hence $a^2=1, $ or $a^2=c$. Likewise, we can conclude that $c^2=1$ or $c^2=a$ (Ask the same questions, is $c^2=b$? No). Suppose $a^2=1$ and $c^2=a$. That would make $c^4=1$, which is against hypothesis. Hence, if $a^2=1$ then $c^2=1$ as well. Likewise, if $c^2=1$, then $a^2=1$ as well. Suppose neither, i.e, $c^2=a$ and $a^2=c$. Then $c^4=a^2=c$ and $a^4=c^2=a$. We have $a^3=1$ and $c^3=1$. $(ba)a^2=b$ which means $ca^2=b \implies c^2=b$. But $c^2=a$. Absurd. Hence, this scenario is impossible. Hence, for the Klein route, $a^2=c^2=1$. 
\\\\ Question for $ac$ and $ca$, then arises. Is $ac=1$? That would mean $a^2c=1c=a$, absurd. Hence, $ac=b$. Similarly, is $ca=1$? we would then have $c=a$ again. Therefore, $ac=ca=b$. This completes the Klein Route: 
\begin{center}

    \begin{tabular}{ c c c c c }
    \textbf{x} & \textbf{1} & \textbf{a} & \textbf{b} & \textbf{c} \\
    \textbf{1} & 1 & a & b & c \\
    \textbf{a} & a & 1 & c & b \\
    \textbf{b} & b & c & 1 & a \\
    \textbf{c} & c & b & a & 1 \\
    \end{tabular}
\end{center}
(\textbf{The $\mathbf{\ZZ/4\ZZ}$ Route}) Suppose that $G$ has an element of order $4$. Since the size of the cyclic subgroup of this element is $4$ as well, this group is cyclic. WLOG, assume that $G=\langle a \rangle$. Then every element is $1$, $a=a$, $a^2=b$, $a^3=c$. We have (for a general 4 membered group)

\begin{center}

    \begin{tabular}{ c c c c c }
    x & 1 & a & b & c \\
    1 & 1 & a & b & c \\
    a & a & x & c & x \\
    b & b & c & 1 & a \\
    c & c & x & a & x \\
    \end{tabular}
\end{center}

Since the group is cyclic, we can immediately write $a^2=b$. Since $a^3=c$, $a^6=a^2=c^2=b$. We can write that in as well. All that is left is $ac$ and $ca$. Let us rule out the obvious: $ac \neq a$, $ca \neq a$, $ac \neq c$, $ca \neq q$. Is $ac=b$? That would mean $a^4=b$, which makes $b=1$. Same way, $ca \neq b$. Hence, $ac$ and $ca$ have only one option left, $1$. We can fill that in to get the $\ZZ/ 4\ZZ$ isomorph:


\begin{center}

    \begin{tabular}{ c c c c c }
    x & 1 & a & b & c \\
    1 & 1 & a & b & c \\
    a & a & b & c & 1 \\
    b & b & c & 1 & a \\
    c & c & 1 & a & b \\
    \end{tabular}
\end{center} 
Note that Klein is the unique 4 membered group with no element of order $4$. $\ZZ/4\ZZ$ isomorph is the unique group with one element with order $4$. 
\begin{center}    
    \rule[1ex]{.8\textwidth}{1pt}\\
    \rule[1.9ex]{.7\textwidth}{.6pt}\\[-6pt]
    
\end{center}
2) Suppose that $\forall x \in G$, $x^2=1$. Suppose there exists $a,b$ so that $ab \neq ba$. This means $a^2b \neq aba \implies b \neq aba$. This then means that $b^2 \neq (ba)(ba)=(ba)^2$. But this boils down to $1 \neq 1$. Absurd. Hence, $\forall a,b \in G$, $ab=ba$.
\begin{center}    
    \rule[1ex]{.8\textwidth}{1pt}\\
    \rule[1.9ex]{.7\textwidth}{.6pt}\\[-6pt]
    
\end{center}
3) Suppose $a \in G$. Consider the case when order of $a$ is finite= $n$. $a^n=1$. $a \cdot a \cdot ... a =1$. Note that $(a_1 a_2 \cdots a_n)^{-1}=a_n^{-1}a_{n-1}^{-1} \cdots a_1^{-1}$ which means that the inverse of $a^n$ is $a^{-n}=(a^{-1})^n=1$. $(a^{-1})^n=1$ would mean that the actual order of $a^{-1}$ is a divisor of $n$. Say order of $a^{-1}$ is $m$. We have $m|n$. But the argument can be reversed with $a^-1$ as some $b$ and $a $ as some $b^{-1}$ to get $n|m$ which gives $m=n$. Hence $\mathfrak{O}(a)=\mathfrak{O}(a^{-1})$.
\begin{center}    
    \rule[1ex]{.8\textwidth}{1pt}\\
    \rule[1.9ex]{.7\textwidth}{.6pt}\\[-6pt]
    
\end{center}
4) Consider $\ZZ / 6 \ZZ \setminus\{0\}$. Let us define multiplication as $\overline{a} \times \overline{b}= \overline{(a \times b)}$. Consider the element $\overline{5}$. $3 \times 1=3 mod(6)$. $3 \times 2=6 mod 6=0mod(6)$, $3 \times 3=3 mod 6$. $3 \times 4=12=0 mod 6$. $3 \times 5=15=3 mod 6$. Hence, we note that $3$ has no inverse. Hence, $\ZZ_6$ is not a group. 
\\\\ Consider $\ZZ_7$. We know the classification of the elements in the multiplicative $(\ZZ_n)^*$, which is $$(\ZZ_{n\ZZ})^*:=\{z \in \ZZ/ n\ZZ: \exists c \in \ZZ/ n\ZZ: \overline{c}\overline{z}=\overline{1}\}$$ Which is actually equivalent to saying
$$(\ZZ_{n\ZZ})^*:=\{z \in \ZZ/ n\ZZ: gcd(z,n)=1\}$$ And we note that since $7$ is a prime, every element smaller than $7$ is coprimes with $7$ which means that every element of $\ZZ/ 7\ZZ-\{0\}$ is in the multiplicative group.
\begin{center}    
    \rule[1ex]{.8\textwidth}{1pt}\\
    \rule[1.9ex]{.7\textwidth}{.6pt}\\[-6pt]
    
\end{center}
5) $U_n:=$ the multiplicative $\ZZ_n$ which is $\{z \in \ZZ_n: \exists c \in \ZZ_n: cz=1\}$, which can be rewritten as $\{z \in \ZZ_n: gcd(z,n)=1\}$. Define $Aut(\ZZ_n,+)$ of the group $\ZZ_n$ as the set of all group isomorphisms from $\ZZ$ to itself, seen as a group under addition $+$. 
\\\\ Look at $aut(\ZZ_n,+)$, the set of all $\phi:\ZZ_n \to \ZZ_n$ such that it is a bijection (invertible) and is a homomorphism, i.e, $\forall x \in \ZZ_n$, we have $\phi(a+b)=\phi(a)+\phi(b)$. Let $\phi(a)=a'$. Define $\phi^{-1}(a')=a$. This is aswell a bijection. Consider $x,y \in \ZZ_n$ so that $\phi(a)=x$, $\phi(b)=y$. We know such unique $a$ and $b$ exist, since it is a bijection. Have a look at $\phi^{-1}(x)$ and $\phi^{-1}(y)$. We see that $\phi^{-1}(x)=a$ and $\phi^{-1}(y)=b$. What is $\phi^{-1}(x+y)$? $\phi^{-1}(\phi(a+b))=a+b$. Hence, $\phi^{-1}$ is aswell a group isomorphism. Now consider $\phi \circ \phi^{-1}$ and $\phi^{-1} \circ \phi$ which is $I$ the identity map. The identity map is an isomorphism. Therefore, for every element $\phi$ in Aut($\ZZ_n, +$), there exists an inverse $\phi^{-1}$. Consider $\phi$ and $\psi$ two group isomorphisms in aut. $\phi(a+b)=\phi(a)+\phi(b)$ and $\psi(a+b)=\psi(a)+\psi(b)$. Consider $\psi \circ \phi(a+b)=\psi(\phi(a)+\phi(b))=\psi \circ \phi(a)+\psi \circ \phi(b)$. Hence, composition operation is also a group isomorphism (since bijectivity is preserved whenever we compose two bijections). Therefore, $Aut(\ZZ_n,+)$ is a group. This is true for any group as well.
\\\\ Consider the map $\gamma: Aut(\ZZ_n,+) \to U_n$ given by $\gamma(\psi \in Aut(\ZZ_n,+))=\psi(\overline{1})$
We have 
$\gamma(\psi)=\psi(\overline{1})$. $\gamma(\varphi)=\varphi(\overline{1})$. What is $\gamma(\psi \circ \varphi)$? it is $\psi \circ \varphi (\overline{1})=\psi(\varphi(\overline{1}))$. What is $\gamma(\psi) \cdot \gamma(\varphi)$? It is $\psi(1) \cdot \varphi(1)$. Is $\psi(\varphi(1))=\psi(1) \cdot \varphi(1)$? 
\\\\ We need to stop to understand that if $\phi$ is an automorphism from $\ZZ_n$ to itself, it must map $1$ to an element $\overline{j} \in \ZZ_n$ so that $gcd(j,n)=1$. To understand this, we note that $\ZZ_n$ is a cyclic group generated by $1$. But we know that $\ZZ_n=\langle 1^x \rangle$ if and only if $gcd(x,n)=1$. So $j$ can generate $\ZZ_n$ if and only if $(j,n)=1$. Moreover, if we are to preserve structure in the homomorphism $\phi$, we need to map generators to generators. To see this, suppose $1 \mapsto k$ where $gcd(k,n)\neq 1$. Note that when we define a homomorphism on the generator, it is basically defined for every other element for $\phi(x)= \phi(1+1+1 \cdots 1)=x\phi(1)$. If $gcd(k,n)\neq 1$, then $\langle j \rangle $ will be a proper subgroup of $\ZZ_n$, meaning it will miss out on a few elements in $\ZZ_n$. Suppose $\phi(1)=j$, this means $\phi(x)=j^x= x(j)$ (in the context of an additive group). But All the multiples of $j$ do not cover the entire group. Hence, $\phi$, an automorphism, maps $1$ to $j$ so that $gcd(j,n)=1$ or equivalently, $\phi$ maps $1$ to an element in the multiplicative $(\ZZ_n)^{*}$ (since every element in the multiplicative group has its gcd with $n$ to be $1$). Notice that if $a \in \ZZ_n^*$ and $b \in \ZZ_n^*$, then $ab \in \ZZ_n^*$. 
\\\\ Now we can answer the question, is $\psi(\varphi(1))=\psi(1)\varphi(1)$? $\psi(1)=k_1$ so that $(k_1,n)=1$ and $\varphi(1)=j_1$ so that $(j_1,n)=1$. $\psi(j_1)=j_1\psi(1)$ from the generator definition, hence we see that $\psi(\varphi(1))=j_1 \times k_1$=$\psi(1) \times \phi(1)$. Hence, this is a group homomorphism. 
\\\\ Now to show that $\gamma$ is bijective, we note that, for a $\phi: \ZZ_n \to \ZZ_n$ to be an isomorphism, one needs to send a generator to a generator. i.e, different isomorphisms can be generated by sending $1$ to each element $j$ so that $(j,n)=1$. Moreover, if we send $1$ to a \emph{non} generator, i.e an element with non unity gcd with $n$, then that ceases to be a group isomorphism since group order wouldn't be preserved. Hence, there are $\text{totient}(n)$ elements in $Aut(\ZZ_n,+)$, which is the same as the size of the multiplicative group $\ZZ_n^{*}$. Hence, $\gamma: aut(\ZZ_n,+) \to U_n$ given by $\gamma(\psi)=\psi(1)$ is a group isomorphism.
\begin{center}    
    \rule[1ex]{.8\textwidth}{1pt}\\
    \rule[1.9ex]{.7\textwidth}{.6pt}\\[-6pt]
    
\end{center}
6) Let $B_n:=\{r \in \ZZ_n: gcd(r,n)=1\}$
If $r, s \in B_n$ then $gcd(r,n)=gcd(s,n)=1$. Consider $rs:= \overline{rs} \in B_n$. $xr \equiv 1 mod(n)$ and $ys \equiv 1 mod(n)$. This means $xy rs \equiv 1 mod(n)$ which means $xy rs=1 mod(n)$ which means $xy rs +zn=1$. This means that $gcd(rs,n)=1$.
\\\\ \textbf{Claim:} Suppose $gcd(x,n)=1$, and $x< n$ with $1<n$, then for any $z$ so that $z \in \overline{x}$, we have $gcd(z,n)=1$. 
\\ \begin{proof}
Consider $z=rn+x$. We then have $n=px+q$ and we keep going to find the gcd as the final remainder in the process of euclid's division algorithm. Therefore, the $gcd$ of $z$ and $n$ as well, is $1$. 
\end{proof}

\textbf{Claim:} Suppose $gcd(x,n)=j \neq 1(\geq 2)$ and $1 \leq x <n$ and $1<n$. Then, the claim is that there exists $1<b<n$ so that $xb\equiv 0 mod(n)$. This would then imply that there would exist no $z<n$ so that $xz\equiv 1mod(n)$.
\\ \begin{proof}
We know that $gcd(x,n)=j$ which means $pj=x$ and $qj=n$ with $p<x$ and $1<q<n$. $pqj=np=xq$. Therefore, $xq\equiv 0 mod(n)$ with $1<q<n$. Suppose there exists some $l$ so that $lx\equiv 1 mod(n)$. This means that $lxq\equiv l(xq) \equiv 0 mod(n) \equiv q mod(n)$ which would imply $q\equiv 0 mod(n)$ which is absurd since $1<q<n$. Hence, no $z<n$ exists so that  
\end{proof}  
\end{document}