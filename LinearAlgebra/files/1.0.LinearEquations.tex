
\documentclass[../Main.tex]{subfiles}

\begin{document}
\chapter{Linear Equations}

\intro{
   We'll go through the Basic formalisms and proof, along with few elementary operations and matrix theory.
}

\section{Fields}

\defn{Fields}{
   A field $\mathcal{F}$ is a set of objects that satisfy the following the properties.
   \begin{enumerate}
      \item $\forall$ x, y $\in$ $\mathcal{F}$, $x+y = y+x$.
      \item $\forall$ x, y $\in$ $\mathcal{F}$, $x+(x+y)=(x+y)+z$.
      \item $\forall$ x $\in$ $\mathcal{F}$, $\exists$! 0 $\in$ $\mathcal{F}$, s.t. $x+0=x$. 
      \item $\forall$ x $\in$ $\mathcal{F}$, $\exists$! $(-x)$ $\in$ $\mathcal{F}$, s.t. $x+(-x)=0$.
      \item $\forall$ x, y $\in$ $\mathcal{F}$, $xy=yx$.
      \item $\forall$ x, y, z $\in$ $\mathcal{F}$, $x(yz)=(xy)z$.
      \item $\forall$ x $\in$ $\mathcal{F}$, $\exists$! 1 $\in$ $\mathcal{F}$, s.t. $x1=x$.
      \item \label{multiplicative inverse} $\forall$ non-zero $x$ $\in$ $\mathcal{F}$, $\exists$! $x^{-1}$ $\in$ $\mathcal{F}$, s.t. $x^{-1}x=1$.
      \item $\forall$ x, y, z $\in$ $\mathcal{F}$, $x(y+z)=xy+xz$.
   \end{enumerate}}
\rmkb{All fields are closed under addition and multiplication.}
\defn{Subfields}{
   A field $\mathcal{S}$ is a subfield of a field $\mathcal{F}$ if, all $x$ $\in$ $\mathcal{S}$, also $\in$ $\mathcal{F}$.
}
\textbf{Examples:}

$\mathbb{R}$, the field of real numbers is a subfield of $\mathbb{C}$.

$\mathbb{Q}$, the field of rational numbers is a subfield of $\mathbb{R}$.

\textbf{Non-examples:}

$\mathbb{N}$, the set of positive integers, is not a subfield of $\mathbb{R}$, since 0 $\notin \mathbb{N}$.

$\mathbb{Z}$, the set of integers, is not a subfield of $\mathbb{R}$ (multiplicative inverses not included in field)

\defn{Characteristic of a Field}{
   (this part needs to be re-written for clarity)

   The least $n$ such that $n 1's$ is 0 is called the Characteristic of the field $\mathcal{F}$. If it does not happen in a field $\mathcal{F}$, then it is called characteristic zero.
}

\section{Systems of Linear Equations}
A collection of $m$ linear equations with $n$ unknowns, where $y_1, y_2, \dots y_{m}$ and $A_{ij}$, $1 \leq i \leq m, 1 \leq j \leq n,$ are given elements of F. 
$$\begin{matrix}
   A_{11}x_{1} &+& A_{12}x_{2} &+ &\cdots&+& A_{1n}x_{n} &=& y_{1}\\
   A_{21}x_{1} &+& A_{22}x_{2} &+ &\cdots&+& A_{2n}x_{n} &=& y_{2}\\
   \vdots\\
   A_{m1}x_{1} &+& A_{m2}x_{2} &+ &\cdots&+& A_{mn}x_{n} &=& y_{m} 
\end{matrix}$$
Any n-tuple $(x_1, x_2, \dots, x_n)$ of elements in $\mathcal{F}$ which satisfies each of the equations above is called a \textbf{Solution} of the system.\\
If $y_1=y_2=\dots=y_m=0$, then we say that the system of linear equations is \textbf{Homogeneous}.

\fact{\textbf{Equivalent Systems}: Two systems of linear equations are said to be Equivalent if each equation in each system is a linear combination of the equations in the other system.
}
\thmp{}{Equivalent systems of linear equations have exactly the same solutions}{I mean, it is clear from fact 2.1, since they are linear combinations one can be converted into another, hence the solutions are exactly same.}

\rmk{
   TODO: (once i understand characteristic of a field thing) Prove that each field of chracteristic zero contains a copy of the rational number field
}



\end{document}
