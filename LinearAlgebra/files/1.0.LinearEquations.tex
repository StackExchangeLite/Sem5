
\documentclass[../Main.tex]{subfiles}

\begin{document}
\chapter{Harry Potter and the Linear Equations}

\intro{
   We'll go through the Basic formalisms and proof, along with few elementary operations and matrix theory.
}

\section{Fields}

\defn{Fields}{
   A field $\mathcal{F}$ is a set of objects that satisfy the following the properties.
   \begin{enumerate}
      \item $\forall$ x, y $\in$ $\mathcal{F}$, $x+y = y+x$.
      \item $\forall$ x, y $\in$ $\mathcal{F}$, $x+(x+y)=(x+y)+z$.
      \item $\forall$ x $\in$ $\mathcal{F}$, $\exists$! 0 $\in$ $\mathcal{F}$, s.t. $x+0=x$. 
      \item $\forall$ x $\in$ $\mathcal{F}$, $\exists$! $(-x)$ $\in$ $\mathcal{F}$, s.t. $x+(-x)=0$.
      \item $\forall$ x, y $\in$ $\mathcal{F}$, $xy=yx$.
      \item $\forall$ x, y, z $\in$ $\mathcal{F}$, $x(yz)=(xy)z$.
      \item $\forall$ x $\in$ $\mathcal{F}$, $\exists$! 1 $\in$ $\mathcal{F}$, s.t. $x1=x$.
      \item \label{multiplicative inverse} $\forall$ non-zero $x$ $\in$ $\mathcal{F}$, $\exists$! $x^{-1}$ $\in$ $\mathcal{F}$, s.t. $x^{-1}x=1$.
      \item $\forall$ x, y, z $\in$ $\mathcal{F}$, $x(y+z)=xy+xz$.
   \end{enumerate}}
\rmkb{All fields are closed under addition and multiplication.}
\defn{Subfields}{
   A field $\mathcal{S}$ is a subfield of a field $\mathcal{F}$ if, all $x$ $\in$ $\mathcal{S}$, also $\in$ $\mathcal{F}$.
}
\textbf{Examples:}

$\mathbb{R}$, the field of real numbers is a subfield of $\mathbb{C}$.

$\mathbb{Q}$, the field of rational numbers is a subfield of $\mathbb{R}$.

\textbf{Non-examples:}

$\mathbb{N}$, the set of positive integers, is not a subfield of $\mathbb{R}$, since 0 $\notin \mathbb{N}$.

$\mathbb{Z}$, the set of integers, is not a subfield of $\mathbb{R}$ (multiplicative inverses not included in field)

\defn{Characteristic of a Field}{
   (this part needs to be re-written for clarity)

   The least $n$ such that $n 1's$ is 0 is called the Characteristic of the field $\mathcal{F}$. If it does not happen in a field $\mathcal{F}$, then it is called characteristic zero.
}

\section{Systems of Linear Equations}

A collection of $m$ linear equations with $n$ unknowns, where $y_1, y_2, \dots y_{m}$ and $A_{ij}$, $1 \leq i \leq m, 1 \leq j \leq n,$ are given elements of F. 

\begin{equation}
   \begin{matrix} \label{linear eq sys}
      A_{11}x_{1} &+& A_{12}x_{2} &+ &\cdots&+& A_{1n}x_{n} &=& y_{1}\\
      A_{21}x_{1} &+& A_{22}x_{2} &+ &\cdots&+& A_{2n}x_{n} &=& y_{2}\\
      \vdots\\
      A_{m1}x_{1} &+& A_{m2}x_{2} &+ &\cdots&+& A_{mn}x_{n} &=& y_{m}
   \end{matrix}
\end{equation}

Any n-tuple $(x_1, x_2, \dots, x_n)$ of elements in $\mathcal{F}$ which satisfies each of the equations above is called a \textbf{Solution} of the system.\\
If $y_1=y_2=\dots=y_m=0$, then we say that the system of linear equations is \textbf{Homogeneous}.

\fact{
   \textbf{Equivalent Systems}: Two systems of linear equations are said to be Equivalent if each equation in each system is a linear combination of the equations in the other system.
}
\thm{}{
   Equivalent systems of linear equations have exactly the same solutions
}

\section{Matrices and Elementary Row Operations}

A $m \times n$ \textbf{matrix over a field} $\mathcal{F}$ is a function A from the set of pairs of integers $(i,j)$, $1 \leq i \leq m$, $1 \leq j \leq n$, into the field $\mathcal{F}$. 

\begin{equation}
   A =
   \begin{bmatrix} \label{Matrix A}
      A_{11} & A_{12} & \dots & A_{1n}\\
      A_{21} & A_{22} & \dots & A_{2n}\\
      \vdots &&& \vdots\\
      A_{m1} & A_{m2} & \dots & A_{mn}
   \end{bmatrix}
\end{equation}

\rmkb{
   The system of linear equations \ref*{linear eq sys} can be represented using matrices as $AX=Y$, where
   \begin{equation}
   X=
   \begin{bmatrix}
      x_0\\
      x_1\\
      \vdots\\
      x_n\\
   \end{bmatrix}
   Y=
   \begin{bmatrix}
      y_1\\
      y_2\\
      \vdots\\
      y_m\\
   \end{bmatrix}
   \end{equation}
   .
}

\subsection{Elementary Row Operations}

\fact{
   An \textbf{elementary rwo operation} is a special kind of function (rule) $e$ which associates each $m \times n$ matrix $A$ with a $m \times n$ matrix $e(A)$. We can precisely describe e in the three cases as follows:
   \begin{enumerate}
      \item for $i \neq r$, when $e(A_{rj}) = cA_{rj}$, then $e(A_{ij}) = A_{ij}$ holds.
      \item for $i \neq r$, when $e(A_{rj}) = A_{rj} + A_{sj}$, then $e(A_{ij}) = A_{ij}$ holds.
      \item for $i \neq r$ and $i \neq s$, when $e(A_{rj}) = A_{sj}$, then $e(A_{ij}) = A_{ij}$ holds.
   \end{enumerate}
   Here $A_{rj}$ and $A_{sj}$ are $r$th and $s$th rows of the matrix.
}
\begin{remark}
   The elementary row operation $e$ is defined on a class of all $m \times n$ matrices over $\mathcal{F}$, for some fixed $m$ but any $n$.
\end{remark}

\thmp{}{
   To each elementary row operation $e$ there is corresponds an elementary row operation $e_1$, of the same type as $e$, such that $e_1(e(A)) = e(e_1(A)) = A$ for each $A$. In other words, the inverse operation (function) of an elementary row operation exists and is an elementary row operation of the same type.
}{
   \begin{enumerate}
      \item Suppose e is the operation which multiplies the $r$th row of a matrix by the non-zero scalar $c$. Clearly, there exists a $c^{-1}$ multipying with which will bring back the matrix A.
      \item Suppose $e$ is the operation which replaces row $r$ by the row $r$ plus $c$ times the row $s$, $r \neq s$. Let $e_1$ be the operation which replaces row $r$ by row $r$ plus $(-c)$ times row $s$.
      \item If e interchange rows $r$ and $s$, let $e_1=e$.
   \end{enumerate}
   In each of these three cases we clearly have a $e_1(e(A)) = e(e_1(A)) = A$  for each $A$.
}

\defn{}{
   If $A$ and $B$ are $m \times n$ matrices over a field $\mathcal{F}$, we can say that $B$ is row equivalent of \textit{$A$ if $B$ can be obtained from $A$ by a finite sequence of elementary row operations.} 
}

\thmp{}{
   If $A$ and $B$ are two row-equivalent $m \times n$ matrices, the Homogeneous systems of linear equations $AX=0$ and $BX=0$ have exactly the same solutions. 
}{ 
   Suppose we pass $A$ through some $k$ number of elementary operations to B.
      \begin{equation}
         A \to A_0 \to A_1 \to \dots \to A_k = B
      \end{equation}
   Since these are homogeneous system of linear equations it is clear that the elementary row operations do not distrub the set of solutions.\\
   Note that, no matter what elementary operation you perform each equation in the system $BX = 0$ will be a linear combination of the equations in the system $AX = 0$.\\
   Since the inverse of an elementary row operation is an elementary row operation, each equation in $AX = 0$ will also be a linear combination of the equations in $BX = 0$.\\
   Hence these two are equaivalent, and by Theorem 1, they have the same solutions.
}

\defn{Row Reduced Matrix}{ An $m \times n$ matrix $R$ is called \textbf{Row Reduced} if:
\begin{enumerate}
   \item \textit{the first non-zero entry in each non-zero row of $R$ is equal to 1;}
   \item \textit{each column of $R$ which contains the leading non-zero entry of some row has all its entries 0.}
\end{enumerate}
}

\defn{Identify Matrix}{
A $n \times n$ matrix defined by
   \begin{equation}
      \mathcal{I}_{ij} = \delta_{ij} = 
      \begin{cases} 
         0 & \text{if } i = j, \\
         1 & \text{if } i \neq j.
      \end{cases}
   \end{equation}
The $\delta$ is also known as the \textbf{kronecker delta}. 
}

\thmp{}{Every $m \times n$ matrix over the field $\mathcal{F}$ is a row-equivalent to a row-reduced matrix.}{}

\section{Row-Reduced Echleon Matrices}

\defn{RREF}{
   An $m \times n$ matrix $\mathcal{R}$ is called a row-reduced echeleon matrix if:
   \begin{enumerate}
      \item $\mathcal{R}$ is row-reduced.
      \item every row of $\mathcal{R}$ which all the entries zero occur below every row which has non-zero entries.
      \item if rows 1, $\dots$, $r$ are the non-zero rows of $\mathcal{R}$, and if the leading non-zero entry of row $i$ occurs in column $k_i$, $i = 1, \dots, r$ then $ k_1 < k_2 < \dots < k_r.$
   \end{enumerate}
}

\cor{
   An \textbf{alternate definition} for row-reduced echeleon matrix $\mathcal{R}$ is as follows:\\
      Either every entry in $\mathcal{R}$ is 0, or $\exists$ a positive integer $r$, $1 \leq r \leq m$, and $r$ positive intergers $k_1, \dots, k_r$ with $1 \leq k_i \leq n$ and
      \begin{enumerate}
         \item $R_{ij} = 0$ for $i>r$, and $R_{ij} = 0$ if $j < k_i$.
         \item $R_{i{k_j}} = \delta_{ij}$, $1 \leq i \leq r$, $1 \leq j \leq r$.
         \item $k_1 < \dots < k_r$. 
      \end{enumerate}
}

\thmp{Existance of a RREF for every $m \times n$ matrix}{
   Every $m \times n$ matrix $\mathcal{A}$ is row-equivalent to a row-reduced echeleon matrix.
}{
   proof follows from theorem 3.7, i will formulate it properly soon.
}

\thmp{}{
   If $\mathcal{A}$ is an $m \times n$ matrix and $m < n$, then the homogeneous system of linear equations $\mathcal{A}X=0$ has a non-trivial solution.
}{
   Let $\mathcal{R}$ be a row-reduced echeleon matrix which is row-equaivalent to $\mathcal{A}$. Then the systems $\mathcal{A}X=0$ and $\mathcal{R}X=0$ have the same solutions by theorem 3.4. If $r$ is the number of non-zero rows in $\mathcal{R}$, then certainly $r \leq m$, and since $m < n$, we have $r < n$. For since $r < n$, we can choose some $x_j$ which is not among the r unknowns $x_{k_1}, \dots, x_{k_n}$ and we can then construct a solution to show that a solution $(x_1, \dots, x_n)$ in which not every $x_j$ is 0. It follows that $\mathcal{A}X = 0$ has a non-trivial solution. 
}

\thmp{}{
   If $\mathcal{A}$ is an $n \times n$ (square) matrix, then $\mathcal{A}$ is row-equaivalent to the $n \times n$ identity matrix if and only if the system of equations $\mathcal{A}X$=0 has only the trivial solution.
}{
   If $\mathcal{A}$ is row-equivalent to $\mathcal{I}$, then $\mathcal{A}X=0$ and $\mathcal{I}X=0$ have the same solutions. Conversely, suppose $\mathcal{A}X=0$ has only the trivial solution $X=0$. Let $\mathcal{R}$ be an $n \times n$ row-reduced echeleon matrix which is row-equivalent to $\mathcal{A}$, and let $r$ be the number of non-zero rows of $\mathcal{R}$. Then $\mathcal{R}X=0$ has no non-trivial solution. Thus $r \geq n$. But since $\mathcal{R}$ has $n$ rows, certainly $r \leq n$, and we have r = n. Since this means that $\mathcal{R}$ actually has a leading non-zero entry of 1 in each of its $n$ rows, and since these 1's occur each in a different one of the n-columns, $\mathcal{R}$ must be the $n \times n$ identity matrix.
}

\defn{Augmented Matrix}{
   The Augmented Matrix $\mathcal{A}'$ of a system $\mathcal{A}X=Y$ is the $m \times (n+1)$ matrix whose first n columns are the columns of $\mathcal{A}$ and last column is Y. More precisely,
   \begin{itemize}
      \item $\mathcal{A}'_{ij} = \mathcal{A}_{ij}$, if $l \leq n$
      \item $\mathcal{A}'_{i(n+1)} = y_i$
   \end{itemize}
}





































\end{document}
