\documentclass[../Main.tex]{subfiles}

\begin{document}
\chapter{Sets, Real and Complex Number Systems}

\section{Prelimnaries}

\defn{Prelimnary definitions}{
\begin{enumerate}
    \item (Cartesian Product): if $A$ and $B$ are non empty sets, the \emph{Cartesian Product} $A \times B$ is defined as the set of ordered pairs $a,b$ wherein $a \in A, b \in B$. i.e, $ A \times B := \{ (a,b): a \in A, b \in B  \}$ 
    \item (Function): A function from $A$ to $B$ is a set $f \subseteq A \times B$ such that, $a,b \in f \text{ and } a,c \in f \implies b=c$. A is called the \textbf{Domain of $f$}. $Range(f):=f(A)$ (see next definition)
    \item (Direct Image): Direct image $f(A):=\{y \in B: \exists x\in A \text{ such that } f(x)=y \}$
    \item (Inverse Image):$f^{-1}(S\subseteq B):=\{x \in A: f(x) \in S\}$
    \item (Relation): Any subset $R\subseteq A \times B$ is a relation from $A$ to $B$.\\We say $x \in X$ is "related to" $y \in Y$ under the relation $R$, or simply $xRy$ or $R(x)=y$ if $(x,y) \in R \subseteq X \times Y$. 
    \item (Injection): $f: A \to B$ is injective if $\forall x_1,x_2 \in A, (x_1,b) \in f \text{ and } (x_2,b) \in f \iff x_1=x_2$
    \item (Surjection): $f:A \to B$ is surjective if $\forall b \in B, \exists a \in A \text{ such that } f(a)=b$
    \item (Bijection): $f:A\to B$ is bijective if its both surjective and injective.
    \item {(Identity function on a set)}: $I_A: A \to A$ defined by $\forall x \in A, I_A(x)=x$
    \item  (Permutation): Simply a bijection from $A$ to itself is called a permutation.
\end{enumerate}
}

\defn{(Left Inverse)}{ We say $f:A \to B$ has a left inverse if there is a function $g: B \to A$ such that $g \circ f=I_A$}
\thmp{}{$f:A \to B$ has a left inverse if and only if it is injective.}{
$\implies$) If $f$ has a left inverse $g$, Consider $x,y \in A$ such that $f(x)=f(y)=p$. 

We have $g \circ f(x)=g(p)=x=g \circ f(y)=y$. Hence, $x=y$, Injective.\\
$\impliedby$) Given that $f:A \to B$ is injective, define $g:B \to A$ as: 
$$
g(z \in B)=
\begin{cases}
a, \text{ where } f(a)=z, \ \text{if } z\in f(A) \\
\text{whatever}, \ \text{if } z \not\in f(A)\end{cases}
$$
consider $g \circ f (x \in A) = g \circ (f(x))$.

Obviously, $f(x) \in f(A)$, therefore, $g(f(x))=$ that $a$ such that $f(a)=f(x)$. 

That $a$ is $x$. 
Hence, $g(f(x))=x$
}
\defn{(Right Inverse)}{$f:A \to B$ is said to have a right inverse if there is a function $g:B \to A$ such that $f \circ g=I_B$}

\thmp{}{$f:A \to B$ has a right inverse if and only if $f$ is Surjective.}
{$\implies$) If $f$ has a right inverse $g$, such that $f \circ g=I_B:B \to B$, then it is evident that the range of $f$ is B, for if not, range of $f \circ g$ wouldn't be $B$ either.\\
$\impliedby$) If $f$ is surjective, then for all $b \in B$, there exists atleast one $a \in A$ such that $f(a)=b$ define $g$ as:
$$g(x \in B)= \text{one of those a} \in A \text{ such that} f(a)=b$$
Consider $f \circ g (x \in B) = f ( \text{ one of the } a \text{ such that } f(a)=b)=b, \forall b \in B$

Hence, $f \circ g=I_B$
}

\thmp{}{If $f$ has left inverse $g_1$ and right inverse $g_2$, then $g_1=g_2$.
\rmk{(True for anything that is Associative, and function composition is associative.)}}{$g_1 \circ f=I_A$ and $f \circ g_2 =I_B$

$g_1 \circ (f \circ g_2)=g_1 \circ I_B= g$

$= (g_1\circ f) \circ g_2=I_A \circ g_2=g_2$

Hence $g_1=g_2$}
\corp{ $f$ is invertible (i.e, both left and right inverse exist) if and only if it is bijective.}{Obvious}


\subsection{Operations on Relations}
If $R$ and $S$ are binary relations over $X \times Y$:
\begin{enumerate}
  \item $R \bigcup U := \{(x,y) | xRy \text{ or } xSy \}$
  \item $R \bigcap S := \{(x,y) | xRy \text{ and } xSy \}$
  \item Given $S:Y \to Z$ and $R:X \to Y$, $S \circ R:=\{(x,z) | \exists y \text{ such that } ySz \ \& \ xRy \}$
  \item If $R$ is binary over $X \times Y$, $\bar{R} := \{(x,y) | \neg (xRy)\}$
\end{enumerate}

\subsection{Homogeneous Relations}
If $R$ is a binary relation over $X \times X$, it is Homogeneous.
\defn{Definitions Regarding Relations}{
\begin{enumerate}
  \item (Reflexive): $\forall x \in X, xRx$
  \item (Symmetric): $\forall x,y \in X, xRy \implies yRx$
  \item (Transitive): $\forall x,y,z \in X, \text{ if } xRy \ \& \ yRz \implies xRz$
  \item (Dense): $\forall x,y \in X, \ \text{if} \ xRy, \text{ then there is some } z \in X \text{ such that } xRz \ \& \ zRy$
  \item (\textbf{Equivalence Relation}): $R$ is an equivalence relation if it is Reflexive, Symmetric and Transitive.
  \item (Equivalence class of $a\in A$(where there is an equivalence relation defined)): Set of all $b \in A$ such that $bRa$. 
  \item (Partition of $A$): Any collection of sets $\{A_i :i\in I \}$ (where $I$ is some indexing set) such that: $$A= \bigcup_{i\in I} A_i$$ $$A_i \cap A_j = \phi \text{ if } \forall i,j \in I, i \neq j$$
\end{enumerate}
  }

\thmp{}{Let $A$ be a non-empty set. If $R$ defines an equivalence Relation on $A$, then the set of all equivalence classes of $R$ form a partition of A}{Define our collection $\{ A_{\alpha}\}$ as the set of all equivalence classes of $A$. Clearly, $\bigcup_{\alpha \in I} A_{\alpha}=A$. If $A$ only has one element, obviously, that singleton set makes up the partition. Let $A_{\alpha}$ and $A_{\alpha'}$ be equivalence classes of two elements $a$ and $a'$ in $A$. If $a R a'$, then $A_{\alpha}=A_{\alpha'}$ since every element in the equivalence class of $a$ will, from the transitive property, be in the equivalence class of $a'$ . Suppose $\neg (a R a')$. If, then, $\exists x \in A_{\alpha}$ such that $x\in A_{\alpha'}$, this means that $xR\alpha$ and $xR\alpha'$, but from transitive property, this means $\alpha R\alpha'$, which is a contradiction. Therefore, the pairwise intersection is disjoint.}

\thmp{}{If $\{A_i: i \in I \}$ is a partition of A, then there exists an equivalence relation $R$ on $A$ whose equivalence classes are $\{A_i:i \in I\}$.}{ Define $R(x,y)$ if and only if $\exists$ unique $m \in I$ such that $x \in A_m$ and $y \in A_m$.
    

$R(x,x)$ is obvious if non empty, hence $R$ is reflexive.\\\\   
Suppose $R(x,y)$ and $R(y,z)$. Then, there exists a unique $m \in I$ such that $x,y$ are in $A_m$. Similarly, there exists a unique $n \in I$ such that $y,z$ are in $A_n$ . Obviously, if $n \neq m$, intersection of $A_n$ and $A_m$ would be non empty, hence, $n=m$. Hence, $R$ is transitive.\\\\
Consider $R(x,y)$, which means $\exists$ unique $n\in I$ such that $x,y \in A_n \implies R(y,x)$. 
Hence, $R$ is an equivalence relation.}

\section{Induction, Naturals, Rationals and the Axiom of Choice }
\asum{Peano Axioms, characterisation of $\mathbb{N}$}{\begin{enumerate}
  \item $1\in \mathbb{N}$
  \item every $n \in \mathbb{N}$ has a predecessor $n-1 \in \mathbb{N}$ except $1$
  \item $\text{if } n \in \mathbb{N} \implies n+1 \in \mathbb{N}$
\end{enumerate} }
\defn{(Sequence of something)}{A sequence of some object is simply a collection of objects $\{O_l: l\in \mathbb{N}\}$ which can be counted.}

\asum{Well Ordering Property of $\mathbb{N}$}{Every non empty subset of $\mathbb{N}$ has a least element.}

\asum{Weak Induction}{For all subsets $S \subseteq \mathbb{N}$, (($1\in S$)\&(($\forall k \in \mathbb{N}$)($k\in S \implies k+1 \in S$))
$\iff$ $S=\mathbb{N}$)\\\\\textbf{Weak Induction's Negation}:(One direction)

There exists subset $S_0 \subseteq \mathbb{N}$, (($1\in S_0$)\&(($\forall k \in \mathbb{N}$)($k\in S_0 \implies k+1 \in S_0$))
but $S_0 \neq \mathbb{N}$)}

\asum{Strong Induction}{For all subsets $S \subseteq \mathbb{N}$, (($1\in S$)\&(($\forall k \in \mathbb{N}$)(${1,2,...k}\in S' \implies k+1 \in S'$))
$\iff$ $S=\mathbb{N}$)\\\\ \textbf{Strong Induction's Negation}:(One direction)

There exists subset $S' \subseteq \mathbb{N}$, (($1\in S'$)\&(($\forall k \in \mathbb{N}$)(${1,2,...k}\in S' \implies k+1 \in S'$))
but $S' \neq \mathbb{N}$)}

\thmp{}{Weak Induction $\iff$ Strong Induction.}{$\implies$) Suppose Weak induction is true, but not strong induction. Take our set to be that $S'$ in the negation of the Strong Induction Statement. $S' \neq \mathbb{N}$ implies that, either $1 \not\in S'$ or $\exists k \in \mathbb{N}$ such that $k\in S'$ but $k+1 \not\in S'$. We know that $1 \in S'$, so it must be that $\exists k \in \mathbb{N}$ such that $k \in S'$ but $k+1 \not\in S'$. $\{1\} \in S' \implies \{1,2\} \in S'$. Assume that for $n$, $\{1,2,..n\} \in S'$. This means that $\{1,2,...n+1\} \in S'$. This means that for every $n \in \mathbb{N}$, $\{1,2,...n \}\in S' \implies n \in S'$. Contradiction.\\
$\impliedby$)Suppose Strong Induction is true, but not weak induction. Take the set $S_0$ from the negation of Weak Induction. $S_0 \neq \mathbb{N}$. This means, from strong induction, either $1\not\in S_0$ or $\exists k \in \mathbb{N}$ such that $1,2, \dots , k \in S_0$ but $k+1 \not\in S_0$. $1 \in S_0$, hence, $2 \in S_0$ and $\{1,2\}\in S_0$. assume that $\{1,2,....n\} \in S_0$. This means, $n\in S_0 \implies n+1 \in S_0$, which means that $\forall k \in \mathbb{N}, \{1,2,..k\} \in S_0 \implies k+1 \in S_0$. Therefore, $S_0$ is $\mathbb{N}$.}

\thmp{}{Weak Induction $\iff$ Strong Induction $\iff$ Well ordering.}{$\implies$) Suppose that, on the contrary, $S_0$ is a non empty subset of $\mathbb{N}$ , with no least element. Does $1$ exist in $S_0$? No, for that will be the least element. Likewise, then, $2$ does not belong in $S_0$. Assume that $\{1,2,...n\} \not\in S_0$. Does $n+1$ exist in $S_0$? No, for that will become the least element then. From Strong Induction, $\mathbb{N}-S_0=\mathbb{N}\implies S_0=\phi$. Contradiction.\\
$\impliedby$)Suppose $\exists S_0 \subseteq \mathbb{N}$ such that $1 \in S_0$ and $\forall k \in \mathbb{N}$, $k\in S_0 \implies k+1 \in S_0$. Suppose on the contrary, $S_0$ is not $\mathbb{N}$. $\mathbb{N}-S_0$ is then, non-empty. From Well Ordering, there is a least element $q \in \mathbb{N}-S_0$. $\implies, q-1 \in S_0$. But this would imply $q-1+1 \in S_0$. Contradiction. $\mathbb{N}-S_0$ is empty.}

\defn{(Finite Sets)}{A set $X$ is said to be finite, with $n$ elements in it, if $\exists n \in \mathbb{N}$ such that there exists a bijection $f: \{1,2..,n\} \to X$. Set $X$ is \emph{infinite} if it is non-finite.}


\thmp{}{If $A$ and $B$ are finite sets with $m$ and $n$ elements respectively, and $A \cap B=\phi$, then $A \cup B$ is finite, with $m+n$ elements.}{$f:\mathbb{N}_m \to A$ and $g:\mathbb{N}_n \to B$.

Define $h:\mathbb{N}_{m+n} \to A \cup B$ given by:
$$h(i)=\begin{cases}f(i) \text{ if } i=1,2...m\\
g(i-1) \text{ if } i=m+1,m+2,...m+n

\end{cases}$$

If $i=1,2,...m$,$h(i)$ covers all the elements in $A$ through $f$. If $i=m+1,...m+n$, $h(i)$ covers all the elements in $B$ through $g$.

Moreover, $h(i) \neq h(j); i \in [1,m], j\in[m+1,m+n] $ since $A\cap B= \phi$}

\thmp{}{If $C$ is infinite, and $B$ is finite, then $C-B$ is infinite.}{Suppose $C-B$ is finite. We have $B \cap (C-B)=\phi$ and $B \cup (C-B)=C \cup B$

$n(C \cup B)=n(B \cup (C-B))=n(B)+n(C-B)$
This implies $C \cup B$ is finite. Contradiction.
}

\thmp{}{\textbf{Theorem}: Suppose $T$ and $S$ are sets such that $T \subseteq S$. Then:\\

a) If $S$ is finite, $T$ is finite.\\

b)If $T$ is infinite, $S$ is infinite.}{Given that $S$ is finite, there is a function $f:\mathbb{N}_{m} \to S$. Suppose that $S$ has $1$ element. Then either $T$ is empty, or $S$ itself, which means $T$ is finite. Suppose that, upto $n$, it is true that, if $S$ is finite with $n$ elements, all its subsets are finite. Consider $S$ with $n+1$ elements. 

$f:\mathbb{N}_{n+1} \to S$.\\
If $f(n+1) \in T$, consider $T_1:=T-\{f(n+1)\}$
We have $T_1 \subseteq S-\{f(n+1)\}$, and since $S-\{f(n+1)\}$ is a finite set with $n-1$ elements, from induction hypothesis, $T_1$ is finite. Moreover, since $T=T \cup \{f(n+1) \}$, $T$ is also finite with one more element than $T_1$.\\
If $f(n+1) \not\in T$, then $T \subseteq S-\{f(n+1)\}$, we are done.\\\\
(b) is simply the contrapositive of (a).}

\defn{(Countable Sets)}{A set $S$ is said to be \emph{countable},or \emph{denumerable} if, either $S$ is finite, or $\exists f:\mathbb{N} \to S$ which is a bijection. If $S$ is \emph{not countable}, $S$ is said to be \emph{uncountable}}

\thmp{}{The set $\mathbb{N} \times \mathbb{N}$ is countable.}{The number of points on diagonals $1,2,...l$ are: $\psi(k)=1+2+...k=\frac{k(k+1)}{2}$

The point $(m,n)$ occurs on the $(m+n-1$ th diagonal, on which the number $m+n$ is an invariant. The $(m,n)$ point occurs $m$ points down the diagonal. So, to characterise a point, it is enough to specify the diagonal it falls in, and its ordinate (the "rank" of that point on that diagonal). Count the elements till the $m+n-2$nd diagonal, then add $m$, and this would be the position of the point $(m,n)$.\\
Define $r:\mathbb{N}\times \mathbb{N} \to  \mathbb{N} $ given by $r(m,n)= \psi(m+n-2)+m$
That this is a bijection is pretty clear because we are counting the position of the point $(m,n)$. For a given point $(m,m)$, there can only be one unique diagnonal on which it exists, and on the diagonal, its rank is unique. Moreover, for every $q \in N$, there corresponds an $(m,n)$ such that $r(m,n)=q$, for, we simply count along each diagonal in the "zig-zag" manner until we reach that $(m,n)$ for which the position is given by $q$. Therefore, $r$ is a bijection. (There are other explicit bijections too)}

\thmp{}{The following are equivalent:
\begin{enumerate}
  \item $S$ is countable
  \item $\exists$ a surjective function from $\mathbb{N} \to S$
  \item $\exists$ an injective function from $S \to \mathbb{N}$
\end{enumerate}}{($1\implies2)$ is obvious\\
($2\implies3)$ $f:\mathbb{N}\to S$, every element of of $S$ has atleast one preimage in $\mathbb{N}$. Define a function from $S \to \mathbb{N}$ by taking for each $s \in S$ the least such $n \in \mathbb{N}$ such that $f(n)=s$. This defines an injection.\\
($3\implies1$) If there is an injection from $S \to \mathbb{N}$, then there is a bijection from $S \to $ a subset of $\mathbb{N}$, which implies $S$ is countable.}

\corp{The set of Rational Numbers $\mathbb{Q}$ is countable.}{We know that a surjection from $\mathbb{N} \times \mathbb{N}$ to $\mathbb{Q}$ exists (where $f(0,0)=0$, and $f(m,n)=\frac{m}{n}$).
We know that $\mathbb{N}\times \mathbb{N}$ is bijective to $\mathbb{N}$. This means $\mathbb{N}$ is surjective to $\mathbb{Q}$. We are Done.}

\thmp{}{Every infinite subset of a countable set is countable.}{Consider $N_s \subseteq \mathbb{N}$ which is infinite. 

Define $g(1)=$ least number in $N_s$\\

Having defined $g(n)$, define $g(n+1)$= least number in $N_s$ which is larger than $g(n)$.\\

That it is an injection is obvious, for $g(m)>g(n)$ if $m>n$.\\

Suppose it is not a surjection, i.e, $g(\mathbb{N}) \neq N_s \implies g(\mathbb{N})\subset N_s\implies N_s-g(\mathbb{N}) \neq \phi$\\
Therefore, $N_s-g(\mathbb{N})$ has a least element, $k$. This means that $k-1$ is in $g(\mathbb{N})$. Therefore, there exists $q$ in $\mathbb{N}$ such that $g(q)=k-1$. But then, $g(q+1)$= least number in $N_s$ such that it is bigger than $g(q)$. This would, ofcourse be, $k$, which means $k=g(q+1)$, which puts $k$ in $g(\mathbb{N})$. Contradicton. Hence, $g(\mathbb{N})=N_s$, therefore, $g$ is a bijection from $\mathbb{N} \to N_s$. Since every countable set is bijective to $\mathbb{N}$, and every infinite subset of a countable set is bijective to an infinite subset of $\mathbb{N}$, the theorem holds generally for countable sets.}

\thmp{}{$\mathbb{N} \times \mathbb{N} \cdots \mathbb{N}$ is bijective to $\mathbb{N}$}{$\mathbb{N} \times \mathbb{N}$ is bijective to $\mathbb{N}$ obviously. Assuume that $f:\mathbb{N} \to \mathbb{N} \cdots \mathbb{N} (\text{ n times })$ is bijective.\\

Consider $g:\mathbb{N}\times \mathbb{N}\to \mathbb{N} \cdots \mathbb{N} (n+1\text{ times })$
given by $g(m,n)=(f(m),n)$. Clearly, this is bijective. }



\subsection{Axiom of Choice}
\asum{Axiom of Choice (AC)}{For any collection of non empty sets $C=\{ A_l: l \in L\}$, there exists a function $f$ called the "counting function" which maps each set $A_l$ to an element in $A_l$.\\
Formally: $f:C \to \bigcup_l A_l$ such that $\forall l\in L, f(A_l)\in A_l$}
\thmp{}{Countable union of Countable sets is countable
\rmk{(This theorem is an example of a theorem that requires \emph{Axiom of Choice})}}{Suppose we are given a sequence of countable sets $\{S_n : n\in \mathbb{N}\}$. Since each $S_j$ is countable, we have for each $j$, at least one bijective map $f_j:\mathbb{N} \to S_j$. Define $k:\mathbb{N} \times \mathbb{N} \to \bigcup_j S_j$ given by: $k(m,n)=f_m(n)$. Suppose $x \in \bigcup_j S_j$, i.e, $x \in S_j$ for some $j$. This means that, $f(n)=x$ for some $n$. Therefore, $k(j,n)=x$. Hence, $k$ is surjective. From theorem 2.14, we are done.\\\\
(\textbf{Remark:}{ Keep in mind, for each $S_j$, there are a myriad of functions $(f_j)_k:\mathbb{N}\to S_j$. For each $S_j$, which is countably infinite, we have to choose one of the many functions that biject $\mathbb{N}$ to $S_j$. So we have a countable collection of sets $C=\{E_j: j\in \mathbb{N}\}$, where $E_j$ denotes the set of all functions that biject $\mathbb{N}$ into $S_j$. So for every element in $C$, we need to choose one element in each element of $C$. This is where the Axiom of Choice comes into play.)}}

\thmp{}{If a $f:A \to B$ is a surjection, then $B$ is bijective to a subset of $A$}{We are told that $f(A)=B$, i.e, for every $b\in B$, $\exists x_b$(many such $x_b$-s are possible) such that $f(x_b)=b$. Define a functino $g:B \to A$ as: $g(b)=\text{one of those } x_b \text{ such that } f(x_b)=b$. $g$ is bijective to the set of all the chosen $x_b$ for every $b$ }
\rmkb{We make use of the Axiom of Choice in the previous theorem when we choose an $x_b$ from a set of all possible $x_b$-s for $b$. Let $A_b$ be the set of all possible $x_b$-s. Then the collection $\{A_b:b\in B\}$ is a collection of non-empty sets. And we are to select "one" element from each $A_b$. This requires AC.}
\defn{(Power Set of a set)}{Power set of $A$, denoted by $P(A)$ is the set of all subsets of $A$.}

\thmp{Cantor's Theorem}{For any set $A$, there \emph{does not exist} any surjection from $A$ onto $P(A)$}{Suppose, on the contrary, a surjection $\psi:A \to P(A)$ exists. For every subset $A_s$ of $A$, there exists an element $x$ of $A$ such that $\psi(x)=A_s$. Either this $x$ exists in $A_s$, or it doesnt. Conider $D:=\{x \in A: x \not\in \psi(x) \}$. $D$ is a subset of $A$, so there must be some element $y \in A$ such that $\psi(y)=D$. does $y$ belong in $D$? If so, $y \not\in \psi(y)=D$. Which means $y \not\in D$. If, though, $y \not\in D$, that implies $y\not\in \psi(y) \implies y\in D$. Contradictions left and right.}
\newpage

\section{The Real and Complex Fields}
\defn{ (Field ($F,+,\cdot)$)}{set $F$, along with two functions $+:F \times F \to F$ and $\cdot:F \times F \to F$ is called a field if:
\begin{enumerate}
    \item $\forall x,y \in F$, $x+y \in F$ (closed under addition)
    \item $\forall x,y \in F$, $x+y=y+x$ (commutative under addition)
    \item $\forall x,y,z \in F$, $x+(y+Z)=(x+y)+z$ (assiciative under addition)
    \item $\exists 0 (\text{ additive identity}) $ such that $\forall x \in F, x+0=0+x=x$ (Additive identity)
    \item $\forall x\in F, \exists (-x) (\text{ additive inverse})$ such that $x+(-x)=(-x)+x=0$ (Additive inverse)
    \item $\forall x,y \in F, x\cdot y \in F$ (Multiplication is closed)
    \item $\forall x,y \in F, x\cdot y=y\cdot x$ (Multiplication is commutative)
    \item $\forall x,y,z \in F, x\cdot (y\cdot z)=(x \cdot y) \cdot z$ (Multiplication is associative)
    \item $\forall x \in F, x \neq 0, \exists (x^{-1})$ such that $x \cdot x^{-1}=x^{-1} \cdot x=1$ (Multiplicative inverse)
    \item $\forall x \in F, 1\cdot x= x\cdot 1=x$ (Multiplicative identity)
    \item $\forall a,b,c \in F, a\cdot (b+c)=a\cdot b + a\cdot c$ (Left Distributivity)
    \item $\forall a,b,c \in F, (a+b) \cdot c=a\cdot c + b\cdot c$ (Right Distributivity)
\end{enumerate} }
\subsection{The Reals $\mathbb{R}$}
The real numbers are characterised by the following axioms:
\defn{(The Real Field)}{
\begin{enumerate}
    \item $F$, field axioms listed above.
    \item Order Axioms:There exists a subset $P\subset \mathbb{R}$ called "positive numbers" such that:
        \begin{enumerate}
            \item $\forall a \in \mathbb{R},$ only one of $a\in P$, $-a \in P$ or $a=0$ are true. (Trichotomy Law)
            \item $\forall a,b \in P, a+b \in P$ (Positive numbers are closed under addition)
            \item $\forall x,y \in P, x \cdot y \in P$ (Positive numbers are closed under multiplication)

        \end{enumerate}
    \item Completeness axiom*
\end{enumerate}}
\defn{}{
\begin{enumerate}
    \item (Max of a set) $M \in S \subseteq \mathbb{R}$ is said to be the maximum of $S$ if $M \geq x \forall x \in S$
    \item (Min of a set) $m\in S\subseteq \mathbb{R}$ is said to be the minimum of a set if $m \leq x \forall x \in S$
    \item (Upper bound of a set) $L\in \mathbb{R}$ is said to be an upper bound of $S$ if $L \geq x \forall x\in S$
    \item (Lower bound of a set) $l\in \mathbb{R}$ is said to be a lower bound of $S$ if $l \leq x \forall x \in S$
    \item (Sup(S)) $\alpha\in\mathbb{R}$ is said to be the supremum of $S$ if it is the minimum of the set of all Upper bounds of $S$.
    \item (Inf(S)) $\beta \in \mathbb{R}$ is said to be the infimum of $S$ if it is the maximum of the set of all lower bounds of $S$.
\end{enumerate}

}

\asum{Completeness Axiom for the Real Number Field}{Every non empty subset of $\mathbb{R}$ that is bounded above has a supremum.}
\corp{Every non empty subset of $\mathbb{R}$ that is bounded below has an infimum.}{

$S \subseteq \mathbb{R}$ has a lower bound $\implies$
$\exists L \in \mathbb{R}$ such that $L \leq x \forall x \in S \implies$
$-L \geq -x \forall x \in S$
Define $S':=\{-x: x\in S\}$. Then, $-L$ is an upper bound of $S'$.\\
$\implies \exists q \in \mathbb{R}$ such that 
$q \geq y \forall y\in S'$ and $q \leq z \forall z$ such that $z \geq y \forall y \in S'$.

$\implies \exists q'=-q \in \mathbb{R}$ such that $q' \leq x \forall x \in S$ and $q' \geq z'=-z \forall z$ such that $z' \leq x \forall x \in S$\\
Cleaning up a bit: $\implies \exists q'\in \mathbb{R}$ such that $q' \leq x \forall x \in S$ and $q' \geq z' \forall z'$ such that $z' \leq x \forall x \in S$\\
Therefore $q'$ is the greatest lower bound, i.e the infimum.
} 

\lemp{The Lemma}{Suppose $a\in \mathbb{R}^+$ and $0\leq a< \varepsilon$ $\forall \varepsilon \in \mathbb{R}^+ $ then $a=0$ }{Suppose not, i.e, $a>0$. Choose $\varepsilon=\frac{a}{2}$. Contradiction.}
\propp{Supremum of a set is unique.}{Supremum is the "least" of the set of the upper bounds, it itself being part of the set of all the upper bounds. Since minima is unique, Supremum is unique. }
\lemp{}{$U \in \mathbb{R}$ is the supremum of $S \subseteq \mathbb{R} \iff$ \begin{enumerate}
    \item $s\leq U \forall s \in S$
    \item if $v<U, \exists s_v \in S$ such that $v<s_v$
\end{enumerate}}{$\implies)$ Given that $U$ is the supremum, (1) is pretty obvious since it is an upper bound. Suppose $v < U$, but for every $s \in S$, $s \leq v$. This would mean $v$ is the supremum, and not $U$. Absurd. }

\fact{Given that $U$ is an upper bound of $S$, $U$ is the supremum of $S \iff \forall \varepsilon >0, \exists s_{\epsilon} \in S$ such that $U-\varepsilon<s_{\epsilon}$}

\thmp{Archimedean Property of $\mathbb{R}$}{Given $a,b \in \mathbb{R}^+, \exists n \in \mathbb{N}$ such that $an-b>0$}{Suppose not, i.e, $\forall n \in \mathbb{N}, an<b \implies \forall n \in \mathbb{N}, n < \frac{b}{a}$. Consider the set $S=\{ an: n \in \mathbb{N} \}$. This has an upper bound $b$, and therefore, a supremum $u$. consider $u-n$. $\exists n_0$ such that $u-n< an_0 \implies u<a(n_0+1)$. Absurd.}
\cor{Alternate formulation of the previous statement: $\forall x \in \mathbb{R}, \exists n_0 \in \mathbb{N}$ such that $x<n_0$}

\lemp{Useful Lemma}{$\forall \varepsilon>0, x< \varepsilon \iff \forall n\in \mathbb{N}, x<\frac{1}{n}$}{$\implies$)Contrapositive to prove would be: $\exists n_0 \in \mathbb{N}, x \geq \frac{1}{n_0} \implies \exists \varepsilon_0>0, x \geq \varepsilon_0$. Simply choose $\varepsilon_0=\frac{1}{n_0}$\\
$\impliedby$)Contrapositive to prove would be: $\exists \varepsilon_0>0$ such that $x \geq \varepsilon \implies \exists n_0$ such that $x \geq \frac{1}{n_0}$. From Archimedean, $\exists n_0$ such that $n_0 \geq \frac{1}{\varepsilon_0} \implies \varepsilon_0 \geq \frac{1}{n_0} \implies x \geq \varepsilon_0 \geq \frac{1}{n_0}$}

\thmp{Archimedean Properties of $\mathbb{R}$}{\begin{enumerate}
    \item $inf(\{\frac{1}{n}: n \in \mathbb{N} \})=0$
    \item If $t>0, \exists n_0 \in \mathbb{N}$ such that, $0 < \frac{1}{t}<n$
    \item If $y>0$, $\exists n_y \in \mathbb{N}$ such that $n_y-1 \leq y < n_y$
\end{enumerate}}{1)Obvious\\2)Application of Archimedean\\3) We know from archimedean that such an $n_y$ exists such that $y<n_y.$ Consider the set of all $n$ such that $y<n$. Obviously, this is a non empty set. Therefore, from Well Ordering, this has a least element $n_0 \implies n_0 \leq y<n_0$}

\thmp{Density of $\QQ$ in $\RR$}{$\forall x,y \in \RR, x<y \implies \exists q \in \QQ$ such that $x<q<y$}{$y-x>0 \implies$ from archimedean $\exists n_0$ such that $n_0(y-x)>1\implies n_0y>1+n_0x$. Form Archimedean Propery, $\exists m \in \NN$ such that $ m-1< n_0y \leq m$.Since $m\geq n_0y>1+n_0x \implies m-1>n_0x \implies n_0y > m-1 \leq n_0x \implies y >\frac{(m-1)}{n_0}>x$}
\cor{Given $x,y \in \RR, y>x$, $\exists q \in \RR-\QQ$ such that $y>q>x$ (Assumed that $\sqrt2$ is irrational.)}

\thmp{Existence of $n$th Roots in $\RR^+$}{let $y \in \RR^+$ and $n \in \NN$, then $\exists$ a unique $x \in \RR^+$ such that $x^n=y$}{Consider $E:=\{ t\in \RR: t^n<y\}$. Is $E$ bounded above? obviously, $1+y$ is an upper bound. Is it non empty? Of course, consider $t=\frac{y}{1+y}<y$. Hence, $E$ has a supremum $u$. Claim: $u^n=y$. Suppose not. Let $u^n<y$. We want to find an $h\in \RR^+$ such that $(u+h)^n<y$ so that a contradiction can be raised ($u+h$ cannot be in the set). In effect we want to show that $(u+h)^n-u^n<y-u^n$. Recall the identity: $p^n-q^n=(p-q)(p^{n-1}+qp^{n-2} \dots + q^{n-1})$. If $p>q$, we have $p^n-q^n<n(p-q)(p^{n-1})$. Therefore: $(u+h)^n-u^n\leq n(h)(u+h)^{n-1}$. We want $h$ so that $n(h)(u+h)^{n-1}<y-u^n \implies h< \frac{y-u^n}{n(u+h)^{n-1}}.$ Choose $h<1$, which would mean $\frac{y-u^n}{n(u+1)^{n-1}}<\frac{y-u^n}{n(u+h)^{n-1}}$. Now simply choose such an $h$ such that $h<\frac{y-u^n}{n(u+1)^{n-1}}<\frac{y-u^n}{n(u+h)^{n-1}}$ which is possible from density.\\
Suppose now that $u^n>y$, we need an $h$ so that $(u-h)^n>y$ or $-(u-h)^n<-y$, which would mean that $u-h$ is the actual supremum, contradicting the assumption. 
Therefore, we have to show that $u^n-(u-h)^n<u^n-y$. From the identity, we have that $u^n-(u-h)^n \leq n(h)(u)^n$. It would suffice if we find an $h$ so that $nhu^{n-1}<u^n-y$ or $h<\frac{u^n-y}{nu^{n-1}}$. Again, from Density theorem this is possible.\\
Uniqueness, once existence is established, is trivial since, if $q_1>q_2, (q_1)^n>(q_2)^n$.

}
\fact{\begin{enumerate}
    \item $n>0, q>0$ and $r=\frac{m}{n}=\frac{p}{q}$, then $({b^m})^{\frac{1}{n}}=({b^p})^{\frac{1}{q}}$
    \item $x^{({p+q})}=x^px^q$ for $p,q \in \QQ$
\end{enumerate}}
\thmp{Results regarding powers}{Suppose $b>1$. If $x \in \mathbb{R}$, define $B(x):=\{b^t: t \in \QQ: t \leq x\}$. Then $sup(B(r))=b^r$ if $r$ is a rational number.}{Of course, the set is bounded and non-empty, hence, has a supremum. It is clear that $b^r$ cannot be $<sup(B(r))$ because if so, there would exist $t \in \QQ, t \leq r$ such that $b^r < b^t$. Absurd. So $b^r \geq sup(B(r))$. It also can't be strictly greater, since $b^r$ is in the set itself, so it can't exceed its supremum. Hence,$Sup(B(r))=b^r$.}

With the previous result in mind as motivation, we define the following:

\defn{(Real "raised" to Reals)}{Given $b>1$, we define $b^x := sup (B(x):=\{b^t: t \in QQ: t \leq x \in \RR\})$}

\thmp{}{$b^x b^y =b^{x+y}$ for all $b>1, x,y \in \RR$}{
$$b^x:= sup(\{b^p:p\in\QQ: p \leq x\})$$
$$b^y:= sup(\{b^q:q\in\QQ: q \leq y \})$$
$$b^{x+y}:=sup(\{b^t:t\in\QQ: t \leq x+y\})$$
Suppose that $b^x b^y < b^{x+y}$. Then, $\exists q \in \QQ, q<x+y$ such that $b^x b^y<b^q$. Suppose WLOG $x<y$. Choose a $t \in Q^+$ such that $q-x<t<y$. This means, $q<x+y$ as we know, but also, $q-t<x$ and $t<y$. We now have $b^x b^y \leq b^{q-t+t}=b^{q-t}b^{t}$ where $q-t<x$ and $t<y$. Absurd.\\
Now assume $b^x b^y > b^{x+y}$. This implies $b^x > \frac{b^{x+y}}{b^y} \implies \exists q<x$ such that $b^q >\frac{b^{x+y}}{b^y} \implies b^y> \frac{b^{x+y}}{b^q} \implies \exists p<y$ such that $b^p>\frac{b^{x+y}}{b^q} \implies b^pb^q=b^{p+q}>b^{x+y}$ but $p<y$ and $q<x \implies p+q<x+y$. Absurd. So $b^xb^y=b^{x+y}$}

\thmp{Existence of Log}{Let $b>1$, $y>0$, then,$\exists$ a unique $x\in \mathbb{R}$ such that $b^x=y$}{Consider $E:=\{x\in\RR : b^x \leq y\}$. The claim is that $Sup(E)=z$ exists and $b^z=y$.\\
\textbf{Case 1-$y \geq b >1$ }:\\
It is obvious that in this case $E$ is non empty. Suppose that, it is unbounded. i.e, $\forall n \in \NN, b^n \leq y$. Since $b>1$,$b=1+\delta$ for some $\delta>0$. $\implies b^n=(1+\delta)^n=1+n\delta+\frac{n(n-1)}{2}\delta^2 \dots \leq y, \forall n \in \NN$. i.e, $1+n\delta< y, \forall n \in \NN$. This would be absurd, obviously. Hence, $\exists n_0 \in \NN$ such that $b^{n_0}>y$, and obviously $\forall n \geq n_0$. Therefore, in this case, $E$ is bounded and non empty, hence has a supremum $z=sup(E)$.\\
\textbf{Case 2- $b >y$}:\\
\textit{Sub Case 1: $b>y>1$}:\\
Boundedness is clear here. We claim that $\exists n_0\in \NN$ such that $y^{n_0}\geq b$ or $y \geq b^{\frac{1}{n_0}}$. Suppose not, i.e $\forall n \in \NN, y^n<b \implies (1+\delta)^n<b \implies 1+n\delta<b \forall n\in \NN$. Again, this is absurd. Hence, $\exists n_0$ such that $y>b^{\frac{1}{n_0}}$. Hence, it is bounded and non empty.\\
\emph{Sub Case 2: $b>1>y$}:\\
Boundedness is clear here as well. Since $y<1$, $y^{-1}>1$, and say $z=y^{-1}$. Does $\exists r_0 \in \QQ $ such that $ b \geq z^{r_0}$? If $z \geq b>1$, from the proof of case-1, $\exists r_0 \in \QQ$ such that $b \geq z^{r_0} \implies b \geq y^{-r_0} \implies y \leq b^{-\frac{1}{r_0}}$. If $b>z>1$, then that $r_0=1$. From here we see that $b>y^{-1} \implies b^{-1}<y$.
\\\\In all cases. Supremum exists for $E$. Call it $s$\\\\
Does $b^s=y$? suppose not, i.e, let $b^s<y$. We want to establish a number $s+z_0>s$ such that $b^{s+z_0}<y$ which would lead to contradiction since $s$ is supposed to be the supremum of $E$.\\
$\exists \delta \in \RR^+$ such that $b^s+(\delta)b^s=b^s(1+\delta)<y$ from density. We need a $q \in \QQ^+$ such that $b^q<1+\delta$. We know that $b>1$ and $1+\delta>1$, so either from case 1 where $1+\delta \geq b>1$, or from case 2 subcase 1 where $b>1+\delta>1$, we can find such a $q$. Hence, $b^sb^q=b^{s+q}<b^s(1+\delta)<y$. This would be absurd.\\
Consider the case where $b^s>y$. From density, $\exists \delta \in \RR^+$ such that $b^s>y+\delta \implies b^s>y+\delta_0 y \implies b^s >y(1+\delta_0$ for some $\delta_0$. This means that $b^s \frac{1}{1+\delta_0}>y$. We need to find, again, a positive rational such that $b^q<1+\delta_0$. From the previous analysis, it can be done. Hence, $b^{s-q}>y$, which means that for every $z \in \RR$ such that $z>s-q$, we have that $b^z>y$. This means that $s-q$ is an upper bound for $E$, which is absurd. Hence, $b^s=y$.




}
\newpage
\subsection{The Complex field $\CC$}
\defn{}{We define $\CC $ as the set of all ordered pairs in $\mathbb{R}^2$ with the following additional properties: \begin{enumerate}
    \item $x=(a_1,b_1), y=(a_2,b_2)$ with $a_1,b_1,a_2,b_2 \in \RR$, then $x+y$ is defined as $(a_1+a_2,b_1+b_2)$
    \item $x=(a_1,b_1), y=(a_2,b_2)$ with $a_1,b_1,a_2,b_2 \in \RR$, we define multiplication $xy$ as $(a_1a_2-b_1b_2-a_1b_2+a_2b_1)$
\end{enumerate}}
This set $\CC$ with $+$ and juxtaposition obey field axioms with $(0,0)$ the additive identity, and $(1,0)$ the multiplicative one. \\
For consistency, we define $\frac{1}{x}:=(\frac{a}{a^2+b^2},\frac{-b}{a^2+b^2}),$ where $x=(a,b)$.
\defn{}{\begin{enumerate}
    \item (Conjugate) If $z=(a,b)$, the conjugate $\bar z=$ is defined as $\bar z=(a,-b).$
    \item (Mod) If $z=(a,b)$, the mod of $z$, $|z|$ is defined as $(z \bar z)^{\frac{1}{2}}$

    
\end{enumerate}}
\fact{Some facts: \begin{enumerate}
    \item $\bar(x+y)=\bar x +\bar y$
    \item $\bar(zw)=\bar z \bar w$
    \item $z \bar z \geq 0$ and $=(a^2+b^2,0)$
    \item We can identify $\RR$ as a subset of $\CC$ by setting $a \in \RR$ to be $(a,0)$ in $\CC$.
    \item $|zw|=|z||w|$
    \item $|z+w|\leq|z|+|w| $
    \item $|Re(z)| \leq |z|$
\end{enumerate}}
\newpage 
\thmp{Cauchy-Schwartz Inequality}{If $a_1, a_2, \cdots a_n$ and $b_1,b_2 ,\cdots, b_n$ are numbers in $\RR^+$, then $$\big(\sum_{i=1}^n a_i b_i\big)^2 \leq \sum_{i=1}^n(a_i)^2\sum_{j=1}^n(b_j)^2$$
Extending this theorem for $a_j$ and $b_j$ in the complex domain, we have
$$|\big(\sum_{i=1}^n a_i \bar b_i\big)|^2 \leq \sum_{i=1}^n|(a_i)|^2\sum_{j=1}^n|(b_j)|^2$$
}{Consider $\alpha=(a_1b_1+a_2b_2 \dots a_nb_n)^2=(a_1b_1+a_2b_2 \dots a_nb_n)(a_1b_1+a_2b_2 \dots a_nb_n)$ which is $((a_1b_1)^2+(a_2b_2)^2 \dots (a_nb_n)^2)+K$
where $K$ is given by 
$$\begin{matrix}
    0 & +a_1b_1a_2b_2 &+a_1b_1a_3b_3 & \cdots & +a_1b_1a_nb_n \\
    a_2b_2a_1b_1 & +0 & +a_2b_2a_3b_3 & \cdots &+a_2b_2a_nb_n \\
    \vdots\\
    a_nb_na_1b_1 &+a_nb_na_2n_2 &+ a_nb_na_3b_3 & \cdots & +0
    
\end{matrix}$$
Consider $\beta=(a_1^2+a_2^2+\dots)(b_1^2+b_2^2 \dots)$
This would be$(a_1b_1)^2+(a_2b_2)^2 \cdots (a_nb_n)^2 + L$ where $L$ is given by:
$$\begin{matrix}
    0 & +a_1^2b_2^2 &+a_1^2b_3^2 & \cdots & +a_1^2b_n^2 \\
    a_2^2b_1^2 & +0 & +a_2^2b_3^2 & \cdots &+a_2^2b_n^2 \\
    \vdots\\
    a_n^2b_1^2 &+a_n^2b_2^2 &+ a_n^2b_3^2 & \cdots & +0
    
\end{matrix}$$

$$\beta-\alpha=\sum_{i=1}^n \sum_{j=i+1}^n (a_ib_j)^2+(a_jb_i)^2-\sum_{i=1}^n \sum_{j=i+1}2a_ia_jb_ib_j=\sum_{i=1}^n \sum_{j=i+1}(a_ib_j-b_ia_j)^2$$. Hence, $\beta=\alpha+$ some square term. Therefore $$\beta-\alpha \geq 0$$.
}

\thmp{Bernoulli's Inequality}{Given $x>-1$, $(1+x)^n \geq 1+nx$}{For $n=1$, it's trivially true. Assume it's correct for $n=n$. Consider $(1+x)^n(1+x)\geq(1+nx)(1+x)=1+x+nx+nx^2=1+x(n+1)+nx^2 \implies (1+x)^{n+1} \geq 1+(n+1)x$}
\thmp{AM-GM Inequality}{Given $a_1,a_2,\dots,a_n \in \mathbb{R}^+$, $$\bigg(\frac{S_n}{n}\bigg)^n=(\frac{a_1+a_2+\cdots +a_n}{n})^n \geq (a_1a_2\cdots a_n) $$}{For $a_1$, it is trivially true. Assume for $n=n$, and consider 
$$\bigg(\frac{S_{n+1}}{n+1}\bigg)^{n+1}=\bigg(\frac{S_n+a_{n+1}}{n+1}\bigg)^{n+1}\rightarrow $$
$$\bigg(\frac{\frac{nS_n}{n}+a_{n+1}}{n+1}\bigg)^{n+1} \rightarrow$$
$$\bigg(\frac{\frac{(n+1-1)S_n}{n}+a_{n+1}}{n+1}\bigg)^{n+1} =\bigg(\frac{\frac{(n+1)S_n-S_n}{n}+a_{n+1}}{n+1}\bigg)^{n+1} =$$
$$\bigg(\frac{\frac{(n+1)S_n}{n}-\frac{S_n}{n}+a_{n+1}}{n+1}\bigg)^{n+1} =\bigg(\frac{S_n}{n}+\frac{-\frac{S_n}{n}+a_{n+1}}{n+1}\bigg)^{n+1} = $$
$$\bigg(\frac{S_n}{n}\bigg)^{n+1}\bigg(1+\frac{-1+\frac{na_{n+1}}{S_n}}{n+1}\bigg)^{n+1} $$ From Bernoulli inequality,
$$\bigg(\frac{S_n}{n}\bigg)^{n+1}\bigg(1+\frac{-1+\frac{na_{n+1}}{S_n}}{n+1}\bigg)^{n+1} \geq \bigg(\frac{S_n}{n}\bigg)^{n+1}\bigg(1+(n+1)\frac{-1+\frac{na_{n+1}}{S_n}}{n+1}\bigg)=\bigg(\frac{S_n}{n}\bigg)^{n+1}\bigg(\frac{na_{n+1}}{S_n}\bigg)  $$
$$\geq \bigg(\frac{S_n}{n}\bigg)^n\bigg(a_{n+1}\bigg) \geq a_1a_2\cdots a_n a_{n+1} $$
}
\newpage
\subsection{Intervals on the Real Line}
\defn{Intervals}{\begin{enumerate}
    \item (Open Interval): $(a,b) \subset \RR := \{x \in \RR: a<x<b \}$
    \item (Closed Interval): $[a,b] \subset \RR:=\{x \in \RR: a \leq x \geq b \}$
\end{enumerate}}

\defn{Nested Intervals}{If $I_n:=[a_n,b_n]: n \in \NN$ is a sequence of intervals such that $I_n \subseteq I_{n-1} \cdots \subseteq I_1$, then $\{I_n \}$ is said to be a sequence of nested intervals. }
\thmp{Nested Interval Theorem}{Given a sequence of closed and bounded, and non empty nested intervals $\{I_n: n \in \NN \}, \ \exists \xi \in I_n \ \forall n \in \NN$ or equivalently, 
$\xi \in \big.\cap_{n=1}^{\infty} I_n$.}{Let $I_n=[a_n,b_n]$. From the definition, it is clear that $\{a_n\}$ is an increasing sequence of reals, while $\{b_n\}$ is a decreasing sequence. Moreover, from the non empty property of each interval, we have that $a_m < b_n, \forall n \in \NN, \forall m \in \NN$. This implies that the set of $\{a_n\}$ has a supremum $S_a$, while the set $\{b_n \}$ has an infimum $L_b$. $a_n \leq L_b \forall n \in \NN$ while $S_a \leq b_n \forall n \in \NN$. $a_n \leq $. Moreover, $a_n \leq S_a \forall n \in \NN$ while $L_b \leq b_n \forall n \in \NN$. since $a_n$ is a lower bound of $\{b_n\}$, $a_n \leq L_b$, and since $L_b$ is an upper bound of $\{a_n\}$, $a_n \leq S_a \leq L_b \leq b_n: \forall n \in \NN$. From Density, $\exists \xi \in [S_a,L_b]$ such that $\xi \in \cap_{i=1}^{\infty} I_i$
}


\newpage
\subsection{Decimal Expansions, and related results}
Every $x\in \RR$ can be written as an expansion in the following way:
\defn{Decimals}{Let $z \in \RR^+$ be given. Let $n_0$ be the "largest" integer such that $n_0 \leq z$. Let $n_1$ be the largest integer such that $n_0 +\frac{n_1}{10} \leq z$. As such, say $n_k$ is defined for some $k$. Let $n_{k+1}$ be the largest integer such that $n_0+\frac{n_1}{10^1}+\frac{n_2}{10^2} + \dots + \frac{n_k}{10^k} + \frac{n_{k+1}}{10^{k+1}} \leq z$. Consider the set of all such "finite sums", i.e, the set of all 
$$z_k=n_0+\frac{n_1}{10^1}+\frac{n_2}{10^2} + \dots + \frac{n_k}{10^k} + \frac{n_{k+1}}{10^{k+1}} \leq z$$. This set has a supremum and that is $z$ itself. We symbolically write $z=n_0.n_1n_2 \dots$
}
\thmp{}{The set $K=\{z_n: n\in \NN\}$ above, is bounded and non-empty, and Sup($K$)=$z$}{That it is non-empty and bounded is obvious. Suppose that $x=sup(K)$. Since $z$ is an upper bound, let us assume $x<z \implies \exists \xi \in \RR$ such that $\xi=z-x$. From Archimedean, choose a $k \in \NN$ such that $\frac{1}{10^k} < \xi \implies -\frac{1}{10^k} > -\xi$. This means that $z-\frac{1}{10^k}>z-\xi=x$. Consider one such $z_k \in K$, we can see that  $n_0+\frac{n_1}{10^1}+ \dots + \frac{n_k}{10^k} <x<z-\frac{1}{10^k} \implies n_0+\frac{n_1}{10^1}+ \dots + \frac{n_k+1}{10^k}<z$. But this would mean that for some $q \leq k \in \NN$, $n_k$ isn't the largest integer such that $z_k \leq z$.

}
\fact{The above definition is special in that, it ensures that decimal expansions are unique, since supremum's are unique. But the caveat is that, not all series' correspond to any real number as a decimal expansion. For example, in this definition: $0.999999\dots \neq 1$ since the unique decimal expansion for $1=1.00000$. So $0.9999\dots$ doesn't really correspond to any real number but we know obviously that it is $1$.  }
\thmp{}{$x \in \RR$ is rational $\iff$ x has either terminating, or repeating decimal expansion}{$\impliedby$)Obvious\\
$\implies$) Suppose $x=\frac{p}{q}$ for $p,q$ integers. Then $xq=p$. Let $k_0$ be the smallest integer such that $10^{k_0}p > q$. From Euclid's algorithm, we have $$10^{k_0}p=z_0q +r_0 \implies \frac{p}{q}=\frac{z_0}{10^{k_0}}+\frac{\frac{r_0}{q}}{10^{k_0}} $$ with $|r_0|<q$. Choose the smallest $k_1 \in \NN$ such that $10^{k_1}r_0>q $
Now consider $10^{k_1}r_0$, again we have $10^{k_1}r_0=z_1q+r_1$ with $|r_1| <q$. Thus $\frac{r_0}{q}=\frac{z_1}{10^{k_1}}+\frac{r_1}{10^{k_1}q}$. This implies $$\frac{p}{q}=\frac{z_0}{10^{k_0}} + \frac{z_1}{10^{k_0+k_1}}+\frac{r_1}{q 10^{k_0+k_1}}$$. We can keep going on as such, finding $k_n$, and applying Euclid's algorithm so that $$\frac{p}{q}=\frac{z_0}{10^{k_0}} + \frac{z_1}{10^{k_0+k_1}}+\frac{z_2}{10^{k_1+k_2+k_3}} \cdots + \frac{z_n}{10^{k_1+k_2 \dots +k_n}}+ \frac{r_n }{q10^{k_1+k_2...+k_n}}$$
Since for every $n \in \NN, |r_n| < q$, and $q \in \NN$, only finite amount of remainders are possible when dividing by $q$. Hence, at some point $p \in \NN$, $r_n=r_p$ for a previous $n \in \NN$. This means that, $10^{k_{p+1}}p=z_{p+1}q +r_{p+1} \implies \frac{r_p}{q}=\frac{z_{p+1}}{10^{k^{p+1}}} +\frac{r_{p+1}}{10^{k_{p+1}}q}$ $$\implies \frac{r_n}{q}=\frac{r_p}{q} \implies z_{n+1}=z_{p+1}$$ Hence, we can see that it is recurring.  
}
\end{document}
