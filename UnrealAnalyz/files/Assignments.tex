\documentclass[../Main.tex]{subfiles}

\begin{document}
\chapter{Assignment 1}
\lemp{}{If $\alpha$ is the supremum of a set $S$, then $-\alpha$ is the infimum
of the set $-S$ defined as $$-S:=\{-x: x \in S\} $$}{If $\alpha$ is the supremum,
then $\alpha\geq x, \forall x \in S$ and $\alpha \leq M, \forall M$ such that $M\geq x, \forall x \in S$. 
This means that $-\alpha \leq -x, \forall x \in S$ and $-\alpha \geq -M, \forall -M$ such that $-M \leq -x, \forall -x \in S $
If we re-notate the whole thing we have:

$-\alpha \leq z, \forall z \in -S$ and $-\alpha \geq L, \forall L$ such that
$L \leq z \forall z \in -S$. This is precisely the definition for infimum of $-S$, whence we see, we are done.

}
Recall the definitions:

$$LimSup(x_n):=inf(U_n: U_n:=sup(\{x_n,x_{n+1} \cdots \})) $$

$$LimInf(x_n):=sup(L_n: L_n:=inf(\{x_n,x_{n+1} \cdots \})) $$

\section{Problem 2}
Let $x$ be $limsup(-x_n)$. $$x=inf(U_n: U_n=sup\{-x_n,-x_{n+1} \cdots \})$$
$$\implies x=inf(U_n:U_n=-inf\{x_n,x_{n+1},\cdots\})$$
$$\implies x=inf(-inf\{x_n,x_{n+1}\cdots \}, -inf\{x_{n+1},x_{n+2}\cdots \},\cdots)$$ 
$$\implies x=-sup(inf\{x_n,x_{n+1}\cdots \},inf\{x_{n+1},x_{n+2}\cdots \},\cdots) $$
Whence, we are done.
\section{Problem 3}

$$Liminf(x_n)\leq (\text{every subsequential limit of} x_n) \leq Limsup(x_n) $$
Where $Liminf(x_n)$ is the infimum, and $Limsup(x_n)$ the 
supremum of the set of all subsequential limits of $x_n$
$$Liminf(y_n)\leq (\text{every subsequential limit of} y_n) \leq Limsup(y_n) $$
Where $Liminf(y_n)$ is the infimum, and $Limsup(y_n)$ the 
supremum of the set of all subsequential limits of $y_n$

Adding these two inequalities we get:

$Liminf(x_n)+Liminf(y_n)\leq$ $(\text{every subsequential limit of } x_n+ \text{every subsequential limit of } y_n)$ $\leq Limsup(x_n)+Limsup(y_n) $
\\\\ Since the set of all subsequential limits of $x_n+y_n$ falls as a subset of the sum of the set of all subsequential limits of $x_n$ and $y_n$ respectively, we have: 

$Liminf(x_n)+Liminf(y_n)\leq$ $(\text{every subsequential limit of } x_n+y_n)$ $\leq Limsup(x_n)+Limsup(y_n) $
We see now that $Liminf(x_n)+Liminf(y_n)$ is a lowerbound
for the set of all subsequential limits of $x_n+y_n$ 
which gives us $$Liminf(x_n)+Liminf(y_n)\leq liminf(x_n+y_n) $$
Similarly we see that $Limsup(x_n)+Limsup(y_n)$ is an upperbound
for the set of all subsequential limits of $x_n+y_n$
which gives us $$Limsup(x_n+y_n) \leq Limsup(x_n)+Limsup(y_n) $$

Equality of (I) holds when the smallest subsequential limit of $x_n+y_n$ is the sum
of the smallest possible subsequential limits of $x_n$ and $y_n$ respectively. 

Similarly, (II) equailty holds when the largest subsequential
limit of $x_n+y_n$ is equal to the sum of the smallest
subsequential limits of $x_n$ and $y_n$ respectively.

\section{Problem 4}
Suppose $\forall n \geq N$, we have $x_n \leq y_n$. 
It is clear to see that for every $n \geq N$, $U_n^x=sup(x_n,x_{n+1},\cdots)\leq 
U_n^y=sup(y_n,y_{n+1},\cdots)$. We have, in other words:
$U_n^x \leq U_n^y$ for all $n \geq N$. Hence, $Limsup(x_n)\leq limsup(y_n)$.
\\\\ Again consider $\forall n \geq N$, $x_n \leq y_n$
This means that $\forall n \geq N$, $L_n^x= inf(x_n,x_{n+1}\cdots) \leq y_n$
Since for a given $n$, we have $L_n^x \leq y_n$, and owing 
to the fact that $L_n^x$ is a monotone increasing sequence, we
see:
$$L_n^x\leq y_n $$
$$L_n^x \leq L_{n+1}^x \leq y_{n+1} $$
$$\vdots $$
Therefore, we see that $L_n^x \leq inf\{y_n,y_{n+1}\cdots\}=L_y^x$.
From here we can conclude that $Liminf(x_n) \leq Limsup(y_n)$

\section{Problem 5}
Consider $(x_n)^{\frac{1}{n}}$ and $\frac{x_{n+1}}{x_n}$. Where $x_n$ is a positive, bounded sequence.
Let $V=\{v \in \RR: \exists \ n_v \in \NN : \ \forall n \geq n_v, \ x_n \leq v^n \}$
and $V*=\{v \in \RR: \exists \ n_v \in \NN : \ \forall n \geq n_v, \ \frac{x_{n+1}}{x_n} \leq v \}$

Consider arbitrary $v \in V$. We have an $n_v$ so that $\forall n \geq n_v$
we have $x_n \leq v^n \implies x_{n+1} \leq v^{n+1} \implies \frac{x_{n+1}}{x_n}\leq v$.
Hence, we see that this $v \in V*$ aswell, which gives $V \subseteq V*$.

Consider arbitrary $v \in V*$. We have $n_v$ so that
$\forall n \geq n_v$, we have $\frac{x_{n+1}}{x_n} \leq v$

$$\frac{x_{n_v+1}}{x_{n_v}}\leq v$$
$$\frac{x_{n_v+2}}{x_{n_v+1}}\leq v$$
$$ \vdots$$
$$\frac{x_{v_v+(n-n_v)}}{x_{n-1}} \leq v$$
Multiply all these equations. We then get:
$$\frac{x_n}{x_{n_v}}\leq v^{n-n_v} $$
which gives, $\forall n \geq n_v$, $$x_n \leq x_{n_v}v^n v^{-n_v} \implies (x_n)^{1/n} \leq x_{n_v}^{1/n}v (v^{-n_v})^{1/n} $$
We see that, for large enough $n$, the RHS can be made to go below $v$. Whence, we 
see that beyond a certain $n$,  $(x_n)^{1/n}\leq v$. Hence,
$v$ that was initially assumed to be in $V*$, is shown to 
exist in $V$. Therefore, $V=V*$. Hence, $inf(V)=inf(V*)$ which
means $limsup((x_n)^{1/n})=limsup(\frac{x_{n+1}}{x_n})$.

The very same argument can be re-run, by interchanging the $\leq$-s with $\geq$-s to conclude that
$liminf((x_n)^{1/n})=liminf(\frac{x_{n+1}}{x_n})$. Therefore, if $\frac{x_{n+1}}{x_n}$
converges, then we see $(x_n)^{1/n}$ also does. (See Notes for more info)
\end{document}