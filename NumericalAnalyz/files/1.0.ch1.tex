\documentclass[../Main.tex]{subfiles}
\begin{document}
\chapter{Harry Potter and Basics of Numerical Analysis}

\intro{
	We'll go through the Numerical aspects of Taylor's theorem, Computer Arithmetic and Errors
}

\section{Preliminaries}

refer to relevant sections of unreal analyz for now I'll update this later.

\defn{Sequence}{A sequence is a function that associates the real number $a_n$ for each natural number $n$. The notation \(\{a_n\}\) is used to denote the sequence, i.e. \(\{a_n\}\) $\coloneq a_1, a_2, \dots, a_n, a_{n+1}, \dots$}

\defn{Convergence of sequences}{Let \(\{a_n\}\) be a sequence of real number s and a be a real number. The sequence \(\{a_n\}\) is said to converge to a, written as
	\begin{equation}
		\lim_{n \to \infty} a_n = a, \text{(or $a_n$ $\to$ a, as n $\to \inf$)}
	\end{equation}
	if for every $\epsilon > 0$, there exists a natural number $\mathbb{N}$ such that,
	\begin{equation}
		|a_n - a| < \epsilon, \text{whenever $n \geq \mathbb{N}$}
	\end{equation}
}

\thmp{Sandwich Theorem}
{Let \(\{a_n\}\), \(\{b_n\}\), \(\{c_n\}\), be sequences of real numbers such that
	\begin{enumerate}
		\item there exists an $n_0$ $\in$ $\mathbb{N}$ such that for every $n \geq n_0$, the sequences satify the inequality $a_n \leq b_n \leq c_n$ and
		\item $\lim_{n \to \infty} a_n$ = $\lim_{n \to \infty} c_n$ = a, for some real number a.
	\end{enumerate}}
{to be added later}

\defn{}{following definiations to be added here
	\begin{enumerate}
		\item bounded sequence
		\item convergence of sequence
		\item sub-sequences
		\item monotonic sequences
		\item Increasing sequences
		\item decreasing sequences
	\end{enumerate}
}

\thmp{Bolzano-Weierstrass Theorem}{just refer unreal analyz at this point}{}

\defn{Limits}{ A function $\mathcal{f}$}

\section{Taylor's theorem and numerical approximations}

\thmp{Taylor's theorem}
{ if f $\in$ $c^n$ $[a,b]$, and $f^{(n+1)}$ exists on the open interval (a, b), then for any points $x_0$ and $x$ in the closed interval [a,b],
	\begin{equation}
		f(x) = P_n(x) + R_n(x)
	\end{equation}
	where $p_n(x)$ (called taylor's polynomial) is
	\begin{equation}
		P_n(x) = f(x_0) + f'(x_0)(x-x_0) + \frac{f''(x_0)}{2!}(x-x_0)^2 + \dots + \frac{f^{(n)}(x_0)}{n!}(x-x_0)^n
	\end{equation}
	and for some point $\xi$ between $x$ and $x_0$, the remainder term (also called Truncation error) is
	\begin{equation}
		R_n(x) = \left |{\frac{f^{(n+1)}(\xi)}{(n+1)!}(x-x_0)^{n+1}}\right |
	\end{equation}
}{to be added later}

% #TODO: Recall the connection of radius of convergence and taylor theorem and include that here, eg can be why we can approximate log values with increasing number of terms in taylor series.

\corp{ If f $\in$ $C^{n+1}$ $[a, b]$, then the points $x$ and $x+h$ in the closed interval $[a, b]$,
	\begin{equation}
		f(x+h) = P_n(x) + R_n(h)
	\end{equation}
	where
	\begin{equation}
		P_n(x) = f(x) + f'(x)(x + h-x) + \frac{f''(x)}{2!}h^2 + \dots + \frac{h^k}{k!}f^k(x)
	\end{equation}
	\begin{equation}
		R_n(h) = \frac{h^{(n+1)}}{(n+1)!}f^{(n+1)}(\xi)
	\end{equation}
	in which the point $\xi$ lies between $x$ and $x+h$.
}
{ to be added later}

\defn{Taylor Series}
{Let $f$ be $C^\infty$ in a neighborhood of point $x_0$. The power series
	\begin{equation}
		\sum_{k=0}^{\infty} \frac{f^k(x_0)}{k!}(x-x_0)^k
	\end{equation}
	is called the Taylor's series of f about the point $x_0$.
}

\thmp{}
{
	Let $f$ $\in$ $C^{\infty}(I)$ and let $x_0$ $\in$ $I$. Assume that there exists an open neighborhood (interval) $N_{x_0} \subset I$ of the point $x_0$ and there exists a constant $M$ (may depend on $x_0$) such that,
	\begin{equation}
		\left | f^{k}(x) \right | \leq M^k
	\end{equation}
	for all $x \in N_{x_0}$ and $k = 0, 1, 2, \dots$ Then for each $x \in N_{x_0}$, we have
	\begin{equation}
		f(x) = \sum_{k=0}^{\infty} \frac{f^k(x_0)}{k!}(x-x_0)^k
	\end{equation}
}
{}

\section{Round off Errors and Computer Arithmetic}
In computer systems there is a constraint over accuracy of stored data, and hence it is always approximate, due to this errors are unavoidable.

\subsection{Decimal and Binary Number systems}
\textbf{Binary System} represents all number as a sum of multiples of integer powers of 2.\\
There are two digits (as coefficients); 0 and 1 (0 and 1 are called bits).\\\\
\textbf{Decimal System} represents numbers as a sum of multiples of integer powers of 10.\\
There are 10 digits (as coefficients), denoted by 0,1,2, $\dots$, 9.


\subsubsection{Converting a Decimal number into Binary format}
to be added later (yes I'm too lazy to look up a clean method to do this)

\subsubsection{Converting a Binary Number into Decimal format}
\exm
{Let the number in binary be 1101.11, convert it to decimal format.}
{
	The number 1101.11 will be expanded in binary as,
	\begin{equation*}
		(1101.11)_2 = 1 \cdot 2^3 + 1 \cdot 2^2 + 0 \cdot 2^1 + 1 \cdot 2^0 + 1 \cdot 2^{-1} + 1 \cdot 2^{-2} = 8 + 4 + 1 + 0.5 + 0.25 = (13.75)_{10}
	\end{equation*}
}
Clearly we can convert all the other binary representations into decimal similar to the above example.






\end{document}
